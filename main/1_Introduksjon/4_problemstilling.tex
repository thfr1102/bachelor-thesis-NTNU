\section{Problemstilling}
\label{sec:problemstilling}
Hovedproblemstillingen generelt for dette prosjektet er todelt. På en side skal vi bruke metode og verktøy for å finne fram til rotårsaken for tre ulike caser. Hvert case omfatter en hendelse eller et problem NTNU har, som de ønsker å eliminere. 

Det første caset innebærer nedlastning av opphavsrettsbeskyttet materiale ved hjelp av torrenting. 


CASE 2

CASE 3

I tillegg til dette har vi en underliggende problemstilling knyttet til alle tre casene hvor vi redegjør for hvilke metoder og verktøy som fungerer best innenfor informasjonssikkerhet. Denne vurderingen gjøres fortløpende og dokumenteres til senere bruk. Utover de tre casene skal vi bruke dokumentasjon og erfaringer til å utarbeide et dokument for gjennomføring av rotårsaksanalyse for informasjonssikkerhet. Dette skal inneholde informasjon om fremgangsmåte og metode, og dokumentasjon av verktøy som fungerer bra i denne sammenhengen. Dokumentet skal kunne brukes aktivt av fagmiljøet når det skal utføres ytterligere rotårsaksanalyser.

Gjennom prosjektet og rapporten ønsker vi å svare på følgende spørsmål:

\begin{itemize}
    \item Hva er rotårsaken til at studenter laster ned opphavsrettsbeskyttet materiale
    \item Hvordan kan en innføre tiltak som fjerner problemet med ulovlig nedlastning?
    \item CASE 2
    \itme CASE 3
    \item Hvilke metoder og verktøy som ofte brukes i rotårsaksanalyse, lønner seg mest å bruke innen informasjonssikkerhet?
\end{itemize}