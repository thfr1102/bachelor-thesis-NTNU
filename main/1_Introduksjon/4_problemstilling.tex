\section{Problemstilling}

\label{sec:problemstilling}
Hovedproblemstillingen for dette prosjektet er todelt. Den ene delen skal dreie seg om bruk av metode og verktøy for å finne frem til rotårsaken for tre ulike caser. Hvert case omfatter en hendelse eller et problem NTNU har, som de ønsker å eliminere. De tre problemene er som følger: 

\begin{enumerate}
    \item Ulovlig fildeling på universitetsnettet til NTNU
    \item Kompromitterte brukerkontoer ved NTNU
    \item Misbruk av NTNU sine ressurser til utvinning av kryptovaluta
\end{enumerate}

I tillegg til dette har vi en underliggende problemstilling knyttet til alle tre casene hvor vi redegjør for hvordan rotårsaksanalyse kan brukes innenfor informasjonssikkerhet. Denne vurderingen gjøres etter gjennomføring av casene for å se hvordan metoden og verktøyene fungerte. Dette dokumenteres til senere bruk.

Utover de tre casene skal vi bruke dokumentasjon og erfaring fra disse til å utarbeide retningslinjer for gjennomføring av rotårsaksanalyse for informasjonssikkerhet. Dette dokumentet skal inneholde informasjon om metode, fremgangsmåte og dokumentasjon av verktøy som fungerer bra i denne sammenhengen. Dokumentet skal kunne brukes aktivt av fagmiljøet når det skal utføres ytterligere rotårsaksanalyser. Derfor vil retningslinjene inneholde en deler som er fra hovedrapporten, dette dokumentent skal kunne stå for seg selv.

I rapporten ønsker vi å redegjøre følgende forskningsspørsmål:

\begin{itemize}
    \item Hva er rotårsaken til at studenter laster ned opphavsrettsbeskyttet materiale?
    \item Hva er rotårsaken til at brukerkontoer ved NTNU blir kompromittert?
    \item Hva er rotårsaken til misbruk av NTNU sin infrastruktur til utvinning av kryptovaluta?
    \item Hvor godt fungerer rotårsaksanalyse innen informasjonssikkerhet?
    \item Lønner det seg å benytte rotårsaksanalyse i informasjonssikkerhetssammenheng?
    \item Hvilke metoder og verktøy som ofte brukes i rotårsaksanalyse, lønner seg mest å bruke innen informasjonssikkerhet?
\end{itemize}