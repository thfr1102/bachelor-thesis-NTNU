\section{Problemstilling}
\label{sec:problemstilling}
Hovedproblemstillingen for dette prosjektet er todelt. Den ene delen skal dreie seg om bruk av metode og verktøy for å finne frem til rotårsaken for tre ulike caser. Hvert case omfatter en hendelse eller et problem NTNU har, som de ønsker å eliminere. De tre problemene er som følger: 

\begin{enumerate}
    \item Fildeling av opphavsrettsbeskyttet materiale ved bruk av torrents
    \item Kompromitterte ansattkontoer ved NTNU
    \item Bruk av NTNU sine ressurser til utvinning av kryptovaluta 
\end{enumerate}

I tillegg til dette har vi en underliggende problemstilling knyttet til alle tre casene hvor vi redegjør for hvilke metoder og verktøy som fungerer best innenfor informasjonssikkerhet. Denne vurderingen gjøres fortløpende for hvert case og dokumenteres til senere bruk.

NB! Utover de tre casene skal vi bruke dokumentasjon og erfaringer til å utarbeide et dokument for gjennomføring av rotårsaksanalyse for informasjonssikkerhet. NB! 

Dette skal inneholde informasjon om fremgangsmåte og metode, og dokumentasjon av verktøy som fungerer bra i denne sammenhengen. Dokumentet skal kunne brukes aktivt av fagmiljøet når det skal utføres ytterligere rotårsaksanalyser.

Gjennom prosjektet og rapporten ønsker vi å redegjøre følgende forskningsspørsmål:

\begin{itemize}
    \item Hva er rotårsaken til at studenter laster ned opphavsrettsbeskyttet materiale?
    \item Hvordan fungerer rotårsaksanalyse i et case som omhandler ulovlig fildeling?
    \item Hva er rotårsaken til at brukerkontoer ved NTNU blir kompromittert?
    \item Hvordan fungerer rotårsaksanalyse i et case som omhandler kompromitterte kontoer. 
    \item CASE 3
    \item Hvilke metoder og verktøy som ofte brukes i rotårsaksanalyse, lønner seg mest å bruke innen informasjonssikkerhet?
\end{itemize}