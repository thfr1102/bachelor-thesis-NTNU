\chapter*{Akronymer og begreper}
\label{kap:akronymer}

\section*{Akronymer}
\begin{description}
    \item[RCA] står for Root Cause Analysis, og er en metode for problemløsning.
    \item[ROS] står for Risiko- og sårbarhetsanalyse, og er en metode for å analysere risikoer og innføre tiltak for å føre dem ned til et akseptabelt nivå.
    \item[HPC] står for High Performance Computing, og er gjerne snakk om en regneklynge med høy ytelse, også kjent som en superdatamaskin.
    \item[BYOD] står for Bring Your Own Device, og innebærer alle enheter du tar med deg til jobb eller skole og bruker der.
    \item[DNS] står for Domain Name System og er en tjeneste som brukes for å oversette mellom domenenavn og IP-adresse på internett.
    \item[DHCP] står for Dynamic Host Configuration Protocol, og brukes for å dynamisk tildele IP-adresser på en nettverk etter behov.
    \item[ISMS] står for Information Security Management System, og er et system for å behandle og sette krav til ulike aspekter ved informasjonssikkerhet.
    \item[IAM] står for Identity and Access Management, og er et system som behandler og autoriserer dine autentiseringsdata. Brukes også til å sette krav til passord. 
    \item[SOC] Security Operation Center
    \item[2FA] står for to-faktor autentisering og brukes når man snakker om to autentiseringsfaktorer, som for eksempel passord og kodebrikke. 
\end{description}


\section*{Begreper}
\begin{description}
    \item[Aktiva] er et regnskapsmessig uttrykk for eiendeler eller rettigheter som har formueverdi.
    \item[Case] er en enhet, og brukes gjerne som en case-studie, som betyr studie av en enhet eller et tilfelle.
    \item[Signifikans] er et begrep som brukes for å beskrive sannsynligheten for at noe er et resultat av tilfeldigheter \cite{wiki:sig}.
    \item[Konfidensinterval] er en måte å angi feilmarginen av en måling eller en beregning på. Et konfidensintervall angir intervallet som med en spesifisert sannsynlighet inneholder den sanne (men vanligvis ukjente) verdien av variabelen man har målt \cite{wiki:konfidens}.
    \item[Peer-to-peer] er en betegnelse på en måte å organisere ressursdeling på. I denne rapporten snakker vi spesifikt om fildeling. 
    \item[Symptom] brukes i denne rapporten om en indikasjon på at et problem eksisterer. Et symptom kan i denne sammenhengen være notifikasjoner om brudd på opphavsrett, tap av omløpsmidler eller sikkerhetsvarsler. 
\end{description}