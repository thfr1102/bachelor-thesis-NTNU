\chapter{Introduksjon}
\label{kap:introduksjon}
Dette kapittlet handler om hvordan dette prosjektet ble til. Vi gir en kort beskrielse av bakgrunnen til prosjektet. Vi presenterer deretter problemstilling med forskningsspørsmål. Vi beskriver også vår motivasjon for oppgaven, rammer som var forhåndsbestemte og avgrensninger vi satt underveis. Helt til slutt inneholder dette kapittelet en oversikt over ønsket mål med dette prosjektet og kort oversikt over hva rapporten inneholder.


\section{Bakgrunn og introduksjon}
\label{sec:bakgrunn}
Root Cause Analysis, RCA, er en metode for problemløsning som brukes for å identifisere rotårsaken til et problem. RCA er et lite brukt verktøy innen informasjonssikkerhet, men er av økende betydning. Vanlig tilnærming til informasjonssikkerhetsstyring er å utføre en risiko-og sårbarhetsanalyse (ROS-analyse) for så å gjennomføre tiltak som fører risikoene til et akseptabelt nivå. En annen hyppig brukt tilnærming er hendelseshåndtering der en planlegger hvordan det skal responderes på hendelser etter de er inntruffet. Rotårsaksanalyse skiller seg fra disse ved å gå i dybden på problemet, kartlegge hva slags rotårsaker som står bak, og innføre tiltak for å fjerne disse helt. 

Vi tar for oss i denne oppgaven forskjellige verktøy som blir brukt innenfor RCA og se hvordan de fungerer innenfor informasjonssikkerhet. 