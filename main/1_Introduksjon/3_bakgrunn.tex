\chapter{Introduksjon}
\label{kap:introduksjon}
Formålet med dette kapittelet er å introdusere prosjektet . Det gis en kort beskrivelse av bakgrunnen til prosjektet, deretter presenteres problemstilling med forskningsspørsmål. Vi beskriver også vår motivasjon for oppgaven,  forhåndsbestemte rammer og hvilke avgrensninger vi satt underveis. Helt til slutt inneholder dette kapittelet en oversikt over prosjektmål og en kort oversikt over hva rapporten inneholder.


\section{Bakgrunn og introduksjon}
\label{sec:bakgrunn}
Root Cause Analysis (RCA), er en metode for problemløsning som brukes for å identifisere rotårsaken til et problem. RCA er et lite brukt verktøy innen informasjonssikkerhet, men er av økende betydning. Vanlig tilnærming til informasjonssikkerhetsstyring er å utføre en risiko-og sårbarhetsanalyse (ROS-analyse) for så å gjennomføre tiltak som fører risikoene til et akseptabelt nivå. En annen hyppig brukt tilnærming er hendelseshåndtering der en planlegger hvordan det skal responderes på hendelser etter de er inntruffet. Rotårsaksanalyse skiller seg fra disse ved å gå i dybden på problemet, kartlegge hva slags rotårsaker som står bak, og innføre tiltak for å fjerne disse helt. 

I dette prosjektet tar vi for rotårsaksanalyse metodikk og ser hvordan det fungerer innenfor informasjonssikkerhet. 