\section{Organisering av rapporten}
\label{sec:organisering_rapport}
Prosjektrapporten inneholder 8 kapitler i tillegg til vedlegg. Som vedlegg inkluderes delrapportene fra hvert case. 


\begin{description}
    \item [Kapittel 1: Introduksjon] inneholder innledende informasjon om prosjektets helhet.
    
    \item [Kapittel 2: Casebeskrivelser] inneholder detaljert oppgavebeskrivelse av hvert case vi skal analysere.
    
    \item [Kapittel 3: Teori] inneholder bakgrunn og teori om rotårsaksanalyse, samt sammenlikning med eksisterende metodikk innen informasjonssikkerhet.
    
    \item [Kapittel 4: Metode] inneholder beskrivelse av metoden vi benyttet når vi skulle tilnærme oss rotårsaksanalyse av informasjonssikkerhetshendelser. Kapittelet beskriver alle fasene i prosessen, samt verktøyene som ble benyttet. 
    
    \item [Kapittel 5: Gjennomføring av metode] inneholder en gjennomgang av valg av verktøy i hver fase, ulike spesifiseringer vi la til grunn og hvordan vi gjennomførte metoden i de ulike casene. 
    
    \item [Kapittel 6: Resultater og analyse fra Case 1: Ulovlig fildeling på universitetsnettet til NTNU] inneholder resultatene i hver fase fra det første caset.
    
    \item [Kapittel 7: Resultater og analyse fra Case 2: Kompromittertebrukerkontoer ved NTNU] inneholder resultatene i hver fase fra det andre caset.
    
    \item [Kapittel 8: Resultater og analyse fra Case 3: Misbruk av NTNU sin infrastruktur til utvinning av kryptovaluta] inneholder resultatene i hver fase fra det tredje caset.
    
    \item [Kapittel 9: Diskusjon] inneholder drøfting av empirien. Her stiller vi kritiske spørsmål til det materiale som er presentert tidligere i rapporten. 
    
    \item [Kapittel 10: Retningslinjer for bruk av RCA i informasjonssikkerhet] inneholder retningslinjer vi har utarbeidet på bruk av RCA-metode og verktøy innen informasjonssikkerhet, på bakgrunn av våre erfaringer gjennom de tre casene. 
    
    \item [Kapittel 11: Konklusjon] inneholder en konklusjon av problemstillingen og de forskningsspørsmålene som ble satt i starten av rapporten. Her kommer en oppsummering av de viktigste funnene satt i et helhetsperspektiv, samt forslag til videre arbeid. 
\end{description}