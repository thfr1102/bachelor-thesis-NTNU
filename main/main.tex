%% This document gives an example on how to use the ntnubachelorthesis
%% LaTeX document class.
%% Use oneside for PDF delivery and twoside for printing in a book style
%% use language english, norsk, nynorsk and one of the following shortenings
%%  ``BSP'' Bachelor i Spillprogrammering,\\
%%  ``BRD'' Bachelor i drift av nettverk og datasystemer,\\
%%  ``BIS'' Bachelor i Informasjonssikkerhet,\\
%%  ``BPU'' Bachelor i Programvareutvikling, \\
%%  ``BIND'' Bachelor i Ingeniorfad - data, \\
%%  ``BADR'' Bachelor i drift av datasystemer, \\
%%  ``BIT'' Bachelor i informatikk, \\
%%  ``BABED'' Bachelor i IT-støttet bedriftsutvikling.
%%   for example \documentclass[BIS,norsk,twoside]{ntnuthesis/ntnubachelorthesis}

\documentclass[BIS,norsk,oneside]{ntnuthesis/ntnubachelorthesis}

\usepackage{csvsimple}
\usepackage{booktabs}
\usepackage{gnuplottex}
\usepackage[T1]{fontenc}
\usepackage[utf8]{inputenc}     % For utf8 encoded .tex files because...
\usepackage[norsk]{babel}       % For Norwegian labeling
\usepackage{graphicx}           % For inclusion of graphics
\PassOptionsToPackage{hyphens}{url}
\usepackage{url}
\usepackage{hyperref}    % For cross references in pdf
\usepackage{tabularx}
\usepackage[table]{xcolor}
\usepackage{colortbl}
\usepackage{placeins}
\usepackage{pdfpages}
\usepackage{enumitem}
\usepackage{multirow}
\usepackage{lscape}
\usepackage{bigstrut}
\usepackage{verbatim}
\usepackage{float}
\usepackage{subfig}


\definecolor{darkgreen}{rgb}{0,0.5,0}
\definecolor{apricot}{RGB}{255, 217, 178}

\lstset{        basicstyle=\ttfamily,
                keywordstyle=\color{blue}\ttfamily,
                stringstyle=\color{darkred}\ttfamily,
                commentstyle=\color{darkgreen}\ttfamily,
}


%Typesetting of C++
\newcommand{\CPP}[0]{{C\nolinebreak[4]\hspace{-.1em}\raisebox{.1ex}{\small\bf +\hspace{-.1em}+\ }}}



%\newcommand{\comment}[1]{\textcolor{blue}{\emph{#1}}}  %% use of the colour and you can see how to use commands with parts \comment{so what}

%% The class files defines these two
%% \newcommand{\NTNU}{Norwegian University for Science and Technology} %

% you can create you one #define like structures using the \newcommand feature
% you can change behaviour using \renewcommand

\newcommand{\com}[1]{{\color{red}#1}} % supervisor comment
%\renewcommand{\com}[1]{} %remove starting % to remove supervisor comments
% This will appear in text \com{Lecuters comment} and be visible unless you uncomment
% the renewcommand line.

\newcommand{\todo}[1]{{\color{green}#1}} % items to do
%\renewcommand{\todo}[1]{} %remove starting % to remove items to do

\newcommand{\n}[1]{{\color{blue}#1}} % other comment
%\renewcommand{\n}[1]{} %remove starting % to remove notes

\newcommand{\dn}[1]{} % add the d to a note to say that you have finished with it.

\newcommand{\gj}{NTNU i Gj\o{}vik}


% Norwegian Characters,  needs the {} or to be separate from the next letters
% \o{}   \aa{}   \ae{}   so at the end of a word you can use \o  \aa   \ae
% \O{}   \AA{}   \AE{}   you can also just leave a space and latex will remove it
%    eg, NTNU i Gj\o vik  or NTNU i Gj\o{}vik


\graphicspath{ {bilder/} }

\begin{document}

\thesistitle{Root Cause Analysis for Information Security}
\thesisshorttitle{Rotårsaksanalyse for Informasjonssikkerhet} % use this if you have a very long title and want something shorter on the header pages
\thesisauthor{Philip Nyblom}
\thesisauthorA{Fredrik Theien}
\thesisauthorB{Thomas Huse}
\thesisauthorC{Ole Martin Søgnen}
\thesissupervisor{Tom Røise}
%\thesissupervisorA{john smith} %second supervisor

%\thesisOppgaveNo{9}
\firstnamea{First}

\nmtkeywords{Thesis, Latex, Template, IMT}
\nmtdesc{This is the short description of a bachelor thesis. It should contain a short introduction to the area of the thesis and what the thesis contributes to that area. This does not allow for paragraphs so you have to include the entire short description in a single paragraph.}

\nmtoppdragsgiver{Institutt for informasjonssikkerhet og kommunikasjonsteknologi, NTNU}
\nmtcontact{Gaute Wangen, gaute.wangen@ntnu.no, +47 907 08 338}




\thesisdate{\ntnubachelorthesisdate}
\useyear{16.05.2018}

\nmtappnumber{} %numebr of appendixes
\nmtpagecount{} %currently auto calculated but might be wrong




\thesistitleNOR{Rotårsaksanalyse for Informasjonssikkerhet}
\nmtkeywordsNOR{Norway, Norsk}
\nmtdescNOR{Dette b\o{}r være p\aa norsk, jeg tenkte jeg skulle legge til litt mer tekst 
\aa s\o{}rge for at det gikk over flere linjer. Jeg m\aa sjekke p\aa siden 
nummer Feild som det kanskje burde være kun de sidene uten 
vedlegg. Forel\o{}pig returnerer den siste siden av hele dokumentet. 
Hvis jeg ikke kan finne ut av det, vil jeg gi to innganger, gmtnumberpages og gmtappnumber. 
Det burde gj\o{}re jobben. (done in google translate so it is bad norwegian) }

 % this is the file which contains all the details about your thesis

\makefrontpages % make the frontpages

\chapter*{Forord} %the * means do not give the chapter a number
\label{kap:forord}



\tableofcontents
\addtocontents{toc}{\protect\setcounter{tocdepth}{1}}
\listoffigures
\listoftables

\chapter*{Akronymer og begreper}
\label{kap:akronymer}

\section*{Akronymer}
\begin{description}
    \item[RCA] står for Root Cause Analysis, og er en metode for problemløsning.
    \item[ROS] står for Risiko- og sårbarhetsanalyse, og er en metode for å analysere risikoer og innføre tiltak for å føre dem ned til et akseptabelt nivå.
    \item[HPC] står for High Performance Computing, og er gjerne snakk om en regneklynge med høy ytelse, også kjent som en superdatamaskin.
    \item[BYOD] står for Bring Your Own Device, og innebærer alle enheter du tar med deg til jobb eller skole og bruker der.
    \item[DNS] står for Domain Name System og er en tjeneste som brukes for å oversette mellom domenenavn og IP-adresse på internett.
    \item[DHCP] står for Dynamic Host Configuration Protocol, og brukes for å dynamisk tildele IP-adresser på en nettverk etter behov.
    \item[ISMS] står for Information Security Management System, og er et system for å behandle og sette krav til ulike aspekter ved informasjonssikkerhet.
    \item[IAM] står for Identity and Access Management, og er et system som behandler og autoriserer dine autentiseringsdata. Brukes også til å sette krav til passord. 
    \item[SOC] Security Operation Center
    \item[2FA] står for to-faktor autentisering og brukes når man snakker om to autentiseringsfaktorer, som for eksempel passord og kodebrikke. 
\end{description}


\section*{Begreper}
\begin{description}
    \item[Aktiva] er et regnskapsmessig uttrykk for eiendeler eller rettigheter som har formueverdi.
    \item[Case] er en enhet, og brukes gjerne som en case-studie, som betyr studie av en enhet eller et tilfelle.
    \item[Signifikans] er et begrep som brukes for å beskrive sannsynligheten for at noe er et resultat av tilfeldigheter \cite{wiki:sig}.
    \item[Konfidensinterval] er en måte å angi feilmarginen av en måling eller en beregning på. Et konfidensintervall angir intervallet som med en spesifisert sannsynlighet inneholder den sanne (men vanligvis ukjente) verdien av variabelen man har målt \cite{wiki:konfidens}.
    \item[Peer-to-peer] er en betegnelse på en måte å organisere ressursdeling på. I denne rapporten snakker vi spesifikt om fildeling. 
    \item[Symptom] brukes i denne rapporten om en indikasjon på at et problem eksisterer. Et symptom kan i denne sammenhengen være notifikasjoner om brudd på opphavsrett, tap av omløpsmidler eller sikkerhetsvarsler. 
\end{description}
%\chapter*{Kortfattet sammendrag}
\label{kap:kortfattet}

\chapter{Introduksjon}
\label{kap:introduksjon}
Formålet med dette kapittelet er å introdusere prosjektet . Det gis en kort beskrivelse av bakgrunnen til prosjektet, deretter presenteres problemstilling med forskningsspørsmål. Vi beskriver også vår motivasjon for oppgaven,  forhåndsbestemte rammer og hvilke avgrensninger vi satt underveis. Helt til slutt inneholder dette kapittelet en oversikt over prosjektmål og en kort oversikt over hva rapporten inneholder.


\section{Bakgrunn og introduksjon}
\label{sec:bakgrunn}
Root Cause Analysis (RCA), er en metode for problemløsning som brukes for å identifisere rotårsaken til et problem. RCA er et lite brukt verktøy innen informasjonssikkerhet, men er av økende betydning. Vanlig tilnærming til informasjonssikkerhetsstyring er å utføre en risiko-og sårbarhetsanalyse (ROS-analyse) for så å gjennomføre tiltak som fører risikoene til et akseptabelt nivå. En annen hyppig brukt tilnærming er hendelseshåndtering der en planlegger hvordan det skal responderes på hendelser etter de er inntruffet. Rotårsaksanalyse skiller seg fra disse ved å gå i dybden på problemet, kartlegge hva slags rotårsaker som står bak, og innføre tiltak for å fjerne disse helt. 

I dette prosjektet tar vi for rotårsaksanalyse metodikk og ser hvordan det fungerer innenfor informasjonssikkerhet. 
\section{Problemstilling}
\label{sec:problemstilling}
Hovedproblemstillingen generelt for dette prosjektet er todelt. På en side skal vi bruke metode og verktøy for å finne fram til rotårsaken for tre ulike caser. Hvert case omfatter en hendelse eller et problem NTNU har, som de ønsker å eliminere. 

Det første caset innebærer nedlastning av opphavsrettsbeskyttet materiale ved hjelp av torrenting. 


CASE 2

CASE 3

I tillegg til dette har vi en underliggende problemstilling knyttet til alle tre casene hvor vi redegjør for hvilke metoder og verktøy som fungerer best innenfor informasjonssikkerhet. Denne vurderingen gjøres fortløpende og dokumenteres til senere bruk. Utover de tre casene skal vi bruke dokumentasjon og erfaringer til å utarbeide et dokument for gjennomføring av rotårsaksanalyse for informasjonssikkerhet. Dette skal inneholde informasjon om fremgangsmåte og metode, og dokumentasjon av verktøy som fungerer bra i denne sammenhengen. Dokumentet skal kunne brukes aktivt av fagmiljøet når det skal utføres ytterligere rotårsaksanalyser.

Gjennom prosjektet og rapporten ønsker vi å svare på følgende spørsmål:

\begin{itemize}
    \item Hva er rotårsaken til at studenter laster ned opphavsrettsbeskyttet materiale
    \item Hvordan kan en innføre tiltak som fjerner problemet med ulovlig nedlastning?
    \item CASE 2
    \itme CASE 3
    \item Hvilke metoder og verktøy som ofte brukes i rotårsaksanalyse, lønner seg mest å bruke innen informasjonssikkerhet?
\end{itemize}
\section{Motivasjon}
\label{sec:motivasjon}
Oppdragsgiver ønsker å få en bedre forståelse for bruk av rotårsaksanalyse innen informasjonssikkerhet, og eliminere problemer tilknyttet oppdragsgivers ansvarsområder \footnote{Dette inkluderer blant annet: internett ved Sit Bolig, brukerkontoer ved NTNU og infrastrukturen til NTNU}. Prosjektet skal bistå til økt forståelse ved å undersøke verktøy- og metodebruk fra RCA innenfor Seksjon for Digital Sikkerhet sitt fagfelt. Prosjektet skal presentere en tiltaksplan som eliminerer noen av problemene oppdragsgiver har i sitt ansvarsområde. En motivasjon er derfor økt tilgang på ressurser i form av arbeidskraft for å løse problemer. 


Motivasjonen vår innebærer et ønske om økt kompetanse knyttet til risikoer og hendelser, og metodebruk for å håndtere disse. Vi ønsket dermed en oppgave som var styringsrelatert fremfor teknisk. Denne oppgaven handlet om noe som for oss var en ny tilnærming til informasjonssikkerhetsstyring. Vi ønsket derfor å være med å vurdere bruken av RCA i informasjonssikkerhet og om det burde benyttes i tillegg til ROS-analyse og hendelseshåndtering. 
\section{Rammer}
\label{sec:rammer}
Her defineres de forhåndsbestemte rammene som er satt. 

\begin{itemize}
    \item Prosjektet skal være aktivt i perioden 8. januar til 16. mai.
    \item Prosjektet er delt inn i tre hendelser som skal analyseres.
    \item Gruppen skal møtes hver hverdag.
    \item Ukentlig møte med veilleder settes til torsdager kl. 13.30-14.00.
    \item Rapporten skal bli skrevet i ``LaTeX'', gjennom tjenesten ``ShareLaTeX''.
\end{itemize}
\section{Avgrensing}
\label{sec:avgrensing}
Innen rotårsaksanalyse finnes det flere verktøy og metoder, men på grunn av anbefalinger fra oppdragsgiver avgrenser vi oss til metoden og verktøyene beskrevet i boka ``Root Cause Analysis: Simplified Tools and Techniques - second edition'' av Bjørn Andersen og Tom Fagerhaug \cite{RCA}. Generelt for prosjektet vil vi følge føringer fra Seksjon for Digital Sikkerhet.

\noindent Spesielt for hvert case gjelder:

\textbf{Case 1:}
Caset avgrenses til studenter i Gjøvik som leier studenthybel fra Sit. Datainnsamlingen inkluderer ikke ansatte. 

\textbf{Case 2:}
Informasjonsinnsamlingen avgrenses til kun 167 personer av de som har blitt kompromittert. 

\textbf{Case 3:}
Ingen.

\section{Prosjektmål}
\label{sec:prosjektmaal}
Målet for prosjektet er å gå i dybden på tre ulike caser som hver omhandler en unik hendelse eller et problem NTNU har. Prosjektmålene deles inn i effektmål og resultatmål.

\subsection{Effektmål}
Effektmålene er det oppdragsgiver ønsker å oppnå med oppgaven og rapporten etter den er utredet og levert.
\begin{itemize}
    \item Eliminere problemene beskrevet i de tre casene helt eller delvis ved implementering av tiltak beskrevet i rapporten
    \item Økt forståelse for bruk av RCA i informasjonssikkerhet
    \item Finne ut hvor fordelaktig det er å bruke RCA i informasjonssikkerhet og hvilke verktøy som fungerer best til dette
\end{itemize}

\subsection{Resultatmål}
Resultatmålene beskriver det prosjektet skal oppnå i det prosjektperioden er over, og oppgaven er utført og levert.

\begin{itemize}
    \item Alle tre caser er utført og har en individuell rapport hver
    \item En oversikt over hovedfunn fra tre caser med RCA
    \item Dokumentere bruk av RCA i informasjonssikkerhet, med forklaringer og konkluderinger av metode og verktøy
\end{itemize}
\section{Organisering av rapporten}
\label{sec:organisering_rapport}
Prosjektrapporten inneholder 7 kapitler i tillegg til vedlegg.


\begin{description}
    \item [Kapittel 1: Introduksjon] inneholder grunnleggende informasjon om prosjektet.
    \item [Kapittel 2: Casebeskrivelser] inneholder detaljert beskrivelse av hvert case vi skal analysere.
    \item [Kapittel 3: Teori] inneholder bakgrunn og teori om rotårsaksanalyse, samt sammenlikning med eksisterende metodikk innen informasjonssikkerhet.
    \item [Kapittel 4: Metode] inneholder beskrivelse av metoden vi benyttet når vi skulle tilnærme oss rotårsaksanalyse av informasjonssikkerhetshendelser. Kapittelet beskriver alle fasene i prosessen, samt verktøyene som ble benyttet i hver fase. 
    \item [Kapittel 5: Resultater] inneholder hovedfunnene i de tre casene vi analyserte.
    \item [Kapittel 6: Diskusjon] inneholder diskusjon om resultatene.
    \item [Kapittel 7: Konklusjon] inneholder oppgavens konklusjon.
\end{description}
\chapter{Casebeskrivelser}
\label{kap:casebeskrivelser}
Rotårsaksanalyse er foreløpig lite brukt i forbindelse med informasjonssikkerhet, og oppdragene som blir sett på i denne rapporten er gjort ved hjelp av rotårsaksanalyse metodikk. Dette for å få en større forståelse for rotårsaksanalyse i sammenheng med det digitale og spesielt da informasjonssikkerhet.

\section{Case 1: Ulovlig fildeling på skolenettet}
\label{sec:case_fildeling}
Ved NTNU Gjøvik er det NTNU som har ansvar for nettet hos studenthyblene studentene leier i byen. Dette fører med seg et problem i form av mye piratnedlastning på skolenettet. Oppgaven vår i dette caset var å finne, ved hjelp av rotårsaksmetodikken, hvordan skolen skal kunne stoppe studentene fra å laste ned opphavsrettighetsbeskyttet materiale.

\subsubsection{Oversikt over oppdraget}
Advokater til diverse filmselskaper ser etter IP-er til personer som laster ned deres opphavsrettsbeskyttede materiale, og sender disse personene mail. NTNU for flere hunde slike mailer i måneden, bortsett fra om sommeren da studenter er på ferie. Piratnedlastning foregår ved hjelp av en protokoll som heter bit-torrent protokkollen og den bruker peer-to-peer teknologi.  Hvis disse opphavsrettshaverne finner ut at de skal håndheve brevene de sender, kommer NTNU til å være i en dårlig posisjon, der de per idag ikke gjør noe med de mange brevene de får. Grunnen til at de ikke gjør noe er de store mengdene og at det hadde vært en fulltidsjobb i seg selv å håndtere dem.
\section{Case 2: Kompromitterte kontoer}
\label{sec:case_kontoer}
NTNU er et stor universitet med mye forskjellige folk fra forskjellige bakgrunner og fagkunnskap om vidt forskjellige ting. NTNU har som følge av denne mengden med folk et problem med at kontoer kan komme på avveie. Disse kontoene blir brukt til mye forskjellig, fra spam til å hente ned store mengder med forskningsartikler. Måten disse kontoene kommer på avveie er noe uvisst, men det spekuleres i at det er mye phising. Vår oppgave i dette oppdraget var, å bruke rotårsaksanalyse til å finne og eliminere rotårsaken til at kontoer blir kompromitert.

\subsubsection{oversikt over oppdraget}
NTNU har siden ca.2005 fått 5415 kontoer kompromitert disse kontoene ble funnet i en stor datadump i desember 2017, av disse 5415 kontoene var 101 av dem fremdeles aktive, det vil si at folk kunne finne disse 101 folkene sine kontoer og logge seg på på skolenettet med brukerinformasjonen deres. Dette koster skolen en del penger i form av forskningartiklene som bli lastet ned, der hver artikkel koster skolen ---insert cost of each article when they get crawled.-----. Et annet punkt som gjør det å kompromitere kontoer ved skolen lukerativt er at disse kontoene kan bli solgt på brukerinformasjons marked, der de behandles som ferskvare.
\section{Case 3: Misbruk av infrastuktur til utvinning av kryptovaluta}
\label{sec:Misbruk av infrastuktur til utvinning av kryptovaluta}


\chapter{Teori og tidligere arbeid}
\label{kap:teori}

\section{Teori}
Rotårsaksanalyse er en fremgangsmåte for å finne roten til et problem og eliminere det. RCA analyserer de underliggende faktorene og bruker årsakene til å finne roten til problemet. RCA er en reaktiv prosess som finner svar på problemer basert på skjulte årsaker og deres effekt, istedenfor å undersøke den mest åpenbare årsaken. Det gjør at RCA er ofte komplekst og tidkrevende, men når rotårsaken først er funnet vil løsningene kunne fjerne problemet helt. 

\subsection{Historie}
Det har finnes flere varianter av rotåsaksanalyse opp gjennom tidene, men mannen som er kreditert med å finne opp rotårsaksanalyse er grunnleggeren av Toyota, Sakichi Toyoda. Hans versjon av RCA ble tatt i bruk av Toyota produksjons prosess i 1958 og ble kalt ``5 Whys''. Som tidligere sagt har RCA forandret seg over tidene for å imøtekomme de forskjellige feltene. Nå brukes RCA som verktøy i flere felter som transport, medisin og luftfart. 
    
\subsection{Ulike nivåer av årsaker}
Et problem er som regel ikke et resultat av en årsak, men heller en kombinasjon av flere årsaker på flere forskjellige nivåer. Dette vil si at årsaker påvirker andre årsaker helt opp til det synlige problemet. Årsaker defineres i tre forskjellige grupper: 

\begin{description}
    \item[Symptomer] er ikke å regne som faktiske årsaker, men heller bevis på eksisterende problemer.
    \item[Første-nivå årsak] er årsåker som leder direkte til et problem.
    \item[Høy-nivå årsaker] er årsåker som blir til første-nivå årsaker. Selv om de ikke direkte er årsak til problemmet, skaper høy-nivå årsak lenker i kjeden av årsak og effekt forholdet som til slutt fører til problemet.  Den høyeste nivå årsaken er rotårsaken.  
\end{description}
Noen problemer har flere årsaker som er forbundet av de forskjellige faktorer som kombinert blir til problemet.

\begin{figure}[H]
    \centering
    \includegraphics[scale=0.6]{main/bilder/nivaa.pdf}
    \caption[Nivå]{De forskjellige nivåer til problemet}
    \label{fig:nivaa}
\end{figure}

Vi ser fra figuren at rotårsaken er det som setter i gang årsak- og effektkjeden som leder til problemet. 

\subsection{Rotårsaksanalyse sammenlignet med risikoanalyse}
En risikoanalyse ser på sannsynlighet for at en trussel kan skje og mulige konsekvenser av dette. Risikoen regnes ut fra sannsynlighet og konsekvens, samt eksisterende kontroller. I risikoanalysen vil dataene brukes til å finne mulige preventive og reaktive tiltak som kan føre riskoen ned på et akseptabelt nivå. Rotårsaksanalyse vil på sin side gjøre en systematisk gjenomgang for å finne de underliggende årsakene til feil eller svikt. En rotårsaksanalyse gjøres etter et problem har oppstått, i motsetning til riskoanalyse som gjøres for å behandle fremtidige situasjoner.  

\section{Tidligere arbeid innen informasjonssikkerhet}
I 2016 ga NTNU en bacheloroppgave som innebar bruk av rotårsaksanalyse på tre forskjellige caser. Oppgaven viste at rotårsakanalyse fungerer i informasjonsikkerhetssammenheng. Vår oppgave skal jobbe videre med å se på hvordan rotårsaksanalyse fungerer i informasjonsikkerhet. OBS: TIL UTBEDRING!

\chapter{Metode}
\label{kap:metode}
Vår tilnærming til anvendt rotårsaksanalyse i dette prosjektet er gjennom tre caser som omhandler informasjonssikkerhet. Utførelse av rotårsaksanalyse kan gjøres på ulike måter, men grunnstrukturen er ofte lik. Det handler i bunn og grunn om problemløsning. I tillegg til å analysere casene skal bruken av metoden vurderes ut i fra hvor godt den fungerer innen informasjonssikkerhet. Påfølgende seksjon vil forklare valg av metode. 

\section{Metodevalg}
Hovedgrunnen til at rotårsaksanalyse brukes i dette prosjektet er, som nevnt tidligere, at vi skal undersøke bruksområder for denne analysemetoden i fagfeltet informasjonssikkerhet. Et problem er at rotårsaksanalyse er ikke en standardisert metode. Det er i hovedsak en metode for problemløsning, men det er mange foreslåtte tilnærminger til dette. En tidligere bacheloroppgave \cite{RCARapport} kom frem til at boka ``Root Cause Analysis: Simplified Tools and Techniques'' av Fagerhaug og Andersen \cite{RCA} beskriver en god og detaljert metode for hvordan rotårsaksanalyse bør utføres. Oppdragsgiver sa seg enig i at dette var et godt utgangspunkt når det skulle jobbes med rotårsaksanalyse, og anbefalte oss metoden. En grunn til at dette er en god metode er den detaljerte oppdelingen av de ulike fasene i RCA. Den skiller seg ut fra de fleste konkurrenter ved å detaljert beskrive hver fase, hvordan du skal gjennomføre den, og ikke minst verktøyene som kan brukes i gjennomføringen. Andre metoder går som regel bare gjennom det generelle hendelsesforløpet og anbefaler et par verktøy som kan brukes. Boken gir en praktisk beskrivelse av hvordan man gjennomfører rotårsaksanalyse. Den beskriver ikke bare hvilke verktøy du burde bruke, men også hvordan du bruker de og i hvilken sammenheng du bør bruke hvert verktøy. Dette bruker boken flytdiagram til å visualisere, og gjør det lett å velge rett verktøy til rett situasjon. Det er hjelpsomt siden oppgaven vår blant annet dreier seg om å finne ut hvilke verktøy som passer best til informasjonssikkerhetsproblemer. Flytdiagrammene finnes i vedlegg \ref{flytdiagrammer-verktoyvalg}.

\section{Metodekritikk}
Selv om metoden er god, har den et par svakheter. For det første er den sekvensiell. Dette gjør at det blir vanskelig å jobbe parallelt, og i større grupper. 
\section{Anvendt rotårsaksanalyse}
Metoden innebærer syv steg (visualisert i figur \ref{fig:prosess}), der hvert steg anbefaler et sett verktøy for å fullføre det. Bruk av et eller flere verktøy kommer helt an på problemet som skal løses. Valg av verktøy i hver fase er i stor grad basert på flytdiagrammene i boken av Andersen og Fagerhaug \cite{RCA}, som beskriver hvordan en velger riktig verktøy i hvert steg, for et bestemt problem. Selv om verktøyvalg er godt beskrevet tok vi med vår egen vurdering på hvilke verktøy som skulle brukes, siden boken ikke er direkte tilpasset informasjonssikkerhetsoppgaver. Vi har også brukt noen få verktøy utover det boken anbefaler når det kommer til dataanalyse. 

\begin{figure}[H]
    \centering
    \includegraphics[scale=0.6]{prosess}
    \caption[RCA-prosess]{De syv fasene i rotårsaksanalyseprosessen}
    \label{fig:prosess}
\end{figure}

\subsection{Problemforståelse}
Problemforståelse går ut på å få en solid forståelse for problemet en ønsker å løse og kan også hjelpe med å skape enighet i teamet rundt hva problemet egentlig omfatter. Det er også viktig for å passe på at ressursene som benyttes i analysen brukes effektivt videre. Under beskrives de verktøyene som ble brukt i denne fasen. 

\subsubsection{Flytdiagram}
Et flytdiagram viser flyten av aktiviteter i en prosess. I informasjonssikkerhet kan det brukes som en metode for å konkretisere og illustrere et problem ved eller angrep mot virksomhetens aktiva. Formålet er å skape en detaljert forståelse for en prosess som har med problemet å gjøre \cite{RCA}.

\subsubsection{Kritiske hendelser}
Hovedpoenget med kritiske hendelser er å identifisere de mest kritiske symptomene i problemet. Det kan hjelpe med å forstå hvilke aspekter ved problemet som trenger å løses, samt forstå problemets natur og konsekvenser for virksomheten \cite{RCA}. Innen informasjonssikkerhet har man ofte logger med hendelsesdata. Det gjør det naturlig å bruke kritiske hendelser siden informasjonen ofte eksisterer på forhånd og trenger bare å bearbeides. 

\subsubsection{Ytelsesmatrise}
Ytelsesmatrise er et diagram som tar i betraktning viktigheten og den nåværende ytelsen til en variabel. Dette gjør at man enkelt kan vurdere hvilken prioritering variablene som blir analysert har \cite{RCA}. Matrisen går fra en til ni på hver akse og er delt inn i fire like store hjørneområder. Ut i fra hvor høy ytelse ytelse og hvor viktige de er, er variablene enten: uviktig, overdrevent, må forbedres eller ok. I informasjonssikkerhet kan det brukes til å vurdere virksomhetens aktiva opp mot for eksempel eksisterende kontroller.

\subsection{Idémyldring}
Målet med idémyldring er å generere så mange idéer som mulig om et gitt emne. I rotårsaksanalyse er målet stort sett å generere en liste over problemområder som kan forbedres, identifisere mulige konsekvenser, generere en liste over mulige årsaker til problemet og oppmuntre til å tenke på løsninger som kan eliminere problemet. Det finnes i hovedsak to typer idémyldring: strukturert og ustrukturert. I strukturert idémyldring har hver deltaker sin tur til å komme med idéer. Dette fører til lik deltagelse og at ingen dominerer prosessen med egne idéer. I ustrukturert idémyldring kan hvem som helst komme med idéer når som helst. Dette er ofte mer spontant, men kan føre til at en person dominerer prosessen \cite{RCA}. 

\subsubsection{Nominell gruppeteknikk (NGT)}
Nominell gruppeteknikk er en strukturert metode for idémyldring som hjelper med å gå fra mange idéer, til å sitte igjen med de beste. Konseptet går ut på å myldre idéer og skrive dem ned på lapper, for så å anonymt gi idéene poeng fra en til fem. Et tallpoeng kan bare gis én gang. Til slutt blir tallene lagt sammen og du sitter igjen med de beste idéene. 

\subsection{Datainnsamling}
Datainnsamling er et steg i prosessen der man skal være strukturert og samle inn så mye relevant informasjon om problemstillingen som mulig. En god datainnsamling er sentralt for gode resultater i senere faser. Under beskrives verktøyene som ble brukt for å samle inn informasjon. 

\subsubsection{Spørreundersøkelser}
Spørreundersøkelser brukes når en er på utkikk etter å samle inn data om personers holdninger, følelser eller meninger om et spesifikt problem. En kan skille mellom to typer spørreundersøkelser: kvalitative og kvantitative spørreundersøkelser. Kvantitative undersøkelser handler om å få mange svar slik at en kan ta avgjørelser basert på tall som kan brukes til statistisk analyse. Kvalitative undersøkelser går ut på å samle detaljert informasjon om emnet. Dette kan hjelpe med blant annet å formulere hypoteser for å dirigere kvantitativ undersøkelse senere, eller å komplimentere en kvantitativ undersøkelse ved å bruke sitater fra åpne spørsmål \cite{KvalKvant}. Det brukes ofte elementer fra begge typene når en spørreundersøkelse lages. Når det er snakk om informasjonssikkerhet kan kvalitative undersøkelser brukes når det for eksempel skal samles inn informasjon om brukervaner på nett, eller grad av kunnskap og erfaring om informasjonssikkerhet. Kvalitative undersøkelser kan brukes når det kreves detaljert informasjon om et system eller indre forretningsprosesser. 

\subsubsection{Sampling}
Hovedpoenget med sampling er å trekke ut deler av en populasjon, for å trekke konklusjoner om denne uten å trenge å undersøke alle enhetene \cite{wiki:sample}. I rotårsaksanalyse kan det brukes for å effektivt samle inn data om problemer eller årsaker, og skaffe en bedre forståelse av situasjonen \cite{RCA}.

\subsection{Dataanalyse}
I denne fasen blir dataene analysert og visualisert. Hovedmålet er å avklare mulige rotårsaker som har innvirkning på problemet, og hvilke av de som har størst innflytelse. Under beskrives de ulike verktøyene som ble brukt for å analysere dataene.

\subsubsection{Histogram}
Histogrammer, også kjent som søylediagram, brukes for å vise distribusjon og varians i et datasett. Dataene kan vises i form av lengde, tid, kostnad, mengde osv. Hovedoppgaven til et histogram er å presentere data på en oversiktlig måte slik at det er lett å se mulige relasjoner. I rotårsaksanalyse brukes det til å se hvilke årsaker som dominerer og for å forstå distribusjonen av forskjellige problemer, årsaker, konsekvenser osv. \cite{RCA} Det er viktig å ha minst 30 svar for å lage et gyldig histogram \cite{RCA}.

\subsubsection{Affinitetsdiagram}
Affinitetsdiagram er et verktøy som kan brukes til å analysere kvantitative data. Formålet er å gruppere svar for å finne underliggende relasjoner mellom de resterende gruppene \cite{RCA}. I vår rotårsaksanalyse ble det brukt til å utforske relasjoner mellom forskjellige årsaker, og gruppere relaterte årsaker inn i klasser som kan analyseres kollektivt senere. 

\subsubsection{Statistisk analyse}
Statistisk analyse er ikke nevnt i boken \cite{RCA} og inneholder flere verktøy. 
\paragraph{One-way ANOVA} ANOVA, også kjent som variansanalyse, er en samlebetegnelse på en rekke statistiske metoder som tester likhet mellom to eller flere utvalg, der én eller flere faktorer gjør seg gjeldene \cite{ANOVA}. 
\paragraph{Independent-samples t-test} Dette er en type t-test som brukes for å teste gjennomsnittet mellom to uavhengige grupper på samme avhengige variabel \cite{t-test}.

\subsection{Rotårsaksidentifisering}
De foregående fasene skal ha generert en liste over mulige rotårsaker og målet i denne er å identifisere de faktiske årsakene. Det kan kreves flere iterasjoner for å finne rotårsaken(e). Verktøy som ble brukt til identifisering er beskrevet under. 

\subsubsection{Årsak-virkningsdiagram (Fiskebeindiagram)}
Et typisk årsak-virkningsdiagram undersøker og analyserer relasjonen mellom et problem og dets årsaker. Det fungerer som en kombinasjon av idémyldring og systematisk analyse. Det brukes for å generere og gruppere årsaker, og evaluere årsakene til problemet for å finne ut hvilke som mest sannsynlig er rotårsaker. Det finnes to typer årsaks-virkningsdiagrammer: fiskebeindiagram og prosessdiagram. Et prosessdiagram er egentlig en samling av fiskebeindiagrammer der hver prosess har sitt eget diagram. Det finnes to ulike tilnærminger til å skape et fiskebeindiagram: spredningsanalyse og årsaksopplisting. Kort forklart, spredningsanalyse grupperer først og idémyldrer etterpå, mens årsaksopplisting idémyldrer først og grupperer etterpå \cite{RCA}. 

\subsubsection{5 Whys}
5 Whys brukes for å undersøke høyere nivåer av årsaker. Som navnet beskriver, går det ut på å spørre ``Why?'' til en bestemt årsak for å komme frem til en ny årsak. Deretter blir det stilt spørsmål til den nye årsaken. Dette gjentar seg helt til det ikke er noen relevante årsaker å komme med. Den siste er da rotårsaken. Som en tommelfingerregel itererer man gjerne fem ganger, men det kan være både flere eller færre avhengig av problemet. 

\subsubsection{Feiltreanalyse}
Feiltreanalyse er en top-down, deduktiv analyse der en uønsket tilstand av et system analyseres ved hjelp av boolsk logikk for å kombinere en rekke hendelser \cite{wiki:faulttree}. I RCA brukes det for å få en klar oversikt over mulige årsaker, og for å se relasjoner mellom årsaker eller identifisere grupper med relaterte årsaker \cite{RCA}. 

\subsection{Problemeliminering}
Denne fasen innebærer å komme med mulige løsninger til problemet for å eliminere rotårsaken. Boken til Fagerhaug og Andersen \cite{RCA} beskriver to mulige tilnærminger til denne fasen. En tilnærming for å stimulere kreativitet når man leter etter løsninger, og en for å konstruere og utvikle løsninger. Vi har prøvd et verktøy fra hver tilnærming og de er beskrevet under. 

\subsubsection{De seks tenkehattene}
Formålet med de seks tenkehattene er å oppmuntre til å se problemet og løsningene fra forskjellige synsvinkler. Konseptet går ut på at personene får hver sin hatt som skal illustrere deres holdning til problemene \cite{RCA}. 

\begin{description}
    \item[Hvit hatt] skal være kald, nøytral og objektiv, personen skal fokusere på fakta.
    \item[Rød hatt] skal representere sinne, og skal bare fokusere på magefølelsen og egne følelser.
    \item[Svart hatt] skal være pessimistisk og negativ, og fokusere på hvorfor idéen er dårlig.
    \item[Gul hatt] er optimistisk og positiv, og skal fokusere på hvorfor idéen er bra og vil fungere.
    \item[Grønn hatt] representerer gresset, fruktbarhet og vekst, og skal fokusere på å vøre kreativ og komme på nye idéer.
    \item[Blå hatt] er koblet til himmelen, og skal fokusere på å se tingene fra et høyere perspektiv.
\end{description}

\subsubsection{Systematisk Innovativ Tenkning (SIT)}
SIT er en problemelimineringsmetode som baserer seg på konseptet om en ``lukket verden''. Dette betyr at den fokuserer på at løsningene på problemet ofte finnes i det fagområdet problemet eksisterer i. SIT baserer seg på å undersøke en eller flere kjernekomponenter ved hjelp av fem hovedprinsipper \cite{RCA}: 

\begin{description}
    \item[Attributtavhengighet] vurderer å endre en nøkkelvariabel i et produkt for å skape forbedring.
    \item[Komponentkontroll] ser på hvordan et produkt er knyttet til omgivelsene.
    \item[Erstatning] handler om å erstatte en del av et produkt med noe annet fra produktets omgivelser.
    \item[Forkastning] vurderer å forbedre problemet ved å fjerne en komponent. 
    \item[Oppdeling] har som mål å splitte et produkts attributter i to, som for eksempel splittelsen av sjampo fra balsam.
\end{description}

Hovedprinsippet i vårt prosjekt er å fokusere på at løsningene er tilknyttet fagområdet informasjonssikkerhet.

\subsection{Løsningsimplementering}
I den siste fasen er målet å implementere løsningene som ble funnet i foregående fase. I vår rapport vil implementeringen beskrives til beste evne, men ikke implementeres siden vi ikke har mulighet til dette. Implementeringen inkluderer blant annet organisering, utvikling av en implementeringsplan, skape et konsensus om de nødvendige endringene og selvfølgelig implementeringen. Implementeringen av løsningen kan sies å være en suksess når symptomene forsvinner. Verktøyene beskrives under.

\subsubsection{Trediagram}
Et trediagram er et verktøy som er enkelt å bruke og er passende for å dele opp større oppgaver inn i mindre, mer håndterlige aktiviteter. Det er et verktøy som hjelper til å organisere arbeidet som må gjennomføres for å implementere tiltakene som er anbefalt. I informasjonssikkerhet kan dette benyttes for å strukturere oppgavene og planlegge implementeringsprosessen av løsningen. Et trediagram visualiserer hierarkiet i aktivitetene som må gjennomføres, eller enklere sagt, rekkefølgen av gjøremål for å fullføre implementeringen. 

\subsubsection{Kraftfeltsanalyse}
En kraftfeltsanalyse er basert på den oppfatningen at alle situasjoner er resultatet av krefter som virker for og i mot den faktiske tilstanden. En forandring i disse kreftene vil fremkalle en endring, noe som kan brukes til å endre ting i ønsket retning. I rotårsaksanalyse brukes det for å få innsikt i endringsklimaet til en mulig implementering, samt å planlegge aktiviteter som skal til for å implementere løsningen \cite{RCA}. Verktøyet kartlegger krefter som virker for endringen, og krefter som virker i mot. 
\chapter{Gjennomføring av metode}
Dette kapittelet vil gå gjennom bruk av metoden i de tre casene. Det beskrives hvilke verktøy som ble brukt i hver fase, hvorfor vi valgte de, ønsket utbytte av verktøyene, spesifiseringer vi la til grunn og hvordan det ble gjennomført. Tabell \ref{tab:verktoymatrise} under viser de ulike verktøyene vi benyttet i hver fase, i hvert case. 
% Table generated by Excel2LaTeX from sheet 'Ark1'
\begin{table}[htbp]
  \centering
    \begin{tabular}{|l|l|c|r|r|r|}
    \hline
         \cellcolor{yellow} & \cellcolor{yellow} Verktøy & \cellcolor{yellow} Brukt & \multicolumn{1}{c|}{\cellcolor{yellow} Case 1} & \multicolumn{1}{c|}{\cellcolor{yellow} Case 2} & \multicolumn{1}{c|}{\cellcolor{yellow} Case 3} \\
    \hline
    \multicolumn{1}{|l|}{\multirow{4}[8]{*}{Problemforståelse}} & Flytdiagram & x     & \multicolumn{1}{c|}{x} &       &  \\
\cline{2-6}          & Kritiske hendelser & x     & \multicolumn{1}{c|}{x} & \multicolumn{1}{c|}{x} &  \\
\cline{2-6}          & Spiderdiagram & -     &       &       &  \\
\cline{2-6}          & Ytelsesmatrise & x     &       &       & \multicolumn{1}{c|}{x} \\
    \hline
    \multicolumn{1}{|l|}{\multirow{5}[10]{*}{Idémyldring}} & Idémyldring & x     & \multicolumn{1}{c|}{x} & \multicolumn{1}{c|}{x} & \multicolumn{1}{c|}{x} \\
\cline{2-6}          & Idéskriving & -     &       &       &  \\
\cline{2-6}          & Is-is not matrise & -     &       &       &  \\
\cline{2-6}          & NGT   & x     &       &       & \multicolumn{1}{c|}{x} \\
\cline{2-6}          & Parvis sammenligning & -     &       &       &  \\
    \hline
    \multicolumn{1}{|l|}{\multirow{3}[6]{*}{Datainnsamling}} & Sampling & x     &  \multicolumn{1}{c|}{x}  &       &  \\
\cline{2-6}          & Spørreundersøkelse & x     & \multicolumn{1}{c|}{x} & \multicolumn{1}{c|}{x} & \multicolumn{1}{c|}{x} \\
\cline{2-6}          & Sjekkliste & -     &       &       &  \\
    \hline
    \multicolumn{1}{|l|}{\multirow{6}[12]{*}{Dataanalyse}} & Histogram & x     & \multicolumn{1}{c|}{x} & \multicolumn{1}{c|}{x} &  \\
\cline{2-6}          & Paretodiagram & -    &       &        &  \\
\cline{2-6}          & Spredningsdiagram & -     &       &       &  \\
\cline{2-6}          & Problemkonsentrasjonsdiagram & -     &       &       &  \\
\cline{2-6}          & Relasjonsdiagram & -    &       &        &  \\
\cline{2-6}          & Affinitetsdiagram & x     & \multicolumn{1}{c|}{x} & \multicolumn{1}{c|}{x} & \multicolumn{1}{c|}{x} \\
    \hline
    \multicolumn{1}{|l|}{\multirow{4}[8]{*}{Rotårsaksidentifisering}} & Årsak-virkningsdiagram & x     & \multicolumn{1}{c|}{x} & \multicolumn{1}{c|}{x} &  \\
\cline{2-6}          & Matrisediagram & -     &       &       &  \\
\cline{2-6}          & 5 Whys & x     &       & \multicolumn{1}{c|}{x} & \multicolumn{1}{c|}{x} \\
\cline{2-6}          & Feiltreanalyse & x     &       &       & \multicolumn{1}{c|}{x} \\
    \hline
    \multicolumn{1}{|l|}{\multirow{3}[6]{*}{Rotårsakseliminering}} & De seks tenkehattene & x     & \multicolumn{1}{c|}{x} &     &  \\
\cline{2-6}          & TRIZ  & -     &       &       &  \\
\cline{2-6}          & SIT   & x     & \multicolumn{1}{c|}{x} & \multicolumn{1}{c|}{x} & \multicolumn{1}{c|}{x} \\
    \hline
    \multicolumn{1}{|l|}{\multirow{2}[4]{*}{Løsningsimplementering}} & Trediagram & x     & \multicolumn{1}{c|}{x} & \multicolumn{1}{c|}{x} &  \\
\cline{2-6}          & Kraftfeltanalyse & x     &       &       & \multicolumn{1}{c|}{x} \\
    \hline
    \end{tabular}%
  \caption[Matrise som viser valg av verktøy]{Matrise som viser valg av verktøy fra metoden i de tre casene}
  \label{tab:verktoymatrise}%
\end{table}%

Vi kan se over at det er noen verktøy vi har brukt alle casene, og noen vi ikke har brukt i det hele tatt. Under spesifiserer vi grunner til hvorfor dette er tilfellet.

Grunner til bruk av verktøy i alle caser:
\begin{description}
    \item[Idémyldring] ble brukt i alle casene fordi ingen i gruppen dominerte prosessen.
    \item[Spørreundersøkelse] ble brukt i alle casene fordi det var det mest logiske datainnsamlingsverktøyet for hvert av casene.
    \item[Affinitetsdiagram] ble brukt i alle casene fordi vi hadde minst ett spørsmål som var kvalitativt, og det var et behov for å analysere disse.
    \item[SIT] er en utvikling av TRIZ, der TRIZ er mer praktisk rettet. Dette passer ikke alltid så bra i informasjonssikkerhet, og SIT ble heller brukt fordi den er mer friflytene. 
\end{description}

Grunner til at noen verktøy ikke ble brukt i noen av casene:
\begin{description}
    \item[Spiderdiagram] ble ikke brukt fordi det ikke var nødvendig eller mulig med ekstern sammenligning i noen av casene.
    \item[Idéskriving] ble ikke brukt fordi ingen dominerte prosessen, og siden det ikke var behov for å skjule idéer mens de utvikles, ble ikke idéskriving brukt.
    \item[Is-is not matrise] ble ikke brukt fordi forskjeller ikke var forventet.
    \item[Sjekkliste] ble ikke brukt på grunn av tiden det ville ta for å logge hendelser i de ulike casene. 
    \item[Paretodiagram] ble ikke brukt fordi vi ikke hadde tilstrekkelig data til å analysere hvilke årsaker som utgjorde mest effekt. 
    \item[Spredningsdiagram] ble ikke brukt i casene fordi vi ikke hadde noe datasett der resultatene var såpass spredt at spedningsdiagram gir oss noe relevante resultater. 
    \item[Problemkonsentrasjonsdiagram] er i hovedsak fokusert på fysiske lokasjoner, med et fysisk problem. Siden våre caser handler om informasjonssikkerhet, ble det ikke relevant å utføre denne i noen av casene. 
    \item[Relasjonsdiagram] ble ikke brukt fordi vi hadde ikke så altfor mye data i numerisk form. 
    \item[Matrisediagram] ble ikke brukt fordi de andre verktøyene i denne fasen ble ansett som mer passende til det vi ønsket å gjøre.
    \item[TRIZ] er mer praktisk rettet enn SIT, derfor valgte vi heller SIT. 
\end{description}

Verktøyvalgene er i stor grad basert på flytdiagrammene i boka som beskriver vårt utgangspunkt til rotårsaksanalyse \cite{RCA}, og blir valgt basert på disse og et par andre variabler. Flytdiagrammene vi tar utgangspunkt i for hver fase kan sees i vedlegg \ref{flytdiagrammer-verktoyvalg}. I neste avsnitt vil vi gå gjennom alle casene og begrunne verktøyvalg, beskrive ønsket utbytte av verktøyene og dokumentere hvordan de ble brukt, med fokus på spesifiseringen vi la til grunn ved bruk av de. 

%------------------------CASE 1-----------------------------------------
\section{Case 1: Ulovlig fildeling på universitetsnettet}

\subsection{Problemforståelse}

\subsubsection{Flytdiagram}

\paragraph{Valg og ønsket utbytte av verktøy}
Ved å bruke et flytdiagram prøver vi å kartlegge hendelsesforløpet, og hvordan bruksmønsteret til en bruker av nettet til NTNU kan se ut, da med fokus på fildeling. Vi håper dette kan gi oss en helhetlig forståelse av hva årsaken til fildeling kan være, og kanskje vi kan bruke dette til å komme opp med tiltak senere i prosessen. 

\paragraph{Spesifiseringer}
Det gjøres en antagelse i at private fildelingstjenester ikke er med i statistikken fra universitet og at det ikke kommer noen opphavsrettsnotiser fra brukere som bruker private tjenester. Med private tjenester mener vi lukkede nettsamfunn som bare er til for å distribuere opphavsrettsbeskyttet materiale gratis.

\paragraph{Gjennomføring}
Vi fulgte metoden for å lage et flytdiagram, fra boken om rotårsaksanalyse \cite{RCA}.

\begin{enumerate}
    \item Vi samlet alle på gruppen for å diskutere prosessen, og hadde med oss post-it lapper
    \item Vi definerte brukerne som studenter ved NTNU og da, antageligvis mest studenter som bor på studenthybler.
    \item Vi definerte hvilke aktiviteter som gjøres for at skolen skal få et notis.
    \item Så flyttet vi rundt på lappene til de kom i en naturlig rekkefølge.
\end{enumerate}

\subsubsection{Kritiske hendelser}

\paragraph{Valg og ønsket utbytte av verktøy}
For å gå dypere i detalj har vi tatt en titt på kritiske hendelser som inngår i problemstillingen. Vi har valgt å bruke kritiske hendelser til å dele opp problemet i mindre stykker for å spørre studenter som bor i SiT-bolig om hva de laster ned hvis de først gjør det. Ved bruk av dette verktøyet ønsker vi å få en dypere forståelse av hva studentene laster ned slik at vi kan se om det er noen kategorier som er mest fremtredende. Dette kan føre til at vi blir mer effektive i senere arbeid da vi kan fokusere på det som er viktigst.

\paragraph{Spesifiseringer}
Det er bare studenter som bor i Sit bolig som har blitt registrert som svar. Vi prøvde å være så upartiske som mulig når det kom til valg av respondenter, men det var fortsatt stor overvekt av informatikkstudenter.

\paragraph{Gjennomføring}
Det første som ble gjort var å definere de mest relevante kategoriene av nedlastningsmateriale. Dette ble basert på egen erfaring med blant annet hyppigheten til de ulike kategoriene. Vi hadde en viss anelse om hvilke som var de store synderne, men ønsket å bekrefte våre mistanker. Følgende kategorier ble fremhevet: 
\begin{itemize}
    \item Filmer og serier
    \item Skolebøker
    \item Programvare til skolebruk
    \item Programvare og bøker utenom skolebruk
    \item Spill
    \item Musikk
    \item Annet
\end{itemize}

Intervjuene var i stor grad uformelle ``intervjuer'' med få spørsmål. Det viktigste vi trengte å vite var om personen bodde i SiT-bolig, siden det er bare beboere derfra som er relevante i vår problemstilling, og fra hvilke kategorier de laster ned fra. Resultatene ble fortløpende ført inn i statistikken. Intervjuobjektene var i stor grad bekjente vi visste bodde i SIT Bolig, og derfor var det en overvekt av IT studenter.

Spørsmål stilt til intervjuobjekter:
\begin{itemize}
    \item Bor du, eller har du bodd i SiT-bolig i løpet av studiet? (Hvis nei, avslutt intervju)
    \item Bruker du, eller har du brukt Torrents til å laste ned opphavsrettsbeskyttet materiale mens du bodde i SiT-bolig? Hvis ja, hvilke av følgende kategorier laster du ned fra? (Viser kategoriene)
\end{itemize}


\subsection{Idémyldring}

\paragraph{Valg og ønsket utbytte av verktøy}
Det finnes som sagt flere måter å utføre en idémyldring på, men vi benyttet oss av den ustrukturerte idémyldringen på bakgrunn av verktøyets egenskap til å generere mange idéer hurtig, og på grunn av dens spontane natur. 

\paragraph{Spesifiseringer}
Det er spesielt viktig å ikke omformulere eller diskutere forslagene etterhvert som de kommer, dette skal gjøres etter idémyldringsøkten er over.

\paragraph{Gjennomføring}
Det første som ble gjort når økten startet var å kommunisere og skrive opp problemstillingen på en tavle. Vi valgte å strukturere idémyldringen som et tankekart ettersom dette var en kjent løsning for gruppen. 

Problemet ble definert som hvorfor personer bruker Torrents til å laste ned opphavsrettsbeskyttet materiale. Når idéstrømmen begynte å gå langsomt, stoppet vi og vurderte det vi hadde kommet fram til. Vi kom blant annet fram til at vi burde spesifisere problemstillingen ytterligere og valgte derfor å spesifisere den til hvorfor folk laster ned på skolenettet. Vi forkortet dette til: ``Hvorfor Torrenting på skolenettet?'' for enkelthets skyld. Det ble kjørt enda en økt med denne nye problemstillingen og fikk mer spesifikke resultater.


\subsection{Datainnsamling}

\subsubsection{Spørreundersøkelse}

\paragraph{Valg og ønsket utbytte av verktøy}
I vår situasjon har vi valgt kvantitativ undersøkelse på bakgrunn av et par faktorer. For det første ønsker vi at undersøkelsen skal være helt anonym, siden spørsmålene omhandler potensielle lovbrudd. For det andre er målgruppen et stort antall personer, så det kan være nyttig å samle inn data fra så mange av de som mulig. 

Det vi ønsker å få ut av spørreundersøkelsen er data på utvalgte spørsmål vi mener er relevante for oppgaven. Spørsmålene er utarbeidet for å utforske hvorfor studenter som bor i studentbyer laster ned opphavsrettsbeskyttet materiale, som blant annet inkluderer undersøkelse av økonomiske perspektiver og tilgjengelighet på tjenester. I tillegg ønsker vi også innsikt i hvordan dette kan stoppes.

Hypotesen vi går inn i undersøkelsen med er at folk laster ned opphavsrettsbeskyttet materiale fordi det er lett tilgjengelig, tilknyttet liten til ingen kostnad og ikke minst fordi det er svært lav risiko for represalier.

\paragraph{Spesifiseringer}
Vi brukte Google Forms til å lage spørreundersøkelsen og motta data fra den. Vi setter krav til antall respondenter og utfører en rekke tiltak for å oppnå nok besvarelser, slik at undersøkelsen kan si noe om rotårsaken med relativt høy sikkerhet. Det er et krav å få minst 30 besvarelser som hadde lastet ned opphavsrettsbeskyttet materiale mens de har bodd i hybelen. Videre er det også ønskelig med relativ likhet i antall respondenter mellom de ulike fakultetene og studentbyene. Det hadde vært ideelt med minst 30 respondenter i hver kategori her også, men det er ønsketenkning i denne sammenhengen. Under prosessen ble også totalt antall beboere fra alle studentbyene i SiT bolig kartlagt. Boligtorget ga oss innbyggertallene fra hver studentby, og vi regnet oss frem til totalt 522 beboere. Det er viktig å presisere at det kan være usikkerheter knyttet til disse tallene, siden det kan hende ikke alle boligene har en beboer.

\paragraph{Gjennomføring}
En god undersøkelse vil alltid kreve kartlegging av demografi, og i vår undersøkelse valgte vi å kartlegge studentby, kjønn, alder og fakultet. Kjønn og alder er ganske selvforklarende, mens studentby ble valgt på bakgrunn av at Kallerud har mye raskere nedlasting- og opplastingshastighet enn de andre stedene. Vi anså også at det ville være forskjell på hvor mange som laster ned mellom for eksempel informatikkstudiene og helsestudiene. Videre ble resultatene i de foregående fasene brukt til å utforme spørsmålene. Spørreundersøkelsen inkluderer spørsmål om hvor godt en rekke påstander stemmer for den enkelte der respondentene svarer på en likert-skala fra 1-5, der 1 er i liten grad og 5 er i stor grad. Likert-skala ble valgt fordi det er en anerkjent måte å få inn kvantitative svar på hvor enige personer er med en påstand. Samtidig kan man enkelt sammeligne forholdene mellom svarene på de forskjellige påstandene ved hjelp av statistisk analyse. Til slutt inkluderer spørreundersøkelsen et spesielt viktig spørsmål om hva som skal til for at personen stopper med ulovlig nedlasting. Dette er et frisvar der vi kommer til å analysere individuelle svar hver for seg. Spørreundersøkelsen kan leses i sin helhet i vedlegg \ref{undersokelse}.
\newline

Siden spørreundersøkelsen er elektronisk var et av de første tiltakene som ble forsøkt å få tak i en e-post liste fra Sit. Dette fikk vi ikke fra dem. Istedenfor spredde vi den på relevante facebook-sider. Et av prosjektgruppens medlemmer jobber på Studenthuset her på Gjøvik, og fikk spørreundersøkelsen delt på deres facebook-side. Senere i prosessen ble også undersøkelsen delt på facebook-siden til linjeforeningen INGa og klassesidene til sykepleierne og webutvikling. Undersøkelsen ble også delt gjennom venner og bekjente; disse var for det meste informatikkstudenter. I tillegg ble det laget en plakat som ble hengt opp på oppslagstavler på skolen, i vaskeriene ved de ulike studentbyene, og mange ble også plassert i postkassene til SiT boliger. Plakaten finnes i vedlegg \ref{plakat}.


\subsection{Dataanalyse}

\subsubsection{Histogram}

\paragraph{Valg og ønsket utbytte av verktøy}
Hovedgrunnen til å bruke histogrammer er for å skape en visuell forståelse av dataene som en ellers ikke får fra tabeller o.l. Da blir det enklere å se korrelasjoner og sammenhenger i datasettet. For dette caset ønsker vi å visualisere hvilke mulige rotårsaker som er mest utbredt, og forstå distribusjonen av hendelser, problemer, årsaker, konsekvenser osv. Analyse av spørsmålene tilknyttet likert-skala er viktig for å finne mulige rotårsaker. Ellers ønsker vi også å finne sammenhenger mellom de ulike spørsmålene, for å se relevante korrelasjoner mellom de.

\paragraph{Spesifiseringer}
Det statistiske verktøyet SPSS ble brukt til å lage histogrammene. For at et histogram skal være gyldig, det vil si hvis vi skal kunne konkludere sikkert med noe, må det være minst 30 svar. 

\paragraph{Gjennomføring}
For å starte analysen eksporterte vi dataene til det statistiske verktøyet SPSS. Deretter ble diagramverktøyene brukt til å lage histogrammer ut fra svarvariablene. Noen spørsmål ble klynget sammen for å kunne se relasjoner mellom variablene visuelt. Vi testet først ut hypotesene våre, deretter ble det sjekket relativt tilfeldig om det var noen andre relevante data å vise i histogrammene. 


\subsubsection{Affinitetsdiagram}

\paragraph{Valg og ønsket utbytte av verktøy}
I spørreundersøkelsen inkluderte vi et kortsvar spørsmål som skulle gi oss svar på hva som kreves for at de stopper med ulovlig fildeling. For å analysere denne dataen var det bare affinitetsdiagram som passet dette. 

\paragraph{Spesifiseringer}
Det nettbaserte verktøyet draw.io ble brukt til å konstruere affinitetsdiagrammet. 

\paragraph{Gjennomføring}
Alle tekstsvarene, inkludert de engelske, ble analysert og kategorisert. Deretter ble dette lagt inn i verktøyet Draw.io og tildelt fargekoder og tallverdier som tilsier hvor mange som svarte i den kategorien. 


\subsubsection{Statistiske analyseverktøy}

\paragraph{Valg og ønsket utbytte av verktøy}
Fordi vi spurte om påstander og ba respondentene svare på en likertskala fra 1-5. Disse dataene er perfekt for å gjøre beregninger på. Derfor har vi valgt å benytte oss av one-way ANOVA analyse og Independent Sample t-test. Vi ønsket å benytte en independent t-test for å undersøke forskjeller mellom variabler der den uavhengige variabelen bestod av to grupper. Vi ville utforske om det var noen signifikant forskjell når det kom til om folk laster ned, mellom Kallerud og de andre studentbyene. I tillegg ville vi se på om det var noen signifikante forskjeller mellom IT fakultetet og de andre fakultetene. Ønsket utbytte ved bruk av ANOVA-analysen er å gi oss et bilde av om påstandene har noen signifikant verdi knyttet til demografien.

\paragraph{Spesifiseringer}
Vi benytter t-test når den uavhengige variablen har mindre enn to kategorier. One-way ANOVA brukes når det er flere en to kategorier. Generelt regner vi med en signifikans på: \[\alpha \le 0,05\]

\paragraph{Gjennomføring}
Vi delte opp demografien og ga svarene ett tall istedenfor en streng. Vi delte inn Kallerud og de andre studentbyene hver for seg, siden det var få svar på de andre studentbyene, slik at det ble jevn fordeling. Det samme ble også gjort med IT og de andre fakultetene. 

Forholdet mellom tallverdiene og svarkategoriene er som følger:

Kjønn:
\begin{itemize}
    \item Kvinne: 1
    \item Mann: 2
\end{itemize}

Fakultet:
\begin{itemize}
    \item Fakultet for arkitektur og design: 1
    \item Fakultet for informasjonsteknologi og elektroteknikk: 2
    \item Fakultet for ingeniørvitenskap: 3
    \item Fakultet for medisin og helsevitenskap: 4
    \item Fakultet for økonomi: 5
\end{itemize}

Når det er bare snakk om IT fakultetet mot andre:
\begin{itemize}
    \item Andre fakulteter: 1
    \item Fakultet for informasjonsteknologi og elektroteknikk: 2
\end{itemize}

Alder:
\begin{itemize}
    \item Under 20: 1
    \item 20-25: 2
    \item 26-30: 3
    \item 31-35: 4
    \item Over 35: 5
\end{itemize}

Studentby (Kallerud mot andre):
\begin{itemize}
    \item Kallerud: 1
    \item Andre studentbyer: 2
\end{itemize}

Har du lastet ned:
\begin{itemize}
    \item Nei: 1
    \item Ja: 2
\end{itemize}


\subsection{Rotårsaksidentifisering}


\subsubsection{Årsak-virkningsdiagram}

\paragraph{Valg og ønsket utbytte av verktøy}
I dette caset var det kommet frem til hovedårsaker som ble relevante til å sette inn i et fiskebeindiagram. Ved bruk av dette verktøyet ønsker vi å sitte igjen med en visuell fremstilling av rotårsaken til problemet. Dette vil gjøres ved å identifisere hva som skaper årsakene som kom frem av dataanalysen. 

\paragraph{Spesifiseringer}
Ingen.

\paragraph{Gjennomføring}
Det er anbefalt å bruke en tusjtavle for å tegne opp fiskebeindiagrammet, men vi valgte å bruke et nettbasert program som er laget for å skape diagrammer med flere brukere involvert i sanntid. De hadde en egen mal for fiskebeindiagram som vi valgte å gå ut fra. Stegene vi fulgte i prosessen er hentet fra boka om rotårsaksanalyse \cite{RCA} og ble som følger:

\begin{enumerate}
    \item Vi beskrev problemet klart og tydelig
    \item Vi tegnet opp problemet på slutten av fiskebeindiagrammet
    \item Vi identifiserte hovedkategoriene av årsakene til problemet og tegnet det opp på fiskebeinene i diagrammet
    \item Vi idémyldret alle mulige årsaker i hver kategori, en kategori om gangen, og skrev det inn i diagrammet fortløpende
    \item Til slutt analyserte vi de identifiserte årsakene og bestemte de mest sannsynlige rotårsakene
\end{enumerate}


\subsection{Rotårsakseliminering}

\subsubsection{De seks tenkehatter}

\paragraph{Valg og ønsket utbytte av verktøy}
I denne fasen har vi valgt å bruke vektøyet seks tenke hatter fordi vi ville stimulere til kreativitet når forslag ble fremmet. Ønsket utbytte med bruk av seks tenkehatter, er å skape en forståelse rundt rotårsaken og komme opp meg mulige tiltak for å eliminere problemet. Siden problemet virker vanskelig å fikse fra skolen sin side måtte vi komme med noen kreative løsninger, og de seks tenkehattene fungerer godt for dette. 

\paragraph{Spesifiseringer}
Siden vi bare var fire, og det egentlig kreves seks, måtte to av oss innehave to roller samtidig. 

\paragraph{Gjennomføring}
Siden vi bare var fire tok to av oss på seg to hatter og resten en. Så startet vi å diskutere problemstillingen og hvordan vi burde gå inn for å eleminere rotårsaken. Hver enkelt gruppemeldems tilnærming var basert på hatten de hadde på hodet. Etter at vi var ferdig med de seks tenkehattene gikk vi fort igjennom de forskjellige forslagene, for å se på hva som var praktisk gjennomførbare og ikke. 


\subsubsection{Systematisk Innovativ Tenkning}

\paragraph{Valg og ønsket utbytte av verktøy}
Ved å bruke SIT metoden ønsker vi å få kreative løsninger på hvordan vi kan stoppe ulovlig nedlasting.

\paragraph{Spesifiseringer}
Ikke alle hovedprinsippene kunne gi tiltak. Der det ikke var mulig ble det presisert ``Ikke gjennomførbart''. 

\paragraph{Gjennomføring}
Det første som ble gjort var å liste opp alle komponenter som eksisterer i problemets naturlige omgivelser. Deretter ble hver komponent analysert ut fra de fem hovedprinsippene til SIT, der tiltak ble beskrevet. De mest lovende tiltakene ble tatt videre til en mer detaljert gjennomgang. Deretter ble det plukket ut fra de igjen, de tiltak som var mest egnet, og ble videre ført inn i en tiltaksplan. 


\subsection{Løsningsimplementering}

\subsubsection{Trediagram}

\paragraph{Valg og ønsket utbytte av verktøy}
Trediagram benyttes for å skape en liste over aktiviteter som må gjennomføres for å innføre de spesifikke tiltakene. 

\paragraph{Spesifiseringer}
Diagrammverktøyet draw.io ble brukt til å konstruere trediagrammet. 

\paragraph{Gjennomføring}
Dette verktøyet startet med å gruppere hovedtiltak til rotårsaken, deretter ble hver aktivitet som må gjennomføres for at tiltaket skal bli gjennomført gruppert. Disse underpunktene ble gruppert etter hvilken rekkefølge de skal bli implementert slik at tiltaket er mulig å gjennomføre. 


%------------------------CASE 2------------------------------------------
\newpage
\section{Case 2: Kompromitterte brukerkontoer ved NTNU}

\subsection{Problemforståelse}

\subsubsection{Kritiske hendelser}

\paragraph{Valg og ønsket utbytte av verktøy}
For å lære mer om bakgrunnen til problemet bruker vi verktøyet kritiske hendelser for å se på frekvensen av misbruk som er registrert fra de kompromitterte kontoene. Slik kan vi kartlegge og forstå hva de kompromitterte kontoene brukes til. Ved bruk av dette verktøyet ønsker vi å få en oversikt over hvilke handlinger de kompromitterte kontoene utfører. Dette går i stor grad ut på hva som er motivasjonen til trusselaktørene. Vi ønsker å danne et bilde av hva de ønsker å oppnå ved å kompromittere kontoene, slik at vi kan bruke den informasjonen senere til å finne rotårsaken til at de blir kompromittert. 

\paragraph{Spesifiseringer}
Informasjonen som brukes i dette verktøyet ble gitt av oppdragsgiver. Dataene sier bare noe om frekvensen av sikkerhetshendelser, og ikke noe om viktigheten.

\paragraph{Gjennomføring}
Sammen med oppgavebeskrivelsen fikk vi en liste over loggførte sikkerhetshendelser som hadde foregått det siste året hvor kompromitterte kontoer var involvert. Dataene ble sortert i synkende rekkefølge og lagt inn i en tabell for å visualisere frekvensen til de enkelte sikkerhetshendelsene, og dermed fokusområdene til trusselaktørene. 


\subsection{Idémyldring}

\subsubsection{Valg og ønsket utbytte av verktøy}
Vi har valgt å benytte idémyldring på basis av RCA boken \cite{RCA} sin fremgangsmåte for valg av verktøy, og på bakgrunn av vår tidligere kunnskap om hvordan brukere vanligvis kompromitteres. Vi valgte å organisere idémyldringen som et tankekart ettersom dette var en kjent løsning for gruppen. Den ustrukturerte tilnærmingen til idémyldring ble brukt på grunn av dens uformelle struktur. Ingen er heller dominerende i gruppen, som gjør det mulig for alle å komme med innspill. Hvis noen i gruppen hadde vært dominerende hadde vi heller gått over til å bruke skriftlig idémyldring, også kjent som idéskriving. Ønsket utbytte ved å bruke idémyldring var for å få en forståelse av hva som kan være rotårsaken til at ansatte sine kontoer blir kompromittert, og hvordan passord og brukernavn kan komme på avveie. 

\subsubsection{Spesifiseringer}
Det er spesielt viktig å ikke omformulere eller diskutere forslagene etterhvert som de kommer, dette skal gjøres etter idémyldringsøkten er over.

\subsubsection{Gjennomføring}
Det første som ble gjort når økten startet var å kommunisere og skrive opp problemstillingen på en tavle. Vi diskuterte og prøvde å komme på mulige måter trusselaktører kan kompromittere brukerkontoer til de ansatte ved universitetet, og hvordan passord og brukernavn kan komme på avveie.


\subsection{Datainnsamling}

\subsubsection{Spørreundersøkelse}

\paragraph{Valg og ønsket utbytte av verktøy}
Grunnen til at vi valgte i hovedsak kvantitativ spørreundersøkelse er at vi ønsker å finne relasjoner mellom dataene vi samler inn. Det er også mulighet for å gjøre statistiske beregninger på disse, noe vi anser som relevant for datainnsamlingen i dette caset. Fremstilling av data i grafer og tabeller er også et moment som gjorde at valget ble kvantitativ metode. Med den elektroniske spørreundersøkelsen ønsker vi å få informasjon fra personer som allerede har fått brukeren sin kompromittert. Informasjonen vil bestå av blant annet personens passordvaner, kjennskap til retningslinjer om passordbruk og epost-aktivitet. Vi håper å få minst 30 respondenter, som vil være akkurat nok for en kvalitativ analyse. 

\paragraph{Spesifiseringer}
Undersøkelsen var kvalitetskontrollert flere ganger av forskjellige personer, inkludert medstudenter, veileder og ikke minst oppdragsgiver. Undersøkelsen ble laget i SelectSurvey, med tilhørende NTNU tema for utformingen. Dette ble gjort for å få spørreundersøkelsen til å virke legitim, siden den har NTNU sin logo i hjørnet og nettadressen tilhører NTNU sitt domene. 

Spørreundersøkelsen ble sendt ut til totalt 167 personer, men den nådde bare 157 av e-post addressene. Alle disse hadde fått sin NTNU konto kompromittert i tidsperioden 1. November 2016 til 1. April 2018. E-post listen ble opprettet av oppdragsgiver basert på intern data og sent ut på vegne av Seksjon for Digital Sikkerhet. 

Det ble oppdaget et par småfeil i spørreundersøkelsen etter den var utsendt. Blant annet var det glemt et ``vet ikke'' alternativ på spørsmålet om de hadde en formening om hvor lang tid det hadde gått fra kompromittering til de fikk beskjed. Dette fikk vi fikset ved å legge til tekst om at du kunne la det være blankt om du ikke visste, slik at det ikke gikk altfor mye ut over svarene, og spørsmålet ble endret til å ikke være obligatorisk. Det var også glemt en kommentarboks helt i slutten av spørreundersøkelsen, som vi bestemte at vi ikke kunne plassere inn etter den var utsendt. Det var også et par småfeil i formulering, men dette fikk vi endret underveis. Endringene ble gjort på natten da vi antok ingen svarte. 

\paragraph{Gjennomføring}
Det første som ble gjort var å finne ut hva vi ville ha informasjon om. Deretter ble spørsmål for å få svar på dette lagd. Videre definerte vi hypoteser til nesten hvert spørsmål. Vi skrev også en tekst som skulle bli sendt ut sammen med spørreundersøkelsen for å oppmuntre til å ta den. Her ble det brukt mye patos ettesom dette er et ømfintlig tema. Undersøkelsen ble sendt ut på fredag 20. April. 


\subsection{Dataanalyse}

\subsubsection{Histogram}

\paragraph{Valg og ønsket utbytte av verktøy}
Vi har igjen valgt å benytte histogrammer for å fremlegge dataene fra spørreundersøkelsen. Dette gjøres fordi det er en god måte å fremstille data på et tilfredstillende vis for leserne. Ved å benytte histogram håper vi å få en visuell fremstilling av data som gjør det raskt og enkelt å forstå disse, og derfor lettere kunne trekke konklusjoner. I dette caset vil histogrammer stort sett bli brukt for å få en oversikt over svarprosent på enkeltsspørsmål. 

\paragraph{Spesifiseringer}
Det statistiske verktøyet SPSS ble brukt til å lage histogrammene. For at et histogram skal være gyldig, det vil si hvis vi skal kunne konkludere sikkert med noe, må det ha minst 30 svar. 

\paragraph{Gjennomføring}
Dataene ble eksportert til SPSS og diagramverktøyet ble brukt til å konstruere histogrammene.


\subsubsection{Affinitetsdiagram}

\paragraph{Valg og ønsket utbytte av verktøy}
Et av spørsmålene i spørreundersøkelsen var et kortsvar der respondentene kunne svare om de hadde noen formening om hvordan de ble kompromittert. Ved å bruke affinitetsdiagram håper vi på å få samlet og gruppert disse dataene slik at vi kan se om noe blir sagt flere ganger, og kan være en mulig årsak. 

\paragraph{Spesifiseringer}
Det nettbaserte verktøyet draw.io ble brukt til å konstruere affinitetsdiagrammet. 

\paragraph{Gjennomføring}
Hvert enkelt svar ble analysert hver for seg og gruppert under en av mange hovedkategorier. Deretter ble disse hovedkategoriene tegnet som bokser i nettleserverktøyet Draw.io. Disse boksene inkluderer frekvens av svar og differensiering av svarene under hovedkategoriene. 


\subsubsection{Statistiske analyseverktøy}

\paragraph{Valg og ønsket utbytte av verktøy}
Spørreundersøkelsen inkluderer blant annet spørsmål om kjennskap til ulike dokumentasjon og bevissthet på sikkerhet. Dette er spørsmål som blir besvar på en skala fra 1-6. Dette er data som statistisk analyse kan bli benyttet på. Det er også mange ja/nei spørsmål som disse verktøyene kan brukes på- Ved å bruke statistiske analyseverktøy som ANOVA og Independent-samples t-test ønskes det å finne tilsynelatende skjulte relasjoner eller forskjeller mellom demografien og kjennskap til retningslinjer, bevissthet på sikkerhet og e-post og passordvaner. Vi ønsker også å se om det er noen relasjoner mellom andre spørsmål. ANOVA ble brukt dersom den uavhengige variabelen består av mer enn to grupper. Dersom den bestod av to grupper ble Independent t-test heller brukt.

\paragraph{Spesifiseringer}
Vi benytter t-test når den uavhengige variablen har mindre enn to kategorier. One-way ANOVA brukes når det er flere en to kategorier. Generelt regner vi med en signifikans på: \[\alpha \le 0,05\]

\paragraph{Gjennomføring}
For demografien og de andre spørsmålene som ikke hadde tallsvar, ble svarene transformert til tall. 

Forholdet mellom tallverdiene og svarkategoriene er som følger:

Kjønn:
\begin{itemize}
    \item Mann: 1
    \item Kvinne: 2
\end{itemize}

Primærrolle:
\begin{itemize}
    \item Ansatt: 1
    \item Student: 2
\end{itemize}

År ved NTNU:
\begin{itemize}
    \item Under 2: 1
    \item 2-4: 2
    \item 5-9: 3
    \item 10-15: 4
    \item Over 15: 5
\end{itemize}

Alle Ja/Nei spørsmål:
\begin{itemize}
    \item Ja: 1
    \item Nei: 2
\end{itemize}

Deretter ble testene kjørt på variablene. 


\subsection{Rotårsaksidentifisering}

\subsubsection{Årsak-virkningsdiagram}

\paragraph{Valg og ønsket utbytte av verktøy}
I dette caset har vi valgt et fiskebeindiagram på bakgrunn av at årsakene er spredt over flere variabler, og det er mulig at ytterligere årsaker eksisterer. Ved bruk av dette verktøyet ønsker vi å sitte igjen med en visuell fremstilling av rotårsakene til problemet. Dette vil gjøres ved å identifisere hva som skaper årsakene vi har funnet fram til i foregående fase. 

\paragraph{Spesifiseringer}
Ingen.

\paragraph{Gjennomføring}
Det er anbefalt å bruke en tusjtavle for å tegne opp fiskebeindiagrammet, men vi valgte å bruke det nettbaserte programmet draw.io \cite{drawio}. Draw.io er laget for å skape diagrammer med flere brukere involvert i sanntid. De hadde en egen mal for fiskebeindiagram som vi valgte å gå ut fra. Stegene vi fulgte i prosessen er hentet fra boka om rotårsaksanalyse \cite{RCA} og ble som følger:

\begin{enumerate}
    \item Først ble problemet definert og skrevet i slutten av fiskebeindiagrammet.
    \item Deretter ble hovedkategoriene skrevet ned i bokser. Disse er direkte tilknyttet resultatene fra analysen.
    \item Videre startet vi å idémyldre alle mulige årsaker under hver kategori, en kategori om gangen. Disse ble fortløpende skrevet inn i diagrammet.
    \item Til slutt analyserte vi de identifiserte årsakene og bestemte de mest sannsynlige rotårsakene
\end{enumerate}


\subsubsection{5 Whys}

\paragraph{Valg og ønsket utbytte av verktøy}
Etter fiskebeindiagrammet mente vi at det var sannsynlig at høyere nivå av årsaker kunne eksistere bak de identifiserte årsakene. Ved å bruke 5 Whys er det ønskelig å konfirmere om årsakene som ble fremhevet i fiskebeindiagrammet er faktiske rotårsaker, og ikke bare symptomer og/eller lav-nivå årsaker. 

\paragraph{Spesifiseringer}
Det ble tatt utgangspunkt i fem iterasjoner, men det er mulighet for flere eller færre avhengig av om spørsmålet kan besvares på en fornuftig måte. 

\paragraph{Gjennomføring}
Med dette verktøyet tar vi utgangspunkt i casebeskrivelsen, nemlig rotårsaken til kompromitterte kontoer ved NTNU. Ut fra dette brukte vi årsaker fra fiskebeindiagrammet over for å komme på årsaker som skal analyseres, samt prøvde å idémyldre et par nye. For hver av disse årsakene ble det spurt: ``Hvorfor er dette en årsak av det orginale problemet?''. For hvert svar spør vi hvorfor igjen og igjen helt til vi finner rotårsaken. 

\subsection{Rotårsakseliminering}

\subsubsection{Systematisk Innovativ Tenkning (SIT)}

\paragraph{Valg og ønsket utbytte av verktøy}
Grunnen til at SIT ble valgt er at vi mener problemet kan løses i problemets naturlige omgivelser. Ved å bruke SIT metoden ønsker vi å få kreative idéer på hvordan vi kan finne en løsning til kompromitterte kontoer ved NTNU. 

\paragraph{Spesifiseringer}
SIT burde helst gjennomføres av 10-12 personer, fra en rekke forskjellige fagområder, men siden vi ikke hadde så mange tok vi bare utgangspunkt i prosjektgruppen. Ikke alle SIT-prinsipper finner løsninger som er gjennomførbare for alle komponenter. I disse tilfellene vil det stå: ``Ikke gjennomførbart''. 

\paragraph{Gjennomføring}
Først ble alle komponentene som omhandler problemet listet opp. Etter det ble de fem hovedprinsippene fra SIT brukt sekvensielt på komponentene for å utvikle løsninger på problemene. Deretter ble de mest relevante løsningene valgt ut og beskrevet i ytterligere detalj. Videre ble de mest realistiske og gjennomførbare idéene trukket ut og fremhevet i en tiltaksplan.


\subsection{Løsningsimplementering}

\subsubsection{Trediagram}

\paragraph{Valg og ønsket utbytte av verktøy}
For å få en oversikt over hva som må gjøres for å implementere tiltaksplanen bruker vi Trediagram til å dele opp aktivitetene og bestemme rekkefølgen. Ønsket utbytte ved å bruke trediagram er å redegjøre for og strukturere de aktiviteter som må gjøres for å innføre de spesifikke tiltakene. 

\paragraph{Spesifiseringer}
Trediagrammets aktiviteter gjøres i samme rekkefølge som en postorder trealgoritme. Dette betyr at aktivitetene gjøres fra venstre til høyre, der en starten nederst til venstre. Det er derimot ingen gitt rekkefølge man kan innføre tiltakene i, og noen tiltak vil ikke kunne innføres sammen. 

\paragraph{Gjennomføring}
Gjennomføringen av trediagrammet startet med å gruppere hovedtiltak til rotårsaken, deretter ble hver aktivitet som må gjennomføres for at tiltaket skal bli gjennomført delt opp. Disse underpunktene ble plassert etter hvilken rekkefølge de skal bli implementert slik at tiltaket er mulig å gjennomføre. 


%------------------------CASE 3-----------------------------------------
\newpage
\section{Case 3: Misbruk av NTNU sin infrastruktur til utvinning av kryptovaluta}

\subsection{Problemforståelse}

\subsubsection{Ytelsesmatrise}

\paragraph{Valg og ønsket utbytte av verktøy}


\paragraph{Spesifiseringer}


\paragraph{Gjennomføring}



\subsection{Idémyldring}

\subsubsection{Nominell gruppeteknikk (NGT)}

\paragraph{Valg og ønsket utbytte av verktøy}


\paragraph{Spesifiseringer}


\paragraph{Gjennomføring}



\subsection{Datainnsamling}

\subsubsection{Intervju}

\paragraph{Valg og ønsket utbytte av verktøy}


\paragraph{Spesifiseringer}


\paragraph{Gjennomføring}



\subsection{Dataanalyse}

\subsubsection{Affinitetsdiagram}

\paragraph{Valg og ønsket utbytte av verktøy}


\paragraph{Spesifiseringer}


\paragraph{Gjennomføring}



\subsection{Rotårsaksidentifisering}

\subsubsection{5 Whys}

\paragraph{Valg og ønsket utbytte av verktøy}


\paragraph{Spesifiseringer}


\paragraph{Gjennomføring}



\subsubsection{Feiltreanalyse}

\paragraph{Valg og ønsket utbytte av verktøy}


\paragraph{Spesifiseringer}


\paragraph{Gjennomføring}



\subsection{Rotårsakseliminering}

\subsubsection{Systematisk Innovativ Tenkning (SIT)}

\paragraph{Valg og ønsket utbytte av verktøy}


\paragraph{Spesifiseringer}


\paragraph{Gjennomføring}



\subsection{Løsningsimplementering}

\subsubsection{Kraftfeltsanalyse}

\paragraph{Valg og ønsket utbytte av verktøy}


\paragraph{Spesifiseringer}


\paragraph{Gjennomføring}



\subsubsection{Trediagram}

\paragraph{Valg og ønsket utbytte av verktøy}


\paragraph{Spesifiseringer}


\paragraph{Gjennomføring}


\section{Case 1: Ulovlig fildeling på universitetsnettet til NTNU}
Her viser vi alle resultatene i de forskjellige 

\subsection{Problemforståelse}
For å få økt forståelse for problemet og samtidig lære mer om verktøyene i RCA\cite{RCA} valgte vi to verktøy til problemforståelsen. Flytdiagram ble valgt for å få en oversikt i hvordan personer laster ned ulovlig materiale og hvordan dette påvirker universitet. Kritisk hendelser verktøy ble brukt til å finne ut hva det er personer laster ned.  
\subsubsection{Flytdiagram}
Flytdiagrammet her viser det vi anser som et normalt hendelsesforløp til hvordan man laster ned materiale på universitetsnettverket. Det gjøres den antagelsen at private tjenester ikke er med i statistikken fra universitet og at det ikke kommer noen opphavsrettsnotiser fra brukere som bruker private tjenester. Med private tjenester mener vi lukkede nettsamfunn som bare er til for å distribuere opphavsrettsbeskyttet materiale gratis.

\begin{figure}[H]
    \centering
    \includegraphics[scale=0.5]{case_1/bilder/Flowchart.pdf}
    \label{fig:Flytdiagram}
    \caption[Flytdiagram for fildeling]{Flytdiagram for fildeling}
\end{figure}

\subsubsection{Kritiske hendelser}
For å besvare hva som ble lastet ned brukte vi kritisk hendelse for å få et enkelt overblikk. Dette ble gjort på en veldig uformell måte, der vi spurte studenter som var i nærheten. Merk at hver person kan svare at de laster ned fra flere kategorier.


Spørsmål stilt til intervjuobjekter:
\begin{itemize}
    \item Bor du, eller har du bodd i SiT-bolig i løpet av studiet? (Hvis nei, avslutt intervju)
    \item Bruker du, eller har du brukt Torrents til å laste ned opphavsrettsbeskyttet materiale mens du bodde i SiT-bolig? Hvis ja, hvilke av følgende kategorier laster du ned fra? (Viser kategoriene)
\end{itemize}

\noindent Dette er resultatet fra spørringene: \\
\indent Antall spurt: 13 \\
\indent Antall som aldri laster ned: 4
\begin{table} [H]
    \begin{tabular}{ | m{20em} | m{20em} | }
        \hline
            \cellcolor{yellow} Fildelingskategori & \cellcolor{yellow} Frekvens \\
        \hline
            Filmer og serier & 8  \\
        \hline
            Spill & 3 \\
        \hline
            Skolebøker & 2 \\
        \hline
            Musikk & 2 \\
        \hline
            Programvare og bøker utenom skolebruk & 1 \\
        \hline
            Programvare til skolebruk & 0 \\
        \hline
            Annet & 0 \\
        \hline
    \end{tabular}
    \caption{Oversikt over kvantiteten av kritiske hendelser ved torrenting av opphavsrettsbeskyttet materiale}
    \label{kritisk_tabell_1}
\end{table}


\subsection{Idémyldring}
I idémyldring ble verktøyet idémyldring valgt da det fungerer godt for vår gruppe dynamikk 
\subsubsection{Idémyldring}
Etter idémyldringen var ferdig ble det gjort en vurdering av resultatene og de ble kategorisert i henhold til likhetstrekk, under en fellesnevner som for eksempel Økonomi. Resultater og gruppering er som vist i figur \ref{fig:idemyldring} under. Merk at noen årsaker kunne ikke plasseres i én kategori og er derfor direkte knyttet til problemstillingen. 

\begin{figure}[H]
    \centering    \includegraphics[scale=0.45]{case_1/bilder/idemyldring}
    \caption[Idémyldring]{Resultater og gruppering av idémyldringen}
    \label{fig:idemyldring}
\end{figure}

Resultatene er gruppert inn i fire hovedkategorier, Økonomi (som går på kjøpekraften til den enkelte person), Tilgjengelighet (hvor god tilgang en har på tjenester), Anonymitet (foreldre kunne blant annet overvåke før, samt føler seg tryggere når det er flere på samme nett) og nye impulser (mer frihet og fritid, og påvirkning av nedlastningskulturen).


\subsection{Datainnsamling}
Det var forskjellige metoder å drive med datainnsamling, vi valgte kvantitativ spørreundersøkelse for å spørre mest mulig beboere fra Sit. Undersøkelsen ble begrenset til Gjøvik og endte med 97 svar totalt, dette er 18.6\% av de 522 beboerne i Sit bolig. Av disse var det 34 som svarte at de hadde lastet ned opphavsrettsbeskyttet materiale i hybelen, det er 35\% av de spurte. 

Undersøkelsen ble sendt ut gjennom forskjellige facebook sider tilhørende studentenmiljø i Gjøvik. I tillegg til facebook ble en plakat laget og ble lagt i postkassen til cirka halvparten av studentboligen på Kallerud og Sørbyen. Etter at det hadde gått en uke oversatte vi spørreundersøkelsen til engelsk, der vi hadde en kommunikasjonskanal som kunne sende denne til de fleste internasjonale studentene, der mange av disse bor på Sit hybler. Vi fikk rundt 20 respondenter fra de internasjonale, og alle disse resultatene ble oversatt til norsk og lagt inn i et samlet spørreskjema. 

\subsection{Dataanalyse}
Vi brukte tre verktøy til dataanalysen der to var fra boken\cite{RCA} pluss en ANOVA analyse. ANOVA analyse ble gjort får å prøve litt forskjellige verktøy, men er lagt i vedlegg da analysen ikke førte til noe signifikant data. De to andre verktøyene er histogram for å få oversikt og affinitetsdiagram til å få sortert skriveoppgavene. Under viser hvor mange som laster ned av de 97 som ble spurt i prosent. Rundt 37\% sier at det laster ned, som er en stor del av studentmassen. Det kan ha en påvirkning at undersøkelsen har en overvekt av studentene kommer fra informatikk- og datarelaterte studier.  


\begin{figure}[H]
    \centering
    \includegraphics[scale=0.45]{case_1/bilder/lasterned.pdf}
    \label{fig:lasterned}
    \caption[Laster ned]{Hvor mange som laster ned av de spurte}
\end{figure}


\subsubsection{Kjønnsforskjeller}
Vi ønsket også å undersøke om det er noen forskjeller i hvem som laster ned basert på kjønn. Under ser vi forholdet mellom kjønnene.
\begin{figure}[H]
    \centering
    \includegraphics[scale=0.45]{case_1/bilder/kjonn_lasterned.pdf}
    \label{fig:kjønn_lasterned}
    \caption[Kjønn laster ned]{Hvor mange fra hvert kjønn som laster ned}
\end{figure}
Vi kan se over at det er i hovedsak menn som laster ned ulovlig, mens kvinner har svart at de i stor grad ikke laster ned. Dette blir selvfølgelig påvirket at det er få kvinner i IT-relaterte studier, som vi har funnet ut at utgjør noe mer av nedlastingen. Det er derimot vanskelig å vite helt sikkert om det er fordi kvinner er underrepresentert i IT studier som gir utslag, eller om det er kvinner generelt sett som ikke laster ned. Der er likevel en mer signifikant forskjell mellom kjønnene enn det er mellom IT studier og andre studier som vist i figur \ref{fig:IT-lasterned}, så vi velger å tolke det som at kvinner laster ned mindre enn menn.


\begin{comment}


\subsubsection{Forskjeller mellom studentbyer}
Samtlige studentbyer har et kablet nettverk av Uninett godt egnet for nedlasting, enten det er lovlig eller ulovlig nedlasting. Det er derimot noen forskjeller i hastighet på enkelte studentbyer. Nordbyen, Sørbyen og Sentrum har muligheten til 100Mbps nedlasting og 100Mbps opplasting, mens Kallerud har en øvre grense på hele 1000Mbps nedlasting og 1000Mbps opplasting. Dette er ti ganger hastigheten til de andre studentbyene. Derfor ønsket vi å undersøke om dette hadde noe relevans i forhold til hvor mange som laster ned ulovlig. 

\begin{figure}[H]
    \centering
    \includegraphics[scale=0.45]{case_1/bilder/studentby_lasterned.pdf}
    \label{fig:studentby_lasterned}
    \caption[Studentby laster ned]{Hvor mange fra hver studentby som laster ned}
\end{figure}

På grunn av lav oppslutning på Nordbyen og Sentrum er det vanskelig å si noe sikkert på dem, mens Kallerud og Sørbyen ikke varierer så veldig fra hverandre. Når vi bruker histogrammer ser vi ingen signifikant forskjell mellom studentbyene når det kommer til nedlasting som vi kan si med sikkerhet.



\end{comment}

\subsubsection{Konsekvenser ved nedlasting}
Et spørsmål som ble spurt i spørreundersøkelsen var hvor godt kjent de var med mulige konsekvenser ved ulovlig nedlasting, og med det brudd på opphavsretten. Det kunne være relevant å se om det var noen spesiell sammenheng mellom de som ikke kjente til konsekvensene og de som lastet ned. 

\begin{figure}[H]
    \centering
    \includegraphics[scale=0.45]{case_1/bilder/konsekvens_lasterned.pdf}
    \label{fig:konsekvens_lasterned}
    \caption[Konsekvens av å laste ned]{Hvor mange som kjenner til konsekvenser ved å laste ned}
\end{figure}

Det viser seg faktisk at de som laster ned har en bedre kjennskap til konsekvensene enn de som ikke gjør det. Det kan kanskje ha noe å gjøre med at de er mer opptatt av problemområdet enn de som ikke laster ned. De som ikke laster ned i første omgang har kanskje ingen grunn til å sjekke konsekvensene av det. I tillegg fant vi ut at IT-studenter kjenner konsekvenser bedre enn de andre, og de har også høyere andel nedlastere. Vi prøvde å kjøre samme test på hvor godt de kjenner til IT-reglementet til NTNU \cite{ITReg} og kom til samme konklusjon som over. Dette histogrammet kan sees \hyperref[fig:reglement-lasterned]{her}. En grunn til dette er også igjen at IT-studenter kjenner bedre til IT reglementet, som vist \hyperref[fig:reglement-fakultet]{her}, og de er i overvekt. Så dette må tas i betraktning. 

En mulig teori vi ønsket å prøve ut var om mange som lastet ned ikke kjente til konsekvensene ved ulovlig nedlasting eller NTNU sitt IT-reglementet, og lastet ned på grunn av det. Dette ble altså delvis motbevist.

\subsubsection{Årsaker til nedlasting}
I spørreundersøkelsen kom vi med seks påstander til hvorfor respondentene laster ned, som de besvarte på en likert-skala fra 1 til 5, der 1 er i liten grad og 5 er i stor grad. Etter å ha analysert alle seks påstandene ved hjelp av SPSS og histogrammer fant vi én påstand som hadde en graftopp der respondentene svarte positivt. De aller fleste svarte de var enige i at de lastet ned på grunn av tilgjengelighet.

\begin{figure}[H]
    \centering
    \includegraphics[scale=0.45]{case_1/bilder/tilgjengelighet.pdf}
    \label{fig:tilgjengelighet}
    \caption[Tilgjengelighet]{Hvor mange som laster ned av de spurte}
\end{figure}

Dette vil si at det er godt mulig at tilgjengeligheten er en årsak til om en laster ned eller ikke, og er verdt å dokumentere til videre analyse. Siden tilgjengelighet betyr så mye, var det naturlig å utforske det ytterligere. Vi fant ut at det kunne være relevant å vite om de som brydde seg om tilgjengelighet hadde tilgang på strømmetjenester, og i så fall hvor mange.

\begin{figure}[H]
    \centering
    \includegraphics[scale=0.45]{case_1/bilder/tilgjengelighet_antallstromming.pdf}
    \label{fig:tilgjengelighet_antallstromming}
    \caption[Tilgjengelighet vs antall strømmetjenester]{Korrelasjonen mellom de som laster ned på grunn av tilgjengelighet og hvor mange tjenester de har tilgang på}
\end{figure}

Her viser det seg at de som laster ned på grunn av tilgjengelighet også har en god del strømmetjenester. Dette sier at mange av disse er storforbrukere av film og serier, og at det ikke har så mye å si om de har tilgang til tjenestene eller ikke. Dette vil muligens utelukke en løsning i form av at NTNU tilbyr en tjeneste siden de kommer til å laste ned uansett.

%--------------------------------------------------------








\begin{comment}
Her nevner vi kort våre resultater i de viktigste fasene som hadde direkte innvirkning på identifisering og eliminering av rotårsaken. 

\subsection{Datainnsamling}
Det var forskjellige metoder å drive med datainnsamling, vi valgte kvantitativ spørreundersøkelse for å spørre mest mulig beboere fra Sit. Vi fikk svarprosent på ca 18\%, og vi forventet svarprosent på 15-20\% av alle beboere i Sit boligene i Gjøvik. Vi valgte å forholde oss til studentbyene i Gjøvik, ikke i Trondheim eller Ålesund. Av studentbyene vi spurte, fikk vi desidert mest svar fra Kallerud. Av alle som svarte var det 50\% som bodde på kallerud

Vi postet et innlegg på Huset ansatte, og tok kontakt med linjeforeningene der vi spurte om de kunne legge ut en link til undersøkelsen. Her fikk vi best respons fra Huset ansatte og INGa sine facebooksider. Vi la også en plakat i litt under halvparten av postkassene på Kallerud og på Sørbyen, og vi fikk grei respons fra dette.

Etter at det hadde gått en uke oversatte vi spørreundersøkelsen til engelsk, der vi hadde en kommunikasjonskanal som kunne sende denne til alle de internasjonale studentene, der mange av disse bor på Sit hybler. Vi fikk rundt 20 respondenter fra de internasjonale, og alle disse resultatene ble oversatt til norsk og lagt inn i et samlet spørreskjema.


\subsection{Dataanalyse}
Kartlegging av omfanget viste at 35\% av de spurte drev med ulovlig fildeling. Dette var lavere enn vi trodde, men fortsatt mange. De fleste var småforbrukere, men det var også en del storforbrukere som laster ned over ti torrents i måneden. Fra dataanalysen kunne vi også konkludere med at tilgjengelighet var en svært viktig grunn til at folk lastet ned. Selv de som hadde tilgang på mange strømmetjenester svarte at tilgjengeligheten var en viktig grunn til at de lastet ned. Økonomi var viktig for noen, men også uviktig for en god del. Det viste seg også at det var dårlig håndhevelse og kommunikasjon av lover og regler. 

Det var også forskjeller blant demografiene. Menn var en stor andel av de som lastet ned, mens kvinner nesten ikke lastet ned noe. Når det kommer til fakultet var IT fakultetet overrepresentert i nedlastingsstatistikken. De var også de som kjente til IT reglement og konsekvenser best. Generelt sett var det lite kunnskap om IT reglement, og varierende kjennskap til konsekvenser. 

\subsection{Rotårsaksidentifisering}
Vi valgte å bruke fiskebein for å strukturerte vår rotårsaksidentifisering. Som et verktøy fungerte det meget godt, der vi klarte å få organisert årsakene inn i tre hovedgrupper: Økonomi, Tilgjengelighet og Risiko. Vi analyserte hver hovedgruppe nøyere og kom fram til en årsak for hver gruppe: Dårligere utvalg på alternative tjenester i Norge, Tjenestene er ikke verdt prisen og Håndheving og kommunisering av lovene knyttet til ulovlig fildeling blir ikke prioritert. 

\subsection{Rotårsakseliminering}
Det vi kom frem til her var fire forskjellige løsninger for å fjerne rotårsaken, der to av dem var mer globale og ikke gjennomførbare for skolen, og to mer praktisk gjennomførbare som ikke fjerner det at folk laster ned, men skyver problemet over til andre. De ikke gjennomførbare var at alt av materiale blir gratis og tilgjengelig på ett samlet sted, og fjerne de geografiske blokkeringene. De gjennomførbare var at man bytter ISP til studentboligene for å forflytte problemet bort fra NTNU's ansvarsområde, og stenge torrentprotokollen for alle på nettverket. 

\subsection{Løsningsimplementering}
Vi benyttet trediagram (figur \ref{fig:Tre-diagram}) for å illustrere de ulike arbeidsoppgavene som kreves for å implementere tiltakene.
\end{comment}
\section{Case 2: Kompromitterte brukerkontoer}

\section{Case 3: XXX}
\chapter{Diskusjon}
\label{kap:diskusjon}
Utgangspunktet i denne drøftingen er hvorvidt rotårsakene vi kom frem til i hvert case er reelle rotårsaker som, hvis fjernet, vil fjerne symptomene helt. I tillegg ser vi på hvordan erfaringen fra de tre casene viser hvor godt rotårsaksanalyse fungerer innen informasjonssikkerhet. 

\section{Hva er rotårsaken til at studenter laster ned opphavsrettsbeskyttet materiale?}
\subsection{Rotårsak: Dårligere utvalg på alternative tjenester i Norge}
Ut fra vår analyse viser det seg at dette er hovedårsaken til at studenter laster ned ulovlig. Siden Netflix og de andre strømmetjenestene har inntatt markedet har filmer og serier gruppert seg mellom de. Tjenestene ønsker også flest mulig orginale serier som bare er hos dem. Dette gjør at tilgjengeligheten på filmer og serier går ned, med mindre man abonnerer på alt. Men selv da får man ikke tilgang på alt. Mange filmer og serier er geografisk blokkert i Norge, som gjør tilgjengeligheten til et enda større problem. I musikkstrømmingsmiljøet er problemet noe mindre. Selv om ulovlig musikknedlasting ikke er borte, har det blitt redusert \cite{musikkstream}. Noe av grunnen til dette er at musikkbransjen er mer sentralisert i hvem som eier rettighetene, også kjent som et oligopol. Dette gjør det lettere for strømmetjenestene å skaffe lisenser for musikk, og kan tilby det folk trenger på ett sted. Det er også mye mindre orginalt innhold i disse tjenestene i forhold til strømmetjenester for filmer og serier. 

\subsection{Rotårsak: Tjenestene er ikke verdt prisen}
Jo flere tjenester det blir, jo mer må man betale for å få tilgang på mer materiale. Filmer og serier blir spredt utover markedet på flere tjenester som så og si koster det samme. Dette fører til at hver enkelt tjeneste blir mindre verdt pengene man må betale for å få tilgang. Det skal sies at strømming er en revolusjonerende løsning i forhold til å kjøpe hver enkelt film for seg selv, men hvis man må betale for fem forskjellige strømmetjenester for å få tilgang til det man har lyst på, hvorfor ikke bare laste ned gratis? Analysen vår viste at det å betale for tjenester ikke var noe problem for studentene; problemet var at de ikke føler de får det de betaler for. 

\subsection{Rotårsak: Håndheving og kommunisering av lovene knyttet til ulovlig fildeling blir ikke prioritert}
Det eksisterer allerede regler på ulovlig nedlasting på universitetsnettet. Problemet er derimot at det er vanskelig å håndheve de. Andre arbeidsoppgaver har heller blitt prioritert. Enkelte tiltak har heller ikke vært lovlige for NTNU å gjennomføre for å stoppe de som driver med ulovlig fildeling. For eksempel er det ikke lov å overvåke enkeltboliger hos Sit, og heller ikke straffe enkeltpersoner dersom de laster ned, siden det blir regnet som inngrep i den private sfæren. Dette har datatilsynet fortalt Seksjon for Digital Sikkerhet. 


\section{Hva er rotårsaken til at brukerkontoer ved NTNU blir kompromittert?}

\subsection*{Rotårsak: Gjenbruk av brukerkredentialier på tredjepartssider}
Vi har vurdert gjenbruk av brukerkredentialier på andre tjenester som den mest relevante rotårsaken til at NTNU sine brukerkontoer blir kompromittert. Dette gjør vi på bakgrunn av at det var over halvparten som hadde svart at de hadde brukt sine NTNU kredentialier på flere tjenester. Dette betyr ikke nødvendigvis at det er den rotårsaken en bør frykte mest. I en studie gjort på oppdrag fra Google – som tok utgangspunkt i e-postadresser – viste det seg at selv om studien fastslo at det var desidert flest som var blitt kompromittert av datainnbrudd på andre tjenester, hadde flere hadde byttet passord siden de var blitt kompromittert, sammenlignet med de som hadde blitt kompromittert av phishing \cite{46437}. Likevel viser studien også den store mengden kontoer som blir kompromittert som følger av datainnbrudd ved andre tjenester, som bekrefter at det fortsatt er et stort problem. 

\subsection*{Rotårsak: Phishing}
Phishing var en av årsakene som ble belyst, og det viste seg at brukerne ikke hadde fått tilstrekkelig opplæring i deteksjon av phishing e-post. Phishing er, og har lenge vært, en stor årsak til kompromitterte kontoer \cite{SophPhish}. Phishing skjer også svært hyppig; undersøkelsen vår viste at de aller fleste hadde lagt merke til flere hendelser med phishing på sin NTNU e-post. Phishing kan være vanskelig å gjøre noe med. Vår formening er at det alltid vil være en risiko, uansett hva slags tiltak en implementerer. På en side kan både tekniske og bevissthetsmessige tiltak hjelpe, men disse vil aldri fjerne rotårsaken helt. 

\subsection*{Rotårsak: For dårlig kjennskap til styrende dokumenter}
Det er alltid en vanskelig oppgave å gjøre de ansatte oppmerksom på beste praksis innen informasjonssikkerhet. Dette gjelder også NTNU siden det i resultatene våre ble fremhevet at de ansatte hadde liten kjennskap til reglementer, retningslinjer og prinsipper knyttet til IT og informasjonssikkerhet. Det er imidlertid en pågående debatt om det i det hele tatt er verdt tiden og pengene i å forsøke å trene opp ansatte. Mange mener disse pengene kan bli bedre brukt på andre vis. Bruce Schneider skriver i sin blogg at dette er bortkastet tid og penger \cite{SecAware}. Mange er enige med han, men det er også mange eksperter som mener det er nyttig. Vi mener derimot at det er nyttig, men ressursbruken på dette burde holdes lav. 

\subsection*{Rotårsak: Utilstrekkelig tilgangskontroll på brukerkontoer}
Vi har kommet frem til at dette er et problem som kan løse mange av symptomene ved hjelp av tilgangskontroll. 2FA med sms er noe av det som blir anbefalt av oss. Dersom 2FA blir benyttet vil det hindre de fleste kontoer i å bli kompromittert, selv om kredentialiene blir kjent for trusselaktørene. Det er imidlertid mange som mener at SMS meldinger er en usikker løsning på 2FA, siden sms meldinger er relativt enkelt å avlytte \cite{2FA}. De fleste anbefaler enten autentisering gjennom applikasjon eller fysisk kodebrikke. Disse metodene er dessverre noe vanskeligere å implementere, og det er heller ikke alle som har en smarttelefon som kan bruke applikasjonene som kreves. Et annet tiltak som blir mye brukt ellers er å validere brukerkontoen for spesifikke maskiner når de logges på for første gang, eller bare gi beskjed om ny innlogging et annet sted slik at en blir oppmerksom på at kontoen kan være kompromittert. Disse brukes av flere tjenester for å informere om og hindre kontoer fra å bli kompromittert. Google gir deg både beskjed når nye innlogginger finner sted, og gir deg muligheten til å legge til klarerte enheter \cite{trustcomp}. Dette fungerer ofte som en erstatning til 2FA hver gang du logger på. Siden dette har vært effektivt i andre sammenhenger ser vi ingen grunn til at dette ikke vil fungere bra hos NTNU, annet enn den ekstra anstrengelsen for brukerne når de logger på. 


\section{Hva er rotårsaken til misbruk av NTNU sin infrastruktur til utvinning av kryptovaluta?}



\section{Hvor godt fungerer rotårsaksanalyse innen informasjonssikkerhet?}
Det er fortsatt få studier som prøver å sette lys på nytteverdien ved bruk av rotårsaksanalyse innen informasjonssikkerhet. I løpet av dette prosjektet har vi gjort oss en erfaring basert på verktøybruken. Basert på resultatene kan det sies å ha fungert bra. På den ene siden vet vi ikke helt hvor bra det har fungert før tiltakene er implementert, og det er kontrollert at symptomene minker eller forsvinner helt. På den andre siden har et tidligere bachelorprosjekt allerede kommet frem til at nytteverdien er stor. De stilte blant annet spørsmål om hvor godt det fungerer på case med lite tid og ressurser, samt mye tid og ressurser \cite{RCARapport}. Det ble i begge sammenhenger konkludert med at det ga gode resultater. Vi mener at nytteverdien kommer an på hvor god tilgang en har på relevant informasjon. I noen av casene fikk vi et godt datagrunnlag som ga oss gode muligheter til å avdekke rotårsakene, mens spesielt det tredje caset slet vi med lite datagrunnlag. Vi anser dette å være kritisk for hvor god nytteverdien er. Nytteverdien i forhold til tid diskuteres ytterligere i seksjon \ref{sek:tidsbruk_case2}. 

\section{Lønner det seg å benytte rotårsaksanalyse i informasjonssikkerhetssammenheng?}


\section{Hvilke metoder og verktøy som ofte brukes i rotårsaksanalyse, lønner seg mest å bruke innen informasjonssikkerhet?}

\chapter{Veileder i bruk av rotårsaksanalyse innen informasjonssikkerhet}
\label{kap:veiledning-RCA}
Veilederen er skrevet slik at de kan bli benyttet uten bacheloroppgaven. Derfor vil noe av det som er gjennomgått i bacheloroppgaven bli gjentatt.

\section{Formål og bakgrunn}
Formålet med dette dokumentet er å gi leseren en veileder for anvendelse av rotårsaksanalyse. Veilederen beskriver anvendelse av rotårsaksanalyseverktøyene beskrevet i boken ``Root Cause Analysis: Simplified Tools and Techniques - second edition'' av Bjørn Andersen og Tom Fagerhaug \cite{RCA} i forhold til informasjonssikkerhet. Vi anbefaler denne boken som samlet metodikk. Boken beskriver godt hva som skal til for å komme frem til rotårsaken til et problem. 

Dette dokumentet ble skrevet i forbindelse med bacheloroppgave i informasjonssikkerhet der det ble gjennomført tre caser. Veilederene er derfor basert på funn fra disse casene, om hvordan metoden og verktøyene fungerte.

Dette dokumentet vil ikke beskrive verktøyene, men valg av de. Beskrivelsen til verktøyene står i boken til Fagerhaug og Andersen \cite{RCA}.

\section{Valg av verktøy}
Boken beskriver 7 faser i rotårsaksanalyse. Disse må bli fulgt stegvis ettersom hver fase bygger på resultater fra foregående. Verktøyene beskrevet i boken til Fagerhaug og Andersen \cite{RCA} er generelle verktøy som er ofte brukt i rotårsaksanalyse. Vi har sett på et utvalg av disse verktøyene og hvordan disse fungerer innen informasjonssikkerhet. Figur \ref{fig:prosess_veileder} under viser verktøyene i metodikken, og er fargekodet ut i fra hva vi anbefaler. 

\begin{figure}[H]
    \centering
    \includegraphics[scale=0.6]{main/bilder/RCA_Prosess_rettningslinjer.pdf}
    \caption[RCA-prosess]{De syv fasene i rotårsaksanalyseprosessen}
    \label{fig:prosess_veileder}
\end{figure}

\subsection{Problemforståelse}
Problemforståelse går ut på å få en solid forståelse for problemet en ønsker å løse. Det kan også hjelpe med å skape enighet i teamet rundt hva problemet egentlig omfatter. Det er også viktig for å passe på at ressursene som benyttes for analysen brukes effektivt videre. 
Verktøyene vi anbefaler til denne fasen er: 

\subsubsection{Kritiske hendelser} 
Kritiske hendelser er et godt verktøy å bruke når man har mye data som kan gi et innblikk i hva som går galt. Etter vår erfaring kan man bruke logger til å finne de kritiske hendelsene, hvilket passer utmerket i et informasjonssikkerhetsperspektiv. For kritiske hendelser i informasjonssikkerhet anbefaler vi at det utføres slik: 
\begin{description}
    \item [Steg 1: Logg tilgjengelig] Bruk informasjon fra hendelseslogger til å finne informasjon om kritiske hendelser.
    \item [Steg 1: Informasjon ikke i logg] Om hendelsene ikke er logget, kan informasjon hentes fra personell som jobber med hendelsene, eller samles inn selv.
    \item [Steg 2:] Sorter hendelsene etter frekvens.
    \item [Steg 3:] Bruk de mest kritiske hendelsene som startpunkter for analysen.
\end{description}

\subsection{Ytelsesmatrise}
Ytelsesmatrise er et godt verktøy for å få oversikt over hvilke tiltak og kontrollere som blir brukt og hvor godt disse fungerer per nå. Bruk dette verkøyet når du vil undersøke viktigheten og ytelsen til problemområder, og trenger et prioriteringsgrunnlag. 

\begin{description}
    \item[Steg 1:] Lag en tom matrise fra en til ni med horisontal linje som viktighet og den vertikale aksen skal ha nåværende ytelse.
    \item[Steg 2:] Bestem problemer og variabler som skal analyseres, og estimer viktighet og nåværende ytelse til hver variabel.
    \item[Steg 3:] Plasser disse på matrisen etter viktighet og ytelse.
    \item[Steg 5:] Tegn opp fire like områder i matrisen og gi disse følgende navn: Ok (ytelse:5-9, viktighet:5-9), Overdrevent (ytelse:5-9, viktighet:1-5), må forbedres (ytelse:1-5, viktighet:5-9) og uviktig (ytelse:1-5, viktighet:1-5).
    \item[Steg 6:] Dersom variablene er overrepresentert i en eller to av kvadratene, kan skillelinjene justeres for å jevne ut.
\end{description}

Det som kommer nederst i ``må forbedres'' prioriteres først for et startpunkt for videre analyse. De andre prioriteres etter dette. Det som kommer under ``Ok'', bør også vurderes dersom du har nok ressurser. 

\subsection{Idémyldring}
Målet med idémyldring er å generere så mange idéer som mulig om et gitt emne. I rotårsaksanalyse er målet stort sett å generere en liste over problemområder som kan forbedres, identifisere mulige konsekvenser, generere en liste over mulige årsaker til problemet og oppmuntre til å tenke på løsninger som kan eliminere problemet. Verktøyene vi anbefaler til denne fasen er:

\subsubsection{Idémyldring} Idémyldring gir mange idéer om mulige rotårsaker. Vi brukte idémyldring på våres tre caser og fant verktøyet til å fungere strålende. Skulle det være noen i gruppen som dominerer myldringen, anbefales heller idéskriving. Dette verktøyet fungerer på samme måte som idémyldring, men istedenfor å si idéene høyt, blir de skrevet ned og samlet inn før de skrives på tavlen. 

\begin{description}
    \item[Steg 1:] Skriv opp problemstillingen en ønsker å ta utgangspunkt i på en tavle og la personene i gruppen komme med idéer. Husk på å ikke kritisere idéene til hverandre. Prosessen burde la seg dø ut en gang før man tar en runde til, for å sørge for at alle idéer er nevnt. 
    \item[Steg 2:] Sorter alle idéene i grupper og gå videre med de mest lovende idéene.
\end{description}

\subsubsection{Nominell gruppe teknikk} 
Dette verktøyet gir en liste over hva en burde prioritere mest i datainnsamling for å gi ett mer målrettet datagrunnlag for å finne rotårsaken. Vi anbefaler å bruke dette verktøy som et supplement, skulle man ha mange idéer fra idémyldringen og trenger å prioritere.
\begin{description}
    \item[Steg 1:] Gi hver idé fra idémyldringen hver sin bokstav.
    \item[Steg 2:] Gi alle gruppemedlemene hvert sitt stemmeark, med alle bokstavene på. 
    \item[Steg 3:] Gi fem idéer et tallpoeng fra 1-5 der 5 selvfølgelig er høyest.
    \item[Steg 4:] Tell opp poengsummene. De idéene med flest poeng prioriteres videre i prosessen.
\end{description}


\subsection{Datainnsamling}
Datainnsamling er et steg i prosessen der man skal være strukturert og samle inn så mye relevant informasjon om problemstillingen som mulig. En god datainnsamling er sentralt for gode resultater i senere faser. Vi anbefaler følgende verktøy:

\begin{description}
    \item[Sampling] Sampling er et godt verktøy for å begrense datainnsamling til en utvalgt del av en større gruppe. Brukes ofte i kombinasjon ble spørreundersøkelser eller sjekklister.
    \item[Spørreundersøkelser] Spørreundersøkelser er et godt verktøy for å få data fra de berørte personene.
\end{description}

\subsection{Dataanalyse}
I denne fasen blir dataene analysert og visualisert. Hovedmålet er å avklare mulige rotårsaker som har innvirkning på problemet, og hvilke av de som har størst innflytelse. Under beskrives de ulike verktøyene som ble brukt for å analysere dataene.
Når det gjelder histogram vil vi ikke skrive noe om stegene som må tas, da det kommer an på hva programvare som brukes. Verktøyene vi anbefaler til denne fasen er:

\begin{description}
    \item[Histogram] Histogram fungerer veldig godt for å skape en visuell forståelse av dataene, som kan gjøre det lettere å se korrelasjoner mellom variabler. Det gir også en tilfredstillende fremstilling av dataen. 
\end{description}

\subsubsection{Statistisk analyse}
Statistisk analyse er ikke i boken, men er en analyse metode som fungerer veldig godt for å se korrelasjoner. Dette fungerer også ved å se på signifikansen.
\begin{description}
    \item[Steg 1:] Gjør om alle svar til numerisk form
    \item[Steg 2:] Gjennomfør en bivariate korrelasjon på alle svarene
    \item[Steg 3:] På de som korrelerer, gjennomfør en One-Way ANOVA eller en uavhengig t-test avhengig om det er 2 eller flere grupper i den uavhengige variabelen.
    \item[Steg 4:] Lag gjerne et histogram på de som har signifikans for å visualisere korrelasjonen.
\end{description}

\subsubsection{Affinitetsdiagram} Affinitetsdiagram passer veldig godt med spørreundersøkelser, der det er kortsvar- eller langsvaroppgaver. Den gir mulighet for å gruppere etter innhold i svarene. 
\begin{description}
    \item[Steg 1:] Bruk data fra forrige fase til å komme fram til overordnede svar.
    \item[Steg 2:] Skriv svarene på post-it lapper.
    \item[Steg 3:] Grupper svarene. Ofte må lappene flyttes flere ganger før gruppene blir funnet. Det bør ikke overstige 5-10 grupper 
    \item[Steg 4:] Lag tittel på gruppene og lag grafikk. 
\end{description}

\subsection{Rotårsaksidentifisering}
De foregående fasene skal ha generert en liste over mulige rotårsaker og målet i denne fasen er å identifisere de faktiske årsakene. 

Verktøyene vi anbefaler til denne fasen er:

\subsubsection{Årsak-virkning diagram} Som årsak-virkning anbefaler vi å benytte fiskebeindiagram. Ved å bruke fiskebeindiagram får en en visuell fremstilling av rotårsaken til problemet.
    \begin{description}
        \item[Steg 1:] Definer tydelig hva problemstilingen er.
        \item[Steg 2:] Bruke Draw.io eller et annet tegneprogram til tegne fiskebeindiagramet. Kan også bruke penn og papir. 
        \item[Steg 3:] Identifiser hovedkategoriene for årsakene og tegn de opp som greiner.
        \item[Steg 4:] Idémyldre alle årsaker som kan knyttes til hovedkategoriene.
        \item[Steg 5:] Analyser årsakene for å identifisere det som mest sannsynlig er rotårsaken
    \end{description}
    
\subsubsection{5 Whys} 
Hvis det er mistanke om høyere nivå av årsaker bak de identifiserte årsakene kan 5 Whys gi en bekreftelse på om årsakene som er identifisert er faktisk rotårsaken og ikke lav-nivå årsaker. Det kan kreve flere iterasjoner for å finne rotårsaken(e).  
    \begin{description}
        \item[Steg 1:] Identifiser årsakene du ønsker å undersøke
        \item[Steg 2:] Spør hvorfor til hver årsak, og spør igjen for hver nye årsak som du kommer med.
        \item[Steg 3:] Hvis det ikke er noen annen forklaring på årsaken enn Gud, da har du funnet rotårsaken.
    \end{description}
Det kan ta både mindre eller mer enn fem ``hvorfor'' for å komme til rotårsaken, men det er en tommelfingerregel å utføre fem iterasjoner. 
    
\subsubsection{Feiltreanalyse} 
Kan brukes etter 5 Whys for å få en visuell presentasjon på hvordan de forskjellige rotårsakene man kom frem til i 5 Whys henger sammen.


\subsection{Rotårsakseliminering}
Denne fasen innebærer å komme med mulige løsninger til problemet for å eliminere rotårsaken. Boken til Fagerhaug og Andersen \cite{RCA} beskriver to mulige tilnærminger til denne fasen. En tilnærming for å stimulere kreativitet når man leter etter løsninger, som benytter verktøyet seks tenkehatter. Den andre for å konstruere og utvikle løsninger, som benytter TRIZ eller SIT. Vi vil i denne fasen primært anbefale å bruke de seks tenkehatter med de modifikasjonene vi har gjort istedenfor å bruke SIT.

Verktøyene vi anbefaler til denne fasen er:
\subsubsection{De seks tenkehatter} 
Seks tenkehatter fungerer godt for å idéemyldre løsninger når all dataen er klar og kreative løsninger trengs. Seks tenkehatter er også den måten vi tror passer best i infromasjonssikkerhetssammenheng.
\begin{description}
    \item[Hvit hatt] skal være kald, nøytral og objektiv, personen skal fokusere på fakta.
    \item[Rød hatt] skal representere sinne, og skal bare fokusere på magefølelsen og egne følelser.
    \item[Svart hatt] skal være pessimistisk og negativ, og fokusere på hvorfor idéen er dårlig.
    \item[Gul hatt] er optimistisk og positiv, og skal fokusere på hvorfor idéen er bra og vil fungere.
    \item[Grønn hatt] representerer gresset, fruktbarhet og vekst, og skal fokusere på å vøre kreativ og komme på nye idéer.
    \item[Blå hatt] er koblet til himmelen, og skal fokusere på å se tingene fra et høyere perspektiv.
\end{description}

\begin{description}
    \item[Steg 1:] Tildel hatter til diskusjonsgruppen. 
    \item[Steg 2:] Start diskusjonen ved at hvit hatt presenterer fakta om problemet som skal løses
    \item[Steg 3:] Grønn hatt presenterer idéer på hvordan problemet kan løses
    \item[Steg 4:] Diskuter de mulige løsningene, der gul hatt fokuserer på fordeler og svart på ulemper (alle kan delta i tillegg)
    \item[Steg 5:] Rød hatt skal lokke frem gruppens magefølelse på løsningene
    \item[Steg 6:] Blå hatt oppsummerer diskusjonen og stopper møtet
\end{description}
    
\subsubsection{SIT} 
Vi prøvde SIT i hver av casene våres for å se hvordan det fungerte. Vi kom fram til at SIT er et tungvint verktøy å starte, der mange av SIT-prinsippene ikke godt lar seg overføre til informasjonssikkerhet. Resten av stegene går bra og gir gode resultater i form av forslag på en tiltaksplan.  
    \begin{description}
        \item[Steg 1:] Ideelt sett bør man bruk et team som består av personer som innehaver all mulig informasjon om problemet. Er mulig uten, men blir langt vanskeligere å gjøre.
        \item[Steg 2:] List alle komponenten og sørg for å ta med de som virker irrelevant. 
        \item[Steg 3:] Bruk de fem SIT-prinsippene til å finne på løsninger:
                \begin{description}
                    \item[Attributtavhengighet] vurderer å endre en nøkkelvariabel i et produkt for å skape forbedring.
                    \item[Komponentkontroll] ser på hvordan et produkt er knyttet til omgivelsene.
                    \item[Erstatning] handler om å erstatte en del av et produkt med noe annet fra produktets omgivelser.
                    \item[Forkastning] vurderer å forbedre problemet ved å fjerne en komponent. 
                    \item[Oppdeling] har som mål å splitte et produkts attributter i to, som for eksempel splittelsen av sjampo fra balsam.
                \end{description}
        \item[Steg 4:] Velg de idéer som er best egnet for videre utdyping
        \item[Steg 5:] Forsett å utdype idéene og kom opp med en eller flere løsninger for å gå videre til en tiltaksplan.
    \end{description}

\subsection{Løsningsimplementering}
I den siste fasen er målet å implementere løsningene som ble funnet i foregående fase. Implementeringen inkluderer blant annet organisering, utvikling av en implementeringsplan, skape et konsensus om de nødvendige endringene og selvfølgelig implementeringen. Implementeringen av løsningen kan sies å være en suksess når symptomene forsvinner. 

Verktøyet vi anbefaler til denne fasen er:

\subsubsection{Trediagram}
Trediagram fungerer veldig bra for å lage en plan over hva som må gjøres for at tiltaket skal bli implementert. 
\begin{description}
    \item[Steg 1:] Lag en liste over gjøremål for å implementere løsningen(e)
    \item[Steg 2:] Disse skal grupperes og settes opp i en trestruktur, i rekkefølgen som skal til for å få løsningen implementert. Aktivitetene skal plasseres i 
\end{description}

I dette steget er det ikke nødvendig å bruke noe verktøy, men det kan være lurt å benytte trediagram for å få en plan over løsningimplementering.


\chapter{Konklusjon}
\label{kap:konklusjon}


%\input{main/}



\bibliographystyle{ntnuthesis/ntnubachelorthesis}
\bibliography{main/referanser}

\appendix %after this line all chapters will have leters instead of numbers
\chapter{Spørreundersøkelse case 1}
\label{sporreundersokelser}
\includepdf[pages={1-4}]{case_1/bilder/case1_sporreundersokelse}
\label{undersokelse}

\chapter{Spørreundersøkelse case 2 (norsk og engelsk)}
\includepdf[pages={1-6}]{case_2/bilder/sporreundersokelse_norsk.pdf}
\label{undersokelse_norsk}

\includepdf[pages={1-6}]{case_2/bilder/sporreundersokelse_engelsk.pdf}
\label{undersokelse_engelsk}

\chapter{Spørreundersøkelse case 2 resultater (norsk og engelsk)}
\includepdf[pages={1-4}]{case_2/bilder/sporreundersokelse_norsk_resultater.pdf}
\label{undersokelse_norsk_resultater}

\includepdf[pages={1-4}]{case_2/bilder/sporreundersokelse_engelsk_resultater.pdf}
\label{undersokelse_engelsk_resultater}
\chapter{Vedlegg: Plakat}
\label{plakat}
\begin{figure}[H]
    \centering
    \includegraphics[scale=0.25]{case_1/bilder/plakat.pdf}
    \caption[Promoteringsplakat]{Plakat som ble brukt i forbindelse med promotering av spørreundersøkelsen}
    \label{fig:plakat}
\end{figure}
\input{main/8_Vedlegg/3_frekvenstabeller_case1.tex}
\chapter{Vedlegg: Diverse histogrammer fra case 1}
\label{vedlegg:histogrammer}

\begin{figure}[H]
    \centering
    \includegraphics[scale=0.45]{case_1/bilder/IT_lasterned.pdf}
    \caption[Forskjellen mellom fakultetene og om de laster ned]{Forholdet mellom IT studier og andre når det kommer til nedlasting}
    \label{fig:IT-lasterned}
\end{figure}

\begin{figure}[H]
    \centering
    \includegraphics[scale=0.45]{case_1/bilder/reglement_lasterned.pdf}
    \caption[Forholdet mellom kjennskap til IT reglement og om de laster ned]{Forholdet mellom kjennskap til IT reglement og om en laster ned}
    \label{fig:reglement-lasterned}
\end{figure}

\begin{figure}[H]
    \centering
    \includegraphics[scale=0.45]{case_1/bilder/reglement_fakultet.pdf}
    \caption[Forskjellen mellom fakultetene og kjennskap til IT reglement]{Hvor godt kjennskap de ulike fakultetene har med IT reglementet}
    \label{fig:reglement-fakultet}
\end{figure}

\begin{figure}[H]
    \centering
    \includegraphics[scale=0.45]{case_1/bilder/antalltorrents.pdf}
    \caption[Antall torrents som blir lastet ned hver måned]{Hvor mange torrents folk laster ned i løpet av en måned}
    \label{fig:antalltorrents}
\end{figure}

\chapter{Vedlegg: Diverse histogrammer fra case 2}
\begin{figure}[H]
    \centering
    \includegraphics[scale=0.5]{case_2/bilder/spss/oppdaget_virus.pdf}
    \caption[Oppdaget virus]{Oversikt over hvor mange som oppdaget virus på datamaskinen}
    \label{fig:oppdaget-virus}
\end{figure}

\begin{figure}[H]
    \centering
    \includegraphics[scale=0.5]{case_2/bilder/spss/passord_deling.pdf}
    \caption[Oversikt over passorddeling]{Viser hvor mange som har delt sitt NTNU passord før}
    \label{fig:passord-deling}
\end{figure}

\begin{figure}[H]
    \centering
    \includegraphics[scale=0.5]{case_2/bilder/spss/tilfeldig_passord.pdf}
    \caption[Hvor mange som bruker tilfeldig passord]{Viser hvor mange som bruker tilfeldig sammensatt passord}
    \label{fig:tilfeldig-passord}
\end{figure}

\begin{figure}[H]
    \centering
    \includegraphics[scale=0.5]{case_2/bilder/spss/pass_manager.pdf}
    \caption[Hvor mange som bruker passordmanager]{Viser hvor mange som bruker passord manager}
    \label{fig:passord-manager}
\end{figure}
\chapter{Vedlegg: Statistisk analyse case 2}
\label{vedlegg:statanalys}

\section*{År ved NTNU mot bevissthet på sikkerhet og kjennskap til retningslinjer}
\label{aarvedNTNU-mot-bevissthetogkjennskap}
\begin{figure}[H]
    \centering
    \includegraphics[scale=0.7]{case_2/bilder/spss/anova_ttest/ansiennitet_bevissthetogkjennskap_descriptive_1.pdf}
    \caption[Antall år ved NTNU mot bevissthet og kjennskap descriptive 1]{Descriptive av tid ved NTNU mot bevissthet på sikkerhet og kjennskap til retningslinjer, del 1}
    \label{fig:ansiennitet-bevissthetogkjennskap-descriptive-1}
\end{figure}

\begin{figure}[H]
    \centering
    \includegraphics[scale=0.7]{case_2/bilder/spss/anova_ttest/ansiennitet_bevissthetogkjennskap_descriptive_2.pdf}
    \caption[Antall år ved NTNU mot bevissthet og kjennskap descriptive 2]{Descriptive av tid ved NTNU mot bevissthet på sikkerhet og kjennskap til retningslinjer, del 2}
    \label{fig:ansiennitet-bevissthetogkjennskap-descriptive-2}
\end{figure}

\begin{figure}[H]
    \centering
    \includegraphics[scale=0.7]{case_2/bilder/spss/anova_ttest/ansiennitet_bevissthetogkjennskap_anova.pdf}
    \caption[Antall år ved NTNU mot bevissthet og kjennskap ANOVA]{Anova av tid ved NTNU mot bevissthet på sikkerhet og kjennskap til retningslinjer}
    \label{fig:ansiennitet-bevissthetogkjennskap-anova}
\end{figure}

\begin{figure}[H]
    \centering
    \includegraphics[scale=0.7]{case_2/bilder/spss/anova_ttest/ansiennitet_bevissthetogkjennskap_posthoc_1.pdf}
    \caption[Antall år ved NTNU mot bevissthet og kjennskap post-hoc 1]{Post-hoc av tid ved NTNU mot bevissthet på sikkerhet og kjennskap til retningslinjer, del 1}
    \label{fig:ansiennitet-bevissthetogkjennskap-posthoc-1}
\end{figure}

\begin{figure}[H]
    \centering
    \includegraphics[scale=0.7]{case_2/bilder/spss/anova_ttest/ansiennitet_bevissthetogkjennskap_posthoc_2.pdf}
    \caption[Antall år ved NTNU mot bevissthet og kjennskap post-hoc 2]{Post-hoc av tid ved NTNU mot bevissthet på sikkerhet og kjennskap til retningslinjer, del 2}
    \label{fig:ansiennitet-bevissthetogkjennskap-posthoc-2}
\end{figure}

\section*{Kjønn mot bevissthet på sikkerhet og kjennskap til retningslinjer}
\label{kjonn-mot-bevissthetogkjennskap}
\begin{figure}[H]
    \centering
    \includegraphics[scale=0.8]{case_2/bilder/spss/anova_ttest/kjonn_bevissthetogkjennskap_groupstats.pdf}
    \caption[Gruppestatistikk av kjønn mot bevissthet på sikkerhet og kjennskap til retningslinjer]{Gruppestatistikk av kjønn mot bevissthet på sikkerhet og kjennskap til retningslinjer}
    \label{fig:kjonn-bevissthetogkjennskap-groupstats}
\end{figure}

\begin{figure}[H]
    \centering
    \includegraphics[scale=0.4]{case_2/bilder/spss/anova_ttest/kjonn_bevissthetogkjennskap_ttest.pdf}
    \caption[T-test av kjønn mot bevissthet på sikkerhet og kjennskap til retningslinjer]{Independent t-test av kjønn mot bevissthet på sikkerhet og kjennskap til retningslinjer}
    \label{fig:kjonn-bevissthetogkjennskap-ttest}
\end{figure}
\chapter{Vedlegg: Flytdiagrammer for verktøyvalg}
\label{flytdiagrammer-verktoyvalg}

\begin{figure}[H]
    \centering
    \includegraphics[scale=0.35]{main/bilder/verktoyvalg/verktoyvalg_problemforstaelse.pdf}
    \caption[Verktøyvalg for problemforståelse]{De ulike verktøyene boken anbefaler i problemforståelse ut fra spesifikke kriterier}
    \label{fig:verktoyvalg-problemforstaelse}
\end{figure}

\begin{figure}[H]
    \centering
    \includegraphics[scale=0.65]{main/bilder/verktoyvalg/verktoyvalg_idemyldring.pdf}
    \caption[Verktøyvalg for idémyldring]{De ulike verktøyene boken anbefaler i idémyldring ut fra spesifikke kriterier}
    \label{fig:verktoyvalg-idemyldring}
\end{figure}

\begin{figure}[H]
    \centering
    \includegraphics[scale=0.65]{main/bilder/verktoyvalg/verktoyvalg_datainnsamling.pdf}
    \caption[Verktøyvalg for datainnsamling]{De ulike verktøyene boken anbefaler i datainnsamling ut fra spesifikke kriterier}
    \label{fig:verktoyvalg-datainnsamling}
\end{figure}

\begin{figure}[H]
    \centering
\includegraphics[scale=0.65]{main/bilder/verktoyvalg/verktoyvalg_dataanalyse.pdf}
    \caption[Verktøyvalg for dataanalyse]{De ulike verktøyene boken anbefaler i dataanalyse ut fra spesifikke kriterier}
    \label{fig:verktoyvalg-dataanalyse}
\end{figure}

\begin{figure}[H]
    \centering
    \includegraphics[scale=0.65]{main/bilder/verktoyvalg/verktoyvalg_rotarsaksidentifisering.pdf}
    \caption[Verktøyvalg for rotårsaksidentifisering]{De ulike verktøyene boken anbefaler i rotårsaksidentifisering ut fra spesifikke kriterier}
    \label{fig:verktoyvalg-rotarsaksidentifisering}
\end{figure}

\begin{figure}[H]
    \centering
    \includegraphics[scale=0.65]{main/bilder/verktoyvalg/verktoyvalg_rotarsakseliminering.pdf}
    \caption[Verktøyvalg for rotårsakseliminering]{De ulike verktøyene boken anbefaler i rotårsakseliminering ut fra spesifikke kriterier}
    \label{fig:verktoyvalg-rotarsakseliminering}
\end{figure}

\begin{figure}[H]
    \centering
    \includegraphics[scale=0.65]{main/bilder/verktoyvalg/verktoyvalg_losningsimplementering.pdf}
    \caption[Verktøyvalg for løsningsimplementering]{De ulike verktøyene boken anbefaler i løsningsimplementering ut fra spesifikke kriterier}
    \label{fig:verktoyvalg-losningsimplementering}
\end{figure}

\end{document}
