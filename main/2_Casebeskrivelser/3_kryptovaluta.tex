\section{Case 3: Misbruk av NTNU sin infrastruktur til utvinning av kryptovaluta}
\label{sec:case_krypto}
Kryptovaluta har kommet mer i nyhetsbildet i det siste. Kryptovaluta er en anonym måte å betale og man kan utvinne kryptovaluta ved hjelp av mining. Miningen kan bli gjort på flere måter, en kan installere et program på pcen eller det man kan mine direkte gjennom browseren med javascript, sistnevnte kan være som erstatning til reklame, eller en ondsinnet måte for nettsider/andre å tjene penger på.



\subsubsection{Oversikt over oppgaven}
Dette caset går inn på rotårsaken til misbruk av NTNU sine ressurser og infrastruktur til å utvinne kryptovaluta. De to siste årene har både verdien og antallet kryptovaluta økt drastisk. Det finnes per dags dato over 1500 forskjellige kryptovalutaer.NB! KILDE! 
Kryptovaluta blir "minet", eller utvinnet, ved bruk av regnekraft. Dette betyr at enhver datamaskin kan delta i utvinningen. Siden november 2017 har NTNU sett en økning i mining med 8000\% og får i dag flere varsler angående mining om dagen. Siden universitetet ikke har lov til å bruke sine ressurser til kommersiell vinning er dette blitt et stort problem. 

Grunnen til økningen er at i 2017 begynte flere å spekulere i kryptovaluta og siden det er et uregulert marked var det veldig store svinginger som igjen førte til at det kom mange flere aktører på banen både som leverandører og spekulanter. 

Etterhvert vil vanskelighetsgraden for å utvinne nye mynter øke. Når vanskelighetsgraden øker trenger en mer datakraft og større maskinrigger til å utvinne valutaene. 

NTNU forvalter stor regnekraft spredt på flere lokasjoner. NTNU har også hatt supermaskiner før, de har en nå og de får nå en ny supermaskin. Supermaskiner er store datamaskiner med enorm datakraft. Disse er spesielt attraktive for aktører å misbruke til å utvinne kryptovaluta. Siden trenden har økt de siste årene, og NTNU er i besittelse av mye regnekraft, må NTNU aktivt jobbe for å beskytte infrastrukturen. 

Siden dette er av økende trend, og Seksjon for Digital Sikkerhet har oppdaget at noe av universitetet sine ressurser har blitt brukt til utvinning av kryptovaluta, vil de undersøke måter å eliminere dette misbruket. 

Caset sine fokusområder deles inn i to aktører. De som frivillig bestemer seg for å drive med utvinning av kryptovaluta, og de som får ressursene sine misbrukt til å utvinne for kriminelle aktører, der de ikke samtykker til bruk av sine ressurser.

Caset går ut på å identifisere rotårsaken til misbruk av NTNU sine ressurser til utvinning av kryptovaluta, og foreslå tiltak for å eliminere den.
