\section{Case 3: Misbruk av NTNU sin infrastruktur til utvinning av kryptovaluta}
\label{sec:case_krypto}
Kryptovaluta har kommet mer i nyhetsbildet i det siste. Kryptovaluta er en anonym måte å betale og man kan utvinne kryptovaluta ved hjelp av mining. Miningen kan bli gjort på flere måter, en kan installere et program på pcen eller det man kan mine direkte gjennom browseren med javascript, sistnevnte kan være som erstatning til reklame, eller en ondsinnet måte for nettsider/andre å tjene penger på.




\subsubsection{Oversikt over oppgaven}
Siden november 2017 har mengeden med mining på skolen økt med 8000\%, siden universitetet ikke har lov til å bruke sine ressurser til personlig vinning er dette blitt et stort problem. Grunnen til den økningen er at i 2017 begynte flere å spekulere i kryptovaluta og siden det er et uregulert marked var det veldig store svinginger som igjen førte til at det kom mange flere aktører på banen både som leverandører og spekulanter. 

Oppgaven deles inn i to aktører de som frivillig bestemer seg for å drive med mining og de som får ressursene sine missbrukt til å mine for kriminelle selskaper eller nettsider, der de ikke sier ja til bruk av sine ressurser.

Oppgaven har to spørsmål som må besvares og de er hvorfor og hvordan utvinning av kryptovaluta foregår.