\section{Case 3: Misbruk av NTNU sin infrastruktur til utvinning av kryptovaluta}
\label{sec:case_krypto}
Kryptovaluta har kommet mer i nyhetsbildet i det siste. Kryptovaluta er en anonym måte å betale og man kan utvinne kryptovaluta ved hjelp av regnekraft. Utvinningen kan bli gjort på flere måter. En kan installere et program på PCen eller det man kan utvinne direkte gjennom nettleseren med javascriptkode. Sistnevnte kan være en erstatning til reklame, eller en ondsinnet måte for nettsider eller andre å tjene penger på.



\subsubsection{Oversikt over oppgaven}
Dette caset går inn på rotårsaken til misbruk av NTNU sine ressurser og infrastruktur for å utvinne kryptovaluta. De to siste årene har både verdien og antallet kryptovaluta økt drastisk. Det finnes per dags dato over 1500 forskjellige kryptovalutaer \cite{Cryptocurrency}. 
Kryptovaluta blir utvinnet ved bruk av regnekraft. Dette betyr at enhver datamaskin kan delta i utvinningen. Siden november 2017 har NTNU sett en økning i kryptoutvinning med 8000\% \footnote{Møte med oppdragsgiver} og får i dag flere varsler angående kryptoutvinning om dagen. Siden universitetet sine ressurser ikke skal brukes til kommersiell vinning er dette blitt et stort problem \cite{ITReg}. 

Grunnen til økningen er at i 2017 begynte flere å spekulere i kryptovaluta og siden det er et uregulert marked var det veldig store svingninger. Svingninger som igjen førte til at det kom mange flere aktører på banen, som både leverandører og spekulanter. 

Etter hvert vil vanskelighetsgraden for å utvinne nye mynter øke. Når vanskelighetsgraden øker trenger en mer datakraft og større maskinrigger til å utvinne valutaene \footnote{Informasjon fra oppdragsgiver}. 

NTNU forvalter stor regnekraft spredt på flere lokasjoner. NTNU har også hatt supermaskiner før; de har en nå og de får også en ny supermaskin. Supermaskiner er store datamaskiner med enorm datakraft. Disse er spesielt attraktive for aktører å misbruke til å utvinne kryptovaluta. Siden trenden har økt de siste årene, og NTNU er i besittelse av mye regnekraft, må NTNU aktivt jobbe for å beskytte infrastrukturen. 

Siden dette er av økende trend, og Seksjon for Digital Sikkerhet har oppdaget at noe av universitetet sine ressurser har blitt brukt til utvinning av kryptovaluta tidligere, vil de undersøke måter å eliminere dette misbruket. 

Caset deles inn i to fokusområder. Frivillig og ufrivillig utvinning av kryptovaluta. Frivillig utvinning mener vi interne aktører som bruker NTNU sine ressurser til egen vinning. Med ufrivillig utvinning mener vi de som får ressursene sine misbrukt uten samtykke. 

Caset går ut på å identifisere rotårsaken til misbruk av NTNU sine ressurser til utvinning av kryptovaluta, og foreslå tiltak for å eliminere den.
