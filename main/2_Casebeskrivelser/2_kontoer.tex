\section{Case 2: Kompromitterte kontoer}
\label{sec:case_kontoer}
NTNU er et stort universitet med mange ansatte. NTNU er derfor et mål for aktører med ondsinnete hensikter. En av disse er hensiktene er å få tilgang til brukerkontoer. Disse kontoene blir brukt til mye forskjellig. De brukes ofte for spam, innhenting av store mengder forskningsartikler og videre salg. Måten disse kontoene kommer på avveie er uvisst, men noen av hypotesene innebærer phishing og gjenbruk av brukernavn og passord på nettsider som har blitt kompromittert. Vår oppgave i dette caset er å bruke rotårsaksanalyse til å finne og presentere tiltak som eliminerer rotårsaken til at kontoer ved NTNU blir kompromittert.

\subsubsection{Oversikt over oppdraget}
NTNU har siden 2005 fått 5415 kontoer kompromittert. Disse kontoene ble funnet i en stor datadump i desember 2017. Av disse 5415 kontoene var 101 av dem fremdeles aktive. Det vil si at brukernavn og passord fortsatt var gyldige innloggingskredentialer da de ble avdekket. 

Universitetet betaler for tilgang til databaser som inneholder tusenvis av forskningsartikler og andre artikler. 26 av de kompromitterte brukerne ble brukt av utenlandske aktører for å skaffe seg tilgang til forskningsartikler. Konsekvensen ved å ha kompromitterte kontoer som laster ned forskningsartikler er ikke bare at NTNU taper penger, men at NTNU risikerer å bli blokkert fra databasene. Angriperne derimot tjener på å ikke trenge å betale for tilgang til databasene. Et annet punkt som gjør det lukerativt å kompromittere universitetskontoer er at disse kontoene kan bli solgt på nett, der kredentialer behandles som ferskvare. 

Det blir hentet ut flere tusen forskningsartikler på en gang per bruker. Universitetet vet ikke om når det blir hentet ut artikler, eller om det er legitimt bruk av artiklene. De får ofte vite det først når universitetet får beskjed av samarbeidspartnere til NTNU, for eksempel de som tilbyr artiklene. 
