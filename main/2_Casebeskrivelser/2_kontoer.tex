\section{Case 2: Kompromitterte kontoer}
\label{sec:case_kontoer}
NTNU er et stor universitet med mye forskjellige folk fra forskjellige bakgrunner og fagkunnskap om vidt forskjellige ting. NTNU har som følge av denne mengden med folk et problem med at kontoer kan komme på avveie. Disse kontoene blir brukt til mye forskjellig, fra spam til å hente ned store mengder med forskningsartikler. Måten disse kontoene kommer på avveie er noe uvisst, men det spekuleres i at det er mye phising. Vår oppgave i dette oppdraget var, å bruke rotårsaksanalyse til å finne og eliminere rotårsaken til at kontoer blir kompromitert.

\subsubsection{oversikt over oppdraget}
NTNU har siden ca.2005 fått 5415 kontoer kompromitert disse kontoene ble funnet i en stor datadump i desember 2017, av disse 5415 kontoene var 101 av dem fremdeles aktive, det vil si at folk kunne finne disse 101 folkene sine kontoer og logge seg på på skolenettet med brukerinformasjonen deres. Dette koster skolen en del penger i form av forskningartiklene som bli lastet ned, der hver artikkel koster skolen ---insert cost of each article when they get crawled.-----. Et annet punkt som gjør det å kompromitere kontoer ved skolen lukerativt er at disse kontoene kan bli solgt på brukerinformasjons marked, der de behandles som ferskvare.