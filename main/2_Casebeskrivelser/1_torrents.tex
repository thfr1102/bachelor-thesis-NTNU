\chapter{Casebeskrivelser}
\label{kap:casebeskrivelser}
Rotårsaksanalyse er foreløpig lite brukt i forbindelse med informasjonssikkerhet, og oppdragene som blir sett på i denne rapporten er gjort ved hjelp av rotårsaksanalyse metodikk. Dette for å få en større forståelse for rotårsaksanalyse i sammenheng med det digitale og spesielt da informasjonssikkerhet.

\section{Case 1: Ulovlig fildeling på skolenettet}
\label{sec:case_fildeling}
Ved NTNU Gjøvik er det NTNU som har ansvar for nettet hos studenthyblene studentene leier i byen. Dette fører med seg et problem i form av mye piratnedlastning på skolenettet. Oppgaven vår i dette caset var å finne, ved hjelp av rotårsaksmetodikken, hvordan skolen skal kunne stoppe studentene fra å laste ned opphavsrettighetsbeskyttet materiale.

\subsubsection{Oversikt over oppdraget}
Advokater til diverse filmselskaper ser etter IP-er til personer som laster ned deres opphavsrettsbeskyttede materiale, og sender disse personene mail. NTNU for flere hunde slike mailer i måneden, bortsett fra om sommeren da studenter er på ferie. Piratnedlastning foregår ved hjelp av en protokoll som heter bit-torrent protokkollen og den bruker peer-to-peer teknologi.  Hvis disse opphavsrettshaverne finner ut at de skal håndheve brevene de sender, kommer NTNU til å være i en dårlig posisjon, der de per idag ikke gjør noe med de mange brevene de får. Grunnen til at de ikke gjør noe er de store mengdene og at det hadde vært en fulltidsjobb i seg selv å håndtere dem.