\chapter{Casebeskrivelser}
\label{kap:casebeskrivelser}
Oppgavene som blir analysert i denne rapporten er gjort ved hjelp av rotårsaksanalysemetodikk i henhold til ``Root Cause Analysis: Simplified Tools and Techniques - second edition'' av Bjørn Andersen og Tom Fagerhaug \cite{RCA}. Dette for å få en større forståelse for rotårsaksanalyse i sammenheng med det digitale og spesielt da informasjonssikkerhet.

\section{Case 1: Ulovlig fildeling på universitetsnettet}
\label{sec:case_fildeling}
NTNU har ansvar for nettet hos studenthyblene studentene leier fra SiT. Når studenter driver med ulovlig fildeling fra hybelen vil NTNU kunne holdes ansvarlig. Dette er et problem for NTNU, ikke bare i form av skadet omdømme, men det kan også være problematisk hvis opphavsrettshaverne går juridisk til verks. Oppgaven i dette caset er derfor å finne, ved hjelp av rotårsaksanalysemetodikken, rotårsaken til hvorfor studenter laster ned opphavsrettighetsbeskyttet materiale. Det skal også presenteres en tydelig tiltaksplan for å eliminere denne.

\subsubsection{Oversikt over oppgaven}
Advokater til diverse filmselskaper ser etter IP-er til personer som laster ned deres opphavsrettsbeskyttede materiale, og sender disse personene notifikasjoner på e-post. Hver måned får NTNU ca. 150 og 200 unike notifikasjoner om brudd på opphavsretten ved ulovlig fildeling. Vi kan se at dette nummeret går kraftig ned i sommermånedene, som tilsier at studentene er hovedgrunnen til bruddene på opphavsrett. Dersom opphavsrettshaverne begynner å håndheve opphavsretten i brevene de sender, kommer NTNU til å havne i en dårlig posisjon. I dag gjør ikke NTNU noe med de mange brevene de får, grunnet de store mengdene og at det er en fulltidsjobb i seg selv å håndtere dem.

\begin{figure}[H]
    \centering
    \includegraphics[scale=0.8]{case_1/bilder/copyright.jpg}
    \label{fig:case_1/bilder/copyright.jpg}
    \caption[Copyright Infringement Notifications]{Oversikt over antall notifikasjoner i sommerhalvåret}
\end{figure}

%Ulovlig fildeling foregår som regel ved hjelp av en protokoll som heter BitTorrent som bruker peer-to-peer teknologi. Universitetet har per dags dato ikke lov til å blokkere denne da protokollen også innebærer legitimt bruk. NTNU er eneste internettleverandør til studenthyblene og studenter kan derfor ikke velge hvilken leverandør de ønsker seg.