\chapter{Konklusjon}
\label{kap:konklusjon}
Rotårsaksanalyse er ikke en standardisert metodikk. Det finnes mye som blir kalt rotårsaksanalyse, men dette prosjektet tok utgangspunkt i metodikken til Fagerhaug og Andersen \cite{RCA}. Hovedformålet med prosjektet var å undersøke om rotårsaksanalysemetodikk har bruksområder innen informasjonssikkerhet. Tilnærmingen var gjennom tre caser der målet var å finne rotårsaken og gi forslag til tiltak som kunne eliminere disse. Ut fra disse kriteriene ble det definert noen forskningsspørsmål. Tre av de var å finne rotårsaken til: ulovlig fildeling ved universitetsnettet, kompromitterte kontoer ved NTNU og misbruk av NTNU sine ressurser til utvinning av kryptovaluta. De siste forskningsspørsmålene gikk på å vurdere hvor godt rotårsaksanalyse fungerer innen informasjonssikkerhet, om det er lønnsomt å bruke det og hvilke verktøy som fungerer best innen informasjonssikkerhet. 
\newline

\noindent Rotårsaken til ulovlig fildeling viste seg i hovedsak å være tilgjengeligheten, eller mangelen på den. En mindre årsak var at de følte ikke tjenestene var verdt det de måtte betale. En siste årsak ble funnet, nemlig at håndhevingen og kommuniseringen av lovene knyttet til ulovlig fildeling er utilstrekkelig og blir ikke prioritert. Tiltak vi har foreslått for å løse dette problemet er å tilby produkter fra selskaper som sender mest notifikasjoner om brudd på opphavsrett, å ha et kurs for studenter i bruk av universitetets nettverk, at Sit regelrett bytter ISP slik at problemet blir overført til andre og tilslutt at alt materiale blir tilgengelig på ett sted. Det siste tiltaket er urealistisk, men vil fjerne rotårsaken helt. 
\newline

\noindent Når det kommer til kompromitterte kontoer konkluderer vi med at både gjenbruk av kredentialier og phishing er de største rotårsakene. I tillegg resulterte analysen også i at folk som har blitt kompromittert kjenner svært dårlig til styrende dokumenter på IT, informasjonssikkerhet og behandling av autentiseringsdata. Vi kom også frem til at det var utilstrekkelig tilgangskontroll på brukerkontoene da vi mener brukernavn og passord ikke er nok. For å løse dette anbefaler vi en rekke tiltak. Disse inkluderer: en bevisstgjørelseskampanje for god e-postskikk og behandling av autentiseringsdata, krav om strengere passordkontroll, implementere 2-faktor autentisering, klarere enheter for en viss periode gjennom 2-faktor autentisering og informering om innlogginger fra andre maskiner, utbedre IT-reglementet til å inkludere og samle retningslinjer og krav, og til slutt å anbefale eller pålegge brukere å benytte seg av passordmanager. 
\newline

\noindent På caset om misbruk av ressurser til kryptoutvinning kom vi frem til at rotårsaken av en blanding av uklarheter i IT-reglementet angående kryptoutvinning og at Seksjon for Digital Sikkerhet ikke har ressurser til å prioritere problemet. For å løse problemet anbefaler vi å gjennomføre en informasjonskampanje om kommersielt misbruk av NTNU sin infrastruktur, gjøre IT-reglementet klarere på misbruk av ressursene spesifikt når det kommer til kryptoutvinning, DNS blokkering av kryptoforespørsler og tilsutt øke antall personell i seksjonen, enten i form av faste ansatte eller flere bacheloroppgaver. 
\newline

\noindent Basert på erfaring fra utføringen av casestudiene kan vi konkludere med at rotårsaksanalyse fungerer godt innen informasjonssikkerhet. Dette kommer selvfølgelig helt an på hvor god datainnsamlingen er. Det kommer også an på hva slags case det benyttes på, da noen har en rotårsak som ikke kan fjernes. Rotårsaksanalyse gir også mulighet for forskjellige resultater da rotårsaken ofte ikke er den du ser med første øyekast. For å konkludere endelig med hvor godt det fungerer må tiltakene innføres. Deretter kan man se om de fjerner problemet eller ikke. 
\newline

\noindent Rotårsaksanalyse er en strukturert problemløsningsmetode som egner seg godt ved lengre tidsbruk, men også helt greit over kortere tid. Det er spesielt viktig å ikke velge et for komplisert case hvis det er liten tid, da rotårsaksanalysen kan fort bli oppstykket og uferdig. Den største styrken ved rotårsaksanalyse er evnen den gir til å sette seg dypt inn i en problemstilling. Dette er nyttig uansett om det er mulig å fjerne rotårsaken eller ikke. I visse situasjoner kan det være mer fordelaktig å utføre en risiko- og sårbarhetsanalyse, men da kan du miste den dype forståelsen av bakgrunnen til problemet. Tidsbruken gjenspeilet stort sett resultatene vi fikk. Inkludert har vi også en veiledning som skal hjelpe med å veilede personer som ønsker å utføre rotårsaksanalyser på caser som omhandler informasjonssikkerhet. Dette dokumentet beskriver hvilke verktøy en bør bruke i ulike sammenhenger. Veiledningen finnes i kapittel \ref{kap:kap:veiledning-RCA}. 

\section{Videre arbeid}
Vi anbefaler at videre arbeid blir å gjennomgå tiltaksforslagene våre, og se om noen av disse er fornuftige å implementere, samt om de er verdt kostnadene. Ulovlig fildeling på skolenettet er et veldig vanskelig problem å fjerne rotårsaken til, siden tilgjengelighet er den store drivkraften bak nedlasting. Det kan også være nyttig å undersøke hvilke opphavsrettshavere som sender notifikasjoner, for å se om skolen kan tilby tjenester hvis de fleste kommer fra ett sted. Videre arbeid kan også inkludere en risikoanalyse for å underbygge problemstillingen knyttet til ulovlig fildeling på universitetsnettet. 

Siden vi bare tok et sample fra de som tidligere hadde blitt kompromittert kan det være interessant å undersøke hele NTNU når det kommer til passordvaner, e-post, kjennskap til retningslinjer osv. Deretter kan resultatene sammenlignes og se om det er noen forskjeller som bør tas i betraktning. Annet videre arbeid kan være å undersøke keylogging som en mulig årsak til kompromitterte kontoer. Vi gikk ikke så mye inn på det i denne rapporten, men det kan være interessant å se på i forlengelse av ondsinnet programvare. 

Siden datagrunnlaget vårt var noe snevert på case 3 kan det være interessant å hente inn mer data, for å se om det er andre underliggende rotårsaker vi ikke fant. Retningslinjer for bruk utvinning av kryptovaluta burde i teorien få alle ansatte eller studenter til å slutte å utvinne. Men de er mennesker så det kan være greit å se etter flere tekniske løsninger som hinder uautorisert utvinning av kryptovaluta. 

Kost-nytte på tiltakene
videre arbeid på veiledninger