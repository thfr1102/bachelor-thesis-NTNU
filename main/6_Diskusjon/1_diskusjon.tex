\chapter{Diskusjon}
\label{kap:diskusjon}
Utgangspunktet i denne drøftingen er hvorvidt rotårsakene vi kom frem til i hvert case er reelle rotårsaker som, hvis fjernet, vil fjerne symptomene helt. I tillegg ser vi på hvordan erfaringen fra de tre casene viser hvor godt rotårsaksanalyse fungerer innen informasjonssikkerhet, om det lønner seg å bruke og hvilke verktøy som fungerer best innen informasjonssikkerhet. 

\section{Hva er rotårsaken til at studenter laster ned opphavsrettsbeskyttet materiale?}
\subsection*{Rotårsak: Dårligere utvalg på alternative tjenester i Norge}
Ut fra vår analyse viser det seg at dette er hovedårsaken til at studenter laster ned ulovlig. Siden Netflix og de andre strømmetjenestene har inntatt markedet har filmer og serier gruppert seg mellom de. Tjenestene ønsker også flest mulig orginale serier som bare er hos dem. Dette gjør at tilgjengeligheten på filmer og serier går ned, med mindre man abonnerer på alt. Men selv da får man ikke tilgang på alt. Mange filmer og serier er geografisk blokkert i Norge, som gjør tilgjengeligheten til et enda større problem. I musikkstrømmingsmiljøet er problemet noe mindre. Selv om ulovlig musikknedlasting ikke er borte, har det blitt redusert \cite{musikkstream}. Noe av grunnen til dette er at musikkbransjen er mer sentralisert i hvem som eier rettighetene, også kjent som et oligopol. Dette gjør det lettere for strømmetjenestene å skaffe lisenser for musikk, og kan tilby det folk trenger på ett sted. Det er også mye mindre orginalt innhold i disse tjenestene i forhold til strømmetjenester for filmer og serier. 

\subsection*{Rotårsak: Tjenestene er ikke verdt prisen}
Jo flere tjenester det blir, jo mer må man betale for å få tilgang på mer materiale. Filmer og serier blir spredt utover markedet på flere tjenester som så og si koster det samme. Dette fører til at hver enkelt tjeneste blir mindre verdt pengene man må betale for å få tilgang. Det skal sies at strømming er en revolusjonerende løsning i forhold til å kjøpe hver enkelt film for seg selv, men hvis man må betale for fem forskjellige strømmetjenester for å få tilgang til det man har lyst på, hvorfor ikke bare laste ned gratis? Analysen vår viste at det å betale for tjenester ikke var noe problem for studentene; problemet var at de ikke føler de får det de betaler for. 

\subsection*{Rotårsak: Håndheving og kommunisering av lovene knyttet til ulovlig fildeling blir ikke prioritert}
Det eksisterer allerede regler på ulovlig nedlasting på universitetsnettet. Problemet er derimot at det er vanskelig å håndheve de. Andre arbeidsoppgaver har heller blitt prioritert. Enkelte tiltak har heller ikke vært lovlige for NTNU å gjennomføre for å stoppe de som driver med ulovlig fildeling. For eksempel er det ikke lov å overvåke enkeltboliger hos Sit, og heller ikke straffe enkeltpersoner dersom de laster ned, siden det blir regnet som inngrep i den private sfæren. Dette har datatilsynet fortalt Seksjon for Digital Sikkerhet. 


\section{Hva er rotårsaken til at brukerkontoer ved NTNU blir kompromittert?}

\subsection*{Rotårsak: Gjenbruk av brukerkredentialier på tredjepartssider}
Vi har vurdert gjenbruk av brukerkredentialier på andre tjenester som den mest relevante rotårsaken til at NTNU sine brukerkontoer blir kompromittert. Dette gjør vi på bakgrunn av at det var over halvparten som hadde svart at de hadde brukt sine NTNU kredentialier på flere tjenester. Dette betyr ikke nødvendigvis at det er den rotårsaken en bør frykte mest. I en studie gjort på oppdrag fra Google – som tok utgangspunkt i e-postadresser – viste det seg at selv om studien fastslo at det var desidert flest som var blitt kompromittert av datainnbrudd på andre tjenester, hadde flere hadde byttet passord siden de var blitt kompromittert, sammenlignet med de som hadde blitt kompromittert av phishing \cite{46437}. Likevel viser studien også den store mengden kontoer som blir kompromittert som følger av datainnbrudd ved andre tjenester, som bekrefter at det fortsatt er et stort problem. 

\subsection*{Rotårsak: Phishing}
Phishing var en av årsakene som ble belyst, og det viste seg at brukerne ikke hadde fått tilstrekkelig opplæring i deteksjon av phishing e-post. Phishing er, og har lenge vært, en stor årsak til kompromitterte kontoer \cite{SophPhish}. Phishing skjer også svært hyppig; undersøkelsen vår viste at de aller fleste hadde lagt merke til flere hendelser med phishing på sin NTNU e-post. Phishing kan være vanskelig å gjøre noe med. Vår formening er at det alltid vil være en risiko, uansett hva slags tiltak en implementerer. På en side kan både tekniske og bevissthetsmessige tiltak hjelpe, men disse vil aldri fjerne rotårsaken helt. 

\subsection*{Rotårsak: For dårlig kjennskap til styrende dokumenter}
Det er alltid en vanskelig oppgave å gjøre de ansatte oppmerksom på beste praksis innen informasjonssikkerhet. Dette gjelder også NTNU siden det i resultatene våre ble fremhevet at de ansatte hadde liten kjennskap til reglementer, retningslinjer og prinsipper knyttet til IT og informasjonssikkerhet. Det er imidlertid en pågående debatt om det i det hele tatt er verdt tiden og pengene i å forsøke å trene opp ansatte. Mange mener disse pengene kan bli bedre brukt på andre vis. Bruce Schneider skriver i sin blogg at dette er bortkastet tid og penger \cite{SecAware}. Mange er enige med han, men det er også mange eksperter som mener det er nyttig. Vi mener derimot at det er nyttig, men ressursbruken på dette burde holdes lav. 

\subsection*{Rotårsak: Utilstrekkelig tilgangskontroll på brukerkontoer}
Vi har kommet frem til at dette er et problem som kan løse mange av symptomene ved hjelp av tilgangskontroll. 2FA med sms er noe av det som blir anbefalt av oss. Dersom 2FA blir benyttet vil det hindre de fleste kontoer i å bli kompromittert, selv om kredentialiene blir kjent for trusselaktørene. Det er imidlertid mange som mener at SMS-meldinger er en usikker løsning på 2FA, siden SMS-meldinger er relativt enkelt å avlytte \cite{2FA}. De fleste anbefaler enten autentisering gjennom applikasjon eller fysisk kodebrikke. Disse metodene er dessverre noe vanskeligere å implementere, og det er heller ikke alle som har en smarttelefon som kan bruke applikasjonene som kreves. Et annet tiltak som blir mye brukt ellers er å validere brukerkontoen for spesifikke maskiner når de logges på for første gang, eller bare gi beskjed om ny innlogging et annet sted slik at en blir oppmerksom på at kontoen kan være kompromittert. Disse brukes av flere tjenester for å informere om og hindre kontoer fra å bli kompromittert. Google gir deg både beskjed når nye innlogginger finner sted, og gir deg muligheten til å legge til klarerte enheter \cite{trustcomp}. Dette fungerer ofte som en erstatning til 2FA hver gang du logger på. Siden dette har vært effektivt i andre sammenhenger ser vi ingen grunn til at dette ikke vil fungere bra hos NTNU, annet enn den ekstra anstrengelsen for brukerne når de logger på. 


\section{Hva er rotårsaken til misbruk av NTNU sin infrastruktur til utvinning av kryptovaluta?}
\subsection*{Rotårsak: Uklarhet i IT-reglementet angående kryptoutvinning}
Vi har vurdert uklarhet i IT-reglementet som hovedårsaken bak utvinning hos de ansatte og studenter, der de utnytter universitetets ressurser. Dette gjør vi fordi IT-reglementet ikke nevner utvinning av kryptovaluta eksplisitt og fordi kryptovaluta har vært en trend i media den siste tiden. Siden kryptoutvinning ikke er ulovlig og har blitt betegnet som den nye måten å bli rik på, tenker nok flere ikke over at personlig vinning ikke er lov i henhold til IT-reglement. Her er det informasjonskampanjen kommer inn. Den vil gjøre at folk blir oppmerksom på hva de gjør og hvilke represalier som kan forekomme. Det er knyttet to svakhet til denne løsningen for hvorfor NTNUs ressurser blir misbrukt. Disse er: informasjonskampanje vil nødvendigvis ikke endre oppførselen til de som er klar over regelbruddet, og endringen i IT-reglementet og informasjonskampanjen vil kunne hjelpe til å stoppe den frivillige utvinningen av kryptovaluta.

%\subsection*{Rotårsak: Eksterne aktører utvinner krypto med universitetets ressurser}
%En vanlig måte for eksterne aktører å utvinne på er å bruke PCene til intetanende som et botnet \cite{Botnet}. Her blir datamaskinene infisert av skadevare som får dem til å jobbe for den eksterne aktøren. Siden dette er en utbredt måte å angripe på, anser vi det som et godt tiltak å blokkere DNS-adressene til som blir mest brukt. En annen måte som ser ut til å bli mer vanlig er såkalt ``cryptojacking'', der kryptoutvinning blir gjort av javascriptkode på nettsider brukeren besøker \cite{12577042320171101}. 

\subsection*{Rotårsak: Seksjon for Digital Sikkerhet har ikke nok ressurser til å prioritere håndtering av kryptoutvinning}
En vanlig måte for eksterne aktører å utvinne med datamaskiner i et botnett \cite{Botnet}. Botnett er datamaskiner infisert av skadevare som lar den eksterne aktøren utnytte maskinene til deres formål. Siden dette er en utbredt måte å angripe på, anser vi det som et godt tiltak å blokkere DNS-adressene til som blir mest brukt, men dette tar tid og er ressurskrevende.
%En annen måte som ser ut til å bli mer vanlig er såkalt ``cryptojacking'', der kryptoutvinning blir gjort av javascriptkode på nettsider brukeren besøker \cite{12577042320171101}. 



\section{Hvor godt fungerer rotårsaksanalyse innen informasjonssikkerhet?}
Det er fortsatt få studier som prøver å sette lys på nytteverdien ved bruk av rotårsaksanalyse innen informasjonssikkerhet. I løpet av dette prosjektet har vi gjort oss en erfaring basert på verktøybruken, men for å fremskaffe empirisk grunnlag for å mene hvor godt det fungerer, må det gjerne gjøres mer enn én gang. 

På den ene siden vet vi ikke helt hvor godt rotårsaksanalyse har fungert før tiltakene er implementert, og det er kontrollert at symptomene minker eller forsvinner helt. På den andre siden har et tidligere bachelorprosjekt kommet frem til at nytteverdien er stor. De stilte blant annet spørsmål om hvor godt det fungerer på case med lite tid og ressurser, samt mye tid og ressurser \cite{RCARapport}. Det ble i begge sammenhenger konkludert med at det ga gode resultater. Vi mener at nytteverdien kommer an på hvor god tilgang en har på relevant informasjon. I noen av casene fikk vi et godt datagrunnlag som ga oss gode muligheter til å avdekke rotårsakene, mens spesielt det tredje caset slet vi med lite datagrunnlag. Vi anser dette å være kritisk for hvor god nytteverdien er. I samme bacheloroppgave som nevnt over ble det også erfart at ved hjelp av rotårsaksanalyse er det mulig å oppdage problemer som ikke belyses av andre verktøy \cite{RCARapport}.

Vi har erfart at den strukturerte tilnærmingen til casene som rotårsaksanalyse gir oss er nyttig for å forstå problemet i detalj, og foreslå tiltak som hjelper på det faktiske problemet. Om det fungerer godt er en ting, men om det lønner seg å bruke rotårsaksanalyse innen informasjonssikkerhet er en annen, og diskuteres nærmere i neste seksjon. 


\section{Lønner det seg å benytte rotårsaksanalyse i informasjonssikkerhetssammenheng?}
Vi har i alle tre casene dokumentert tidsbruken i de ulike fasene av metodikken. Det er tidsbruken sammenlignet med resultatene vi tar utgangspunkt i når det drøftes om det lønner seg å benytte rotårsaksanalyse innen informasjonssikkerhet. Tidsbruken i henholdsvis case 1, 2 og 3, kan sees i tabell \ref{tab:tidsbruk_case1}, \ref{tab:tidsbruk_case2} og \ref{tab:tidsbruk_case3}. Det er relativt lik tidsbruk på de to første casene, mens den tredje casen ble gjort raskt. Grunnen var en blanding av at det tredje caset ikke var så omfattende, og at det var liten tid til disposisjon. Likevel fikk vi resultater, selv om disse ikke var like detaljerte som i foregående caser. I det første caset var det vanskelig å nå noen endelig løsning siden rotårsaken til problemet er større enn det universitetet kan løse. Det ble likevel utredet noen gjennomførbare tiltak som kan hjelpe med å senke risikoen, men ikke fjerne rotårsaken. Vi følte derimot at dette kunne like godt blitt gjort med en risiko- og sårbarhetsanalyse. Det negative med det er at man kan gå glipp av en dypere forståelse av årsaken til fildelingen, og kanskje hatt mindre fokus på å fjerne problemet. Likevel fikk vi et dypere innblikk i problemet, som viste seg å være noe mer nyansert enn det oppdragsgiver forventet. I caset om kompromitterte kontoer fungerte det derimot ganske bra. Tidsbruken speilet ganske godt de resultatene vi fikk. Til tross for liten tid og svakt datagrunnlag på case 3, kom vi frem til noen nyttige resultater. Dette tror vi er fordi caset var lite og håndterbart. Dersom det hadde vært et større og mer komplekst case hadde det gått mye dårligere. En tidligere bacheloroppgave undersøkte hvordan rotårsaksanalyse fungerer på caser med både god og dårlig tid \cite{RCARapport}. Denne kom frem til at lang tidsbruk på analysen faktisk fører til andre resultater enn en normal risikovurdering av problemet. De konkluderte derfor med at resultatene rettferdiggjorde tidsbruken. Kort tidsbruk viste seg også å lønne seg. De kom frem til forslag til tiltak som ikke hadde blitt vurdert eller implementert tidligere. 

\section{Hvilke metoder og verktøy som ofte brukes i rotårsaksanalyse, fungerer best innen informasjonssikkerhet?}
På bakgrunn av erfaringer vi har tilegnet oss gjennom bruk av rotårsaksanalyse i de tre casene, har vi laget retningslinjer for bruk av rotårsaksanalyse innen informasjonssikkerhet. I tillegg til egen erfaring ble også den tidligere bacheloroppgaven \cite{RCARapport} tatt i betraktning, men ble vektet svakere. Retningslinjene finnes i kapittel \ref{kap:retningslinjer-RCA}. 


\section{Kritikk av oppgaven}
Til tross for engasjerende casestudier var case 1 en hard nøtt å knekke. Rotårsakene vi kom frem til var vanskelige å fjerne helt, som er poenget med rotårsaksanalyse. Dette var muligens på grunn av type case. 

Når det kommer til case 3 burde datagrunnlaget vært bedre. Dette kommer av flere ting, blant annet at caset var noe høytflytende og at det var liten tid. Likevel fikk det konsekvenser for resten av oppgaven. 

Et annet stort problem for oppgaven var at vi hadde et svakt teorigrunnlag. Med bare én bok som beskrev metodikken og én tidligere bacheloroppgave, hadde vi ikke mye å ta utgangspunkt i. Dette kommer litt av at det er lite materiell som beskriver rotårsaksanalyse med utgangspunkt i informasjonssikkerhet. 
