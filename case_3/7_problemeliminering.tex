\chapter{Rotårsakseliminering}
Arbeidet i denne fasen går ut på å finne tiltak som vil fjerne rotårsaken.

\section{Systematisk Innovativ Tenkning (SIT)}
Systematic Inventive Thinking inneholder fem hovedprinsipper:

\begin{enumerate}
    \item \textbf{Attributtavhengighet} Endre på en essensiell variabel.
    \item \textbf{Komponentkontroll} Ser på hvordan et produkt er tilknyttet omgivelsene sine.
    \item \textbf{Erstatning} Bytte ut en del av produktet med noe i omgivelsene til produktet.
    \item \textbf{Forkastning} Fjerne en del av produktet for å bedre det.
    \item \textbf{Oppdeling} Prøver å splitte et produkts attributter i to.
\end{enumerate}

\subsection{Ønsket utbytte}
Ved å bruke SIT-metoden ønsker vi å få kreative idéer på hvordan vi kan finne en løsning til utvinning av kryptovaluta ved NTNU. 

\subsection{Gjennomføring}
SIT burde helst gjennomføres av 10-12 personer, fra en rekke forskjellige fagområder, men siden vi ikke hadde så mange tok vi bare utgangspunkt i prosjektgruppen. 
\subsubsection{Komponenter} 
Her blir alle komponenter som omhandler problemet listet.
\begin{itemize}
    \item IT-reglement
    \item Annonseblokker
    \item Internett
    \item SOCen
    \item Brannmur
    \item Servere og datamaskiner
    \item Datalabber
    \item HPC-cluster
    \item Bring your own device (BYOD)
    \item Strøm
\end{itemize}

Når komponentene er gjort rede for, vil de fem SIT-prinsippene brukes sekvensielt på komponentene for å utvikle løsninger på problemene. Resultatene fremheves i neste seksjon. 


\subsection{Resultater}
Ikke alle SIT-prinsipper finner løsninger som er gjennomførbare for alle komponenter. I disse tilfellene vil det stå: ``Ikke gjennomførbart'

\paragraph{IT-reglement}
\begin{itemize}
    \item \textbf{Attributtavhengighet} Legge til et større fokus på utvinning av krypto.
    \item \textbf{Komponentkontroll} Gjennomføre en informasjonskampanje for å sette fokus på hva som er misbruk.
    \item \textbf{Erstatning} Ikke gjennomførbart.
    \item \textbf{Forkastning} Ikke gjennomførbart.
    \item \textbf{Oppdeling} Ikke gjennomførbart. 
\end{itemize}

\paragraph{Annonseblokker}
\begin{itemize}
    \item \textbf{Attributtavhengighet} Legge til blokkering av mining 
    \item \textbf{Komponentkontroll} Passe på at alle ansatte har annonseblokker installert som også stopper utvinning av krypto.
    \item \textbf{Erstatning} Ikke gjennomførtbart
    \item \textbf{Forkastning} Ikke gjennomførtbart
    \item \textbf{Oppdeling} Ikke gjennomførtbart.
\end{itemize}

\paragraph{Internett}
\begin{itemize}
    \item \textbf{Attributtavhengighet} Blokkere kryptoutvinning
    \item \textbf{Komponentkontroll} Automatisere at alle datamaskiner som utvinner krypto blir kastet av nettet.
    \item \textbf{Erstatning} Ikke gjennomførtbart.
    \item \textbf{Forkastning} Ikke gjennomførtbart.
    \item \textbf{Oppdeling} Ikke gjennomførtbart.
\end{itemize}


\paragraph{SOC}
\begin{itemize}
    \item \textbf{Attributtavhengighet} Øke antall ansatte.
    \item \textbf{Komponentkontroll} Økt prioritet til krypto.
    \item \textbf{Erstatning} Ikke gjennomførtbart.
    \item \textbf{Forkastning} Ikke gjennomførtbart.
    \item \textbf{Oppdeling} Gi forslag til implementering av tiltak til bachelorgrupper.
\end{itemize}

\paragraph{Servere og datamaskin}
\begin{itemize}
    \item \textbf{Attributtavhengighet} Strengere adgangskontroll.
    \item \textbf{Komponentkontroll} Ikke gjennomførtbart.
    \item \textbf{Erstatning} Ikke gjennomførtbart.
    \item \textbf{Forkastning} Ikke gjennomførtbart.
    \item \textbf{Oppdeling} Ikke gjennomførtbart.
\end{itemize}

\paragraph{Datalabber}
\begin{itemize}
    \item \textbf{Attributtavhengighet} Strengere adgangskontroll.
    \item \textbf{Komponentkontroll} Logging.
    \item \textbf{Erstatning} Svakere maskinvare på labbene.
    \item \textbf{Forkastning} Slutte å tilby labber, som kan brukes i sammenheng med krypto utvinning.
    \item \textbf{Oppdeling} Ikke gjennomførbart.
\end{itemize}

\paragraph{HPC-clustere}
\begin{itemize}
    \item \textbf{Attributtavhengighet} Øke tilgangskontrollen ytterligere.
    \item \textbf{Komponentkontroll} Ikke gjennomførbart.
    \item \textbf{Erstatning} Ikke gjennomførbart.
    \item \textbf{Forkastning} Ikke gjennomførbart.
    \item \textbf{Oppdeling} Ikke gjennomførbart.
\end{itemize}


\paragraph{Bring your own device (BYOD)}
\begin{itemize}
    \item \textbf{Attributtavhengighet} Ikke gjennomførbart.
    \item \textbf{Komponentkontroll} Kaste datamaskiner som ikke tilhører NTNU-personell av nettet slik at de manuelt må koble seg på igjen.
    \item \textbf{Erstatning} Ikke gjennomførbart.
    \item \textbf{Forkastning} Ikke gjennomførbart.
    \item \textbf{Oppdeling} Ikke gjennomførbart.
\end{itemize}

\paragraph{Strøm}
\begin{itemize}
    \item \textbf{Attributtavhengighet} Ikke gjennomførtbart
    \item \textbf{Komponentkontroll} Strømkvotering, overstiges kvoten må vedkommende betale for strømmen
    \item \textbf{Erstatning} Ikke gjennomførtbart
    \item \textbf{Forkastning} Ikke gjennomførtbart
    \item \textbf{Oppdeling} Ikke gjennomførtbart
\end{itemize}

Vi sorterer og beskriver de mest relevante idéer til videre utdyping:

\begin{description}
\item[Gjennomføre en informasjonskampanje om kommersielt misbruk av NTNU sin infrastruktur] Kampanjen skal få frem at det å bruke NTNU sine ressurser til kommersiell virksomhet bryter IT-reglementet, og at strøm inngår i NTNU sine ressurser.
\item[Legge til et større fokus på utvinning av krypto i IT-reglementet] Gi IT-reglement et større fokus på kryptoutvinning og klarere retningslinjer på hva som ikke er greit å gjøre.
\item[Legge til annonseblokker som stopper utvinning] Aktivere blokkering av utvinning-protokoll på annonseblokkere og passe på at alle har en annonseblokkeringstjeneste installert.
\item[Blokkere kryptoutvinning] Med dette mener vi å gjøre et eller flere tiltak som å blokkere DNS-forespørsel som omhandler kryptoutvinning.  
\item[Øke antall personell i SOC] SOC har mange oppgaver som er mer kritiske enn kryptoutvinning. Derfor foreslår vi å ansette flere, kanskje i kombinasjon med bacheloroppgaver.
\item[Strengere adgangskontroll] Begrenser tilgang til datalabber. 
\item[Logging] Øke bruk av logging  i datalabbene. 
\item[Kaste datamaskiner som ikke er kritisk infrastruktur av nettet] Ved midnatt blir alle datamaskiner eller servere som ikke er kritisk infrastruktur koblet av nettet og må manuelt koble seg på nettet igjen.
\end{description}

\subsection{Tiltaksplan}
Etter å ha brukt de fem SIT-prinsippene på hver komponent, og filtrert de, sitter vi igjen med et par idéer. I denne delen fremhever vi idéer i en tiltaksplan som vi anbefaler å implementere. 
Under beskrives de ulike tiltakene:

\begin{description}
    \item[Gjennomføre en informasjonskampanje om kommersielt misbruk av NTNU sin infrastruktur] Utvinning av kryptovaluta er en ny ting, hvor mange ikke er klar over hvordan universitetet sitt regelverk håndterer temaet. Vår anbefaling er å ha en kampanje der universitet informerer om hva som regnes som NTNU sine ressuser og hvordan disse ikke skal brukes til kommersiell virksomhet.
    \item[Legge til et større fokus på utvinning av krypto i IT-reglementet] Endre IT-reglementet slik at det blir tydelig at kryptoutvinning ikke er lovlig bruk av NTNU sine ressurser. 
    \item[Blokkere kryptoutvinning] Dette tiltaket går ut på å blokkere DNS-forespørsler tilknyttet kryptoutvinning. Slik at PCer som blir brukt i utvinning ikke kan hente nye oppgaver å løse. De velkjente DNSene blokkeres. Videre kan loggen bli benyttet for å legge nye domener inn i en svarteliste, eller justere de gamle DNSene.
    \item[Øke antall personell i SOC] SOCen kan ikke prioritere å stoppe utvinning av kryptovaluta. Derfor anbefaler vi å enten øke mengden personell i SOCen, eller gi utvikling av implementasjonsstrategi som en bachelor oppgave.
\end{description}


\subsection{Konklusjon av verktøyet}
 Verktøyet fungerte greit, der sorteringen og tiltaksplan fungerte godt. Det å finne komponentene og gjøre de fem SIT-prinsippene er der SIT hvertfall i informasjonsikkerhetssammenheng  ser ut til å fungere dårlig. Vi prøvde SIT i alle casene for å se hvor bra det gikk, og har kommet fram til at selv om sorteringen og tiltaksplanen fungerer bra vil vi heller anbefale seks tenkehatter for rotårsakselimineringen. Da det å bruke SIT-prinsippene i informasjonsikkerhet er tungvint og ser ut til å passe mer for et fysisk produkt. 