\chapter{Diskusjon og konklusjon}
\section{Diskusjon}


\subsection{Hva er rotårsaken til at NTNU sin infrastruktur blir misbrukt til utvinning av kryptovaluta?}
Caset virket i første omgang som temmelig rett frem, der man så at ønske om å bli fort rik var rotårsaken. Da oppgaven ble studert nøyere oppdagdes det at den er langt mer komplisert. Det er ikke en klar rotårsak, men en kombinasjon av flere årsaker, deriblant lønnsomhet.

\subsection*{Rotårsak: Uklarhet i IT-reglementet angående kryptoutvinning}
Vi har vurdert uklarhet i IT-reglementet som hovedårsaken bak utvinning hos de ansatte og studenter, der de utnytter universitetets ressurser. Dette gjør vi fordi IT-reglementet ikke nevner utvinning av kryptovaluta eksplisitt og fordi kryptovaluta har vært en trend i media den siste tiden. Siden kryptoutvinning ikke er ulovlig og har blitt betegnet som den nye måten å bli rik på, tenker nok flere ikke over at personlig vinning ikke er lov i henhold til IT-reglement. Her er det informasjonskampanjen kommer inn. Den vil gjøre at folk blir oppmerksom på hva de gjør og hvilke represalier som kan forekomme. Det er knyttet to svakhet til denne løsningen for hvorfor NTNUs ressurser blir misbrukt. Disse er: informasjonskampanje vil nødvendigvis ikke endre oppførselen til de som er klar over regelbruddet, og endringen i IT-reglementet og informasjonskampanjen vil kunne hjelpe til å stoppe den frivillige utvinningen av kryptovaluta.

%\subsection*{Rotårsak: Eksterne aktører utvinner krypto med universitetets ressurser}
%En vanlig måte for eksterne aktører å utvinne på er å bruke PCene til intetanende som et botnet \cite{Botnet}. Her blir datamaskinene infisert av skadevare som får dem til å jobbe for den eksterne aktøren. Siden dette er en utbredt måte å angripe på, anser vi det som et godt tiltak å blokkere DNS-adressene til som blir mest brukt. En annen måte som ser ut til å bli mer vanlig er såkalt ``cryptojacking'', der kryptoutvinning blir gjort av javascriptkode på nettsider brukeren besøker \cite{12577042320171101}. 

\subsection*{Rotårsak: Seksjon for Digital Sikkerhet har ikke nok ressurser til å prioritere håndtering av kryptoutvinning}
En vanlig måte for eksterne aktører å utvinne med datamaskiner i et botnett \cite{Botnet}. Botnett er datamaskiner infisert av skadevare som lar den eksterne aktøren utnytte maskinene til deres formål. Siden dette er en utbredt måte å angripe på, anser vi det som et godt tiltak å blokkere DNS-adressene til som blir mest brukt, men dette tar tid og er ressurskrevende.
%En annen måte som ser ut til å bli mer vanlig er såkalt ``cryptojacking'', der kryptoutvinning blir gjort av javascriptkode på nettsider brukeren besøker \cite{12577042320171101}. 

\subsection{Tidsbruk}
\label{tidsbruk_case_3}
Tabell \ref{tab:tidsbruk_case3} under viser tidsbruken i enkeltfasene i dette caset. Dette inkluderer tid brukt til å dokumentere alt som har med de enkelte fasene å gjøre. 

% Table generated by Excel2LaTeX from sheet 'Ark1'
\begin{table}[H]
  \centering
  \caption{Tidsbruk de ulike fasene i case 3}
    \begin{tabular}{|lr|l|}
    \hline
    \multicolumn{3}{|c|}{\cellcolor{yellow}\textbf{Case 3}} \\
    \hline
    \multicolumn{1}{|l|}{\cellcolor{apricot}\textbf{Fase}} & \multicolumn{1}{l|}{\cellcolor{apricot}\textbf{Verktøy brukt}} & \cellcolor{apricot}\textbf{Timer totalt} \\
    \hline
    \multicolumn{1}{|l|}{Problemforståelse} & \multicolumn{1}{l|}{Ytelsesmatrise} & 16-20t \\
    \hline
    \multicolumn{1}{|l|}{Idémyldring} & \multicolumn{1}{l|}{Idémyldring og NGT} & 16-20t \\
    \hline
    \multicolumn{1}{|l|}{Datainnsamling} & \multicolumn{1}{l|}{Intervju} & 20-30t \\
    \hline
    \multicolumn{1}{|l|}{Datanalyse} & \multicolumn{1}{l|}{Affinitetsdiagram} & 15-20t \\
    \hline
    \multicolumn{1}{|l|}{Rotårsaksidentifisering} & \multicolumn{1}{l|}{5 whys og feil-tre diagram} & 25-30t \\
    \hline
    \multicolumn{1}{|l|}{Rotårsakseliminering} & \multicolumn{1}{l|}{SIT} & 15-20t \\
    \hline
    \multicolumn{1}{|l|}{Løsningsimplementering} & \multicolumn{1}{l|}{Kraftfeltsdiagram} & 10-15t \\
    \hline
    \multicolumn{2}{|l|}{\textbf{Sum}} & \textbf{117-155t} \\
    \hline
    \end{tabular}%
  \label{tab:tidsbruk_case3}%
\end{table}%


\subsection{Videre arbeid}
Siden datagrunnlaget vårt var noe snevert kan det våre interessant å hente inn mer data, for å se om det er andre rotårsaker vi ikke fant. Rettningslinjer for bruk utvinning av kryptovaluta burde i teorien få alle ansatte eller studenter til å slutte å utvinne. Men de er mennesker så det kan være greit å se etter flere gode tekniske løsninger som ikke lar utvinning være en mulighet.



%Siden vi tok et sample bare fra de som tidligere hadde blitt kompromittert kan det være interessant å undersøke hele NTNU når det kommer til passordvaner, e-post, kjennskap til retningslinjer osv. Deretter kan resultatene sammenlignes og se om det er noen forskjeller som bør tas i betraktning. Annet videre arbeid kan være å undersøke keylogging som en mulig årsak til kompromitterte kontoer. Vi gikk ikke så mye inn på det i denne rapporten, men det kan være interessant å se på i forlengelse av infeksjon fra ondsinnet programvare. 



%\subsubsection{Hvor godt fungerer rotårsaksanalyse i et case som %omhandler misbruk av IT-infrastruktur med begrenset tid?}
%Rotårsaksanalyse fungerer ikke spesielt godt i et case der det er %sterke begrensninger på tid. Det vil si vi følte ikke at analysen ga %noe mer enn de løsningene som allerede var kjent. Siden %datainnsamling og dataanalyse er de mest tidkrevende elementete og %noen av de viktigste elementene i en rotårsaksanalyse.