\chapter{Diskusjon og konklusjon}
\subsection{Diskusjon}


\subsubsection{Hva er rotårsaken til at NTNU sin infrastruktur blir misbrukt til utvinning av kryptovaluta?}
Caset virket i første omgang som temmelig rett frem, der man så at ønske om å bli fort rik var rotårsaken. Da oppgaven ble studert nøyere oppdagdes det at den er langt mer komplisert. Det er ikke en klar rotårsak, men en kombinasjon av flere årsaker, deriblant lønnsomhet.

\subsection{Rotårsak: Uklarhet i IT-reglementet angående kryptoutvinning}
Vi har vurdert uklarhet i IT-reglementet som hovedårsaken bak utvinning hos de ansatte og studenter, der de utnytter universitetets ressurser. Dette gjør vi fordi IT-reglementet ikke nevner utvinning av kryptovaluta eksplisitt og fordi kryptovaluta har vært en trend i media den siste tiden. Siden kryptoutvinning ikke er ulovlig og har blitt betegnet som den nye måten å bli rik på, tenker nok flere ikke over at personlig vinning ikke er lov i henhold til IT-reglement. Her er det informasjonskampanjen kommer inn. Den vil gjøre at folk blir oppmerksom på hva de gjør og hvilke represalier som kan forekomme. Det er knyttet to svakhet til denne løsningen for hvorfor NTNUs ressurser blir misbrukt. Disse er: informasjonskampanje vil nødvendigvis ikke endre oppførselen til de som er klar over regelbruddet, og endringen i IT-reglementet og informasjonskampanjen vil kunne hjelpe til å stoppe den frivillige utvinningen av kryptovaluta.

%\subsection{Rotårsak: Eksterne aktører utvinner krypto med universitetets ressurser}
%En vanlig måte for eksterne aktører å utvinne på er å bruke PCene til intetanende som et botnet \cite{Botnet}. Her blir datamaskinene infisert av skadevare som får dem til å jobbe for den eksterne aktøren. Siden dette er en utbredt måte å angripe på, anser vi det som et godt tiltak å blokkere DNS-adressene til som blir mest brukt. En annen måte som ser ut til å bli mer vanlig er såkalt ``cryptojacking'', der kryptoutvinning blir gjort av javascriptkode på nettsider brukeren besøker \cite{12577042320171101}. 

\subsection{Rotårsak: Universitetet har ikke ressurser til å prioritere håndtering av kryptoutvinning}
Universitetet har 

\subsection{Tidsbruk}
\label{tidsbruk_case_3}


\subsection{Videre arbeid}




%\subsubsection{Hvor godt fungerer rotårsaksanalyse i et case som %omhandler misbruk av IT-infrastruktur med begrenset tid?}
%Rotårsaksanalyse fungerer ikke spesielt godt i et case der det er %sterke begrensninger på tid. Det vil si vi følte ikke at analysen ga %noe mer enn de løsningene som allerede var kjent. Siden %datainnsamling og dataanalyse er de mest tidkrevende elementete og %noen av de viktigste elementene i en rotårsaksanalyse.