\chapter{Idémyldring}
I dette steget i prosessen er målet å generere en liste over det vi tror kan være mulige årsaker til problemet. Det er en del forskjellige verktøy en kan bruke for å oppnå dette, men vi har valgt å benytte Idémyldring på basis av RCA boken \cite{RCA} sin fremgangsmåte for valg av verktøy, og på bakgrunn av vår tidligere kunnskap om hvordan brukere vanligvis kompromitteres. 

\section{Idémyldring}
I rotårsaksanalyse finnes det to ulike måter å gjennomføre idémyldring på: strukturert- og ustrukturert idémyldring. I den strukturerte versjonen får hver deltaker sin tur til å komme med en idé, og dette sikrer at alle får delta like mye. På den ustrukturerte måten kan alle komme med idéer etterhvert som de kommer på dem, og fungerer mye mer spontant enn den strukturelle. Det er spesielt viktig å ikke omformulere eller diskutere forslagene etterhvert som de kommer, dette skal gjøres etter idémyldringsøkten er over.

\subsection{Ønsket utbytte}
Ønsket utbytte ved å bruke idémyldring er å få en forståelse av hva som kan være rotårsaken til at NTNU sine ressurser blir misbrukt til utvinning av kryptovaluta. Vi ønsker også å skape et godt grunnlag av idéer som kan brukes i kvalitativ datainnsamling i neste del av oppgaven.

\subsection{Gjennomføring}
Vi startet idémyldringen med å finne ut hva som burde fokuseres på av hvordan og hvorfor skolen sine ressurser blir missbrukt til utvinning av kryptovaluta. Vi kom frem til at begge deler var like viktige, og at vi burde ha to idémyldringer for å komme med flest mulige idéer som ville kunne hjelpe oss i hva som burde spørres om i datainnsamlingen. De to ``basene'' til Idémyldring vi brukte i idémyldringsprossesen var:  Hvordan/Hvorfor " blir NTNU sine ressurser missbrukt i utvinning av kryptovaluta?". Idémyldring prosessen var litt annerledes enn de vi hadde i de andre oppgavene, der vi denne gangen hadde et mer ``Brainwriting'' oppsett, der vi alle skrev inn de idéene vi hadde inn i et dokument.

\subsection{Resultater}
Etter at økten var ferdig ble det gjort en vurdering av hvilke momenter som hørte sammen eller hadde likhetstrekk og disse ble gruppert under for eksempel utilstrekkelig inngangskontroll, se figur \ref{fig:idemyldring} under.


%Dette kan være ved hjelp av script som miner mens en intern er på nettsiden eller annen skadevare.


\begin{figure}[H]
    \centering
    \includegraphics[scale=0.5]{case_3/bilder/idemyldring-hvordan.pdf}
    \caption[Idémyldring]{Resultater og gruppering av idémyldringen}
     \label{fig:idemyldring}
\end{figure}

\begin{figure}[H]
    \centering
    \includegraphics[scale=0.5]{case_3/bilder/idemyldring-hvorfor.pdf}
    \caption[Idémyldring]{Resultater og gruppering av idémyldringen}
    \label{fig:idemyldring_2}
\end{figure}

\subsection{Konklusjon av verktøyet}
Verktøyet hjelper til å få økt forståelse på hva som er årsaker til problemstilingen eller i denne samenheng to problemstilinger. Finnes det en klar problemstiling er det lett å komme med ideer på hva som er årsakene. Det gir et god overordnet bile av situasjonen, men siden vi har lite tid på dette caset kommer det veldig mange idéer, der noen trolig blir unødvendige å følge.



\section{Nominell gruppeteknikk}
Nominell gruppeteknikk også kalt NGT er noe boken om RCA \cite{RCA} anbefaler å bruke om det er idéer som må prioriteres. 

\subsection{Ønsket utbytte}
Siden vi i det siste oppdraget hadde litt dårligere tid enn på de tidligere har vi valgt å kutte ned på Idéer vi føler at burde følges ved hjelp av NGT.
\subsection{Gjennomføring}
NGTen ble utført med at vi alle satt oss ned sammen og hadde 15 poeng hver å gi til forskjellige idéer. Idéene ble laget utifra forarbeidet som var blitt gjort i idémyldrings prosessen, de idéene som lignet på hverandre ble slått sammen og noen ble litt omformulert. idéene skulle få pengene 1,2,3,4 eller 5 og alle disse 15 poengene skulle gis ut. Hver person ga ut sine 15 poeng for hva de trodde ville være det viktigste å fokusere på videre i analysen, og disse dataene kan sees i tabellen nedenfor.

\subsection{Resultater}
Etter at hvert gruppemedlem hadde gitt ut sine 15 poeng satt vi igjen med 4 idéer som stakk seg ut med hvor mange poeng de fikk, disse er i fet skrift i tabellen. Disse fire årsakene er de vi kommer til å fokusere på i dette caset. De fire fokuspunktene kan deles etter de to promblemstillingene ``hvordan og hvorfor'' :

De to årsakene knyttet til ``hvorfor problemstillingen'' går ut på at det ikke er ulovlig med kryptoutvinning i henhold til gjeldende regelverk og faller derfor i en gråsone, der du kan ikke få noen represalier for å holde på med kryptoutvinning, men konsekvensene kommer av at man misbruker utstyr til egen vinning.

Neste årsaker går på ``hvordan'' forsksjellige aktører og hvilke aktører som bruker skolen sine ressurser til utvinning av kryptovaluta. Internt missbruk av labutstyr går ut på at studenter eller ansatte har tilgang til diverse labber og pcer som kan brukes til utvinning av kryptovaluta. Både ondsinnet programvare som utvinner kryptovaluta og webintegrerte utvinnere gjøres av eksterne aktører, disse pengene som tjenes her går som regel tilbake til diverse kriminelle nettverk. Det med utvinning av kryptovaluta har blitt mye mer utbredt som et alternativ til ransomware, der ransomware har blitt mindre lønnsomt i den siste tiden.








\begin{table} [H]
    \begin{tabular}{ | m{2em} | m{30em} | m{3em} | }
        \hline
            \cellcolor{yellow}  & \cellcolor{yellow} Årsak & \cellcolor{yellow} Poeng \\
        \hline
           \textbf{A}&\textbf{Internt misbruk av labutstyr} & 8  \\
        \hline
          B & Bruteforce av servere & 0 \\
        \hline
          \textbf{C} & \textbf{Ondsinnet programvare som miner} &  16 \\
        \hline
         \textbf{D} & \textbf{Webintegret miner} & 14 \\
        \hline
          E & Utilstrekkelig tilgangskontroll på supermaskiner & 0 \\
        \hline
          F & Enkel tilgang til datamaskiner på campus & 0 \\
        \hline
         G & Liten sannsynlighet for å bli tatt & 2 \\
        \hline
         \textbf{H} & \textbf{Ingen regler om utvinning av kryptovaluta i IT reglementet} & 11 \\
        \hline
         I & Slipper å betale for energi & 0 \\
        \hline
         J & Bra utstyr de ikke betaler for & 5 \\
        \hline
         K & Store svingninger, mulighet for “get rich quick" & 0 \\
        \hline
         L & Kryptovaluta har hatt en stor økning i verdi det siste året & 0 \\
        \hline
         M & Utvinningen av kryptovaluta krever mer og mer datakraft & 4 \\
        \hline
    \end{tabular}
    \caption{Oversikt over prioritering av idéer ved hjelp av NGT}
    \label{NGT}
\end{table}



\subsection{Konklusjon av verktøyet}
Verktøyet hjelper til med å bestemme hvilke idéer fra tidligere idémyldring det er viktig å følge videre i prosessen. Noe som kan være negativt med denne metoden, er at alle har like mye de skulle ha sagt om en sak, selvom de kanskje ikke har like mye kunnskap om de forskjellige momentene. Det positive er at vi får prioritert årsakene i henhold til viktighet. 