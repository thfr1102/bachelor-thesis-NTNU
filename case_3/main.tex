%% This document gives an example on how to use the ntnubachelorthesis
%% LaTeX document class.
%% Use oneside for PDF delivery and twoside for printing in a book style
%% use language english, norsk, nynorsk and one of the following shortenings
%%  ``BSP'' Bachelor i Spillprogrammering,\\
%%  ``BRD'' Bachelor i drift av nettverk og datasystemer,\\
%%  ``BIS'' Bachelor i Informasjonssikkerhet,\\
%%  ``BPU'' Bachelor i Programvareutvikling, \\
%%  ``BIND'' Bachelor i Ingeniorfad - data, \\
%%  ``BADR'' Bachelor i drift av datasystemer, \\
%%  ``BIT'' Bachelor i informatikk, \\
%%  ``BABED'' Bachelor i IT-støttet bedriftsutvikling.
%%   for example \documentclass[BIS,norsk,twoside]{ntnuthesis/ntnubachelorthesis}

\documentclass[BIS,norsk,oneside]{ntnuthesis/ntnubachelorthesis}
\thesistitle{Case 3: Misbruk av NTNU sin infrastruktur til utvinning av kryptovaluta}
\thesisshorttitle{Case 3: Misbruk av NTNU sin infrastruktur til utvinning av kryptovaluta}
\thesisauthor{Philip Nyblom, Fredrik Theien, Thomas Huse, Ole Martin Søgnen}
\thesisdate{16.05.2018}

\usepackage{csvsimple}
\usepackage{booktabs}
\usepackage{gnuplottex}
\usepackage[T1]{fontenc}
\usepackage[utf8]{inputenc}     % For utf8 encoded .tex files because...
\usepackage[norsk]{babel}       % For Norwegian labeling
\usepackage{graphicx}           % For inclusion of graphics
\PassOptionsToPackage{hyphens}{url}
\usepackage{url}
\usepackage{hyperref}    % For cross references in pdf
\usepackage{tabularx}
\usepackage[table]{xcolor}
\usepackage{colortbl}
\usepackage{placeins}
\usepackage{pdfpages}
\usepackage{enumitem}
\usepackage{multirow}
\usepackage{lscape}
\usepackage{bigstrut}
\usepackage{verbatim}
\usepackage{float}


\definecolor{darkgreen}{rgb}{0,0.5,0}
\definecolor{darkred}{rgb}{0.5,0.0,0}

\lstset{        basicstyle=\ttfamily,
                keywordstyle=\color{blue}\ttfamily,
                stringstyle=\color{darkred}\ttfamily,
                commentstyle=\color{darkgreen}\ttfamily,
}


%Typesetting of C++
\newcommand{\CPP}[0]{{C\nolinebreak[4]\hspace{-.1em}\raisebox{.1ex}{\small\bf +\hspace{-.1em}+\ }}}



%\newcommand{\comment}[1]{\textcolor{blue}{\emph{#1}}}  %% use of the colour and you can see how to use commands with parts \comment{so what}

%% The class files defines these two
%% \newcommand{\NTNU}{Norwegian University for Science and Technology} %

% you can create you one #define like structures using the \newcommand feature
% you can change behaviour using \renewcommand

\newcommand{\com}[1]{{\color{red}#1}} % supervisor comment
%\renewcommand{\com}[1]{} %remove starting % to remove supervisor comments
% This will appear in text \com{Lecuters comment} and be visible unless you uncomment
% the renewcommand line.

\newcommand{\todo}[1]{{\color{green}#1}} % items to do
%\renewcommand{\todo}[1]{} %remove starting % to remove items to do

\newcommand{\n}[1]{{\color{blue}#1}} % other comment
%\renewcommand{\n}[1]{} %remove starting % to remove notes

\newcommand{\dn}[1]{} % add the d to a note to say that you have finished with it.

\newcommand{\gj}{NTNU i Gj\o{}vik}


% Norwegian Characters,  needs the {} or to be separate from the next letters
% \o{}   \aa{}   \ae{}   so at the end of a word you can use \o  \aa   \ae
% \O{}   \AA{}   \AE{}   you can also just leave a space and latex will remove it
%    eg, NTNU i Gj\o vik  or NTNU i Gj\o{}vik

\graphicspath{ {bilder/} }

\begin{document}

\thesistitlepage

\tableofcontents
\listoffigures
\listoftables

%\chapter*{Kortfattet sammendrag}
I dette caset undersøkte vi rotårsaken til at NTNU sin infrastruktur blir misbrukt til utvinning av kryptovaluta. Utvinning av kryptovaluta har blitt en økende trend det siste året, der det har kommet frem flere historier i media. Dette gjør at flere vil prøve seg og de bruker da universitetet sin infrastruktur som ikke skal bli brukt til kommersiell virksomhet. Kriminelle vil heller ikke gå glipp av denne gyldne muligheten og har sine script de bruker. 

I dette prosjektet har vi benyttes oss av metodene og verktøyene som boken om rotårsaksanalyse \cite{RCA} tar for seg.

Vi hadde en hypotese om... Dette viste seg å være... Ut ifra intervjuet vi gjennomførte og analysen av dette, kom vi frem til at... ... var den største faktoren som gjorde at NTNU sin infrastruktur ble misbrukt til utvinning av kryptovaluta.
\chapter{Introduksjon}
%--------------------MÅ INN I HOVEDRAPPORT------------------------------------------
Rotårsaksanalyser er et lite brukt verktøy innen informasjonssikkerhet, men er av økende betydning. Vanlig tilnærming til informasjonssikkerhetsstyring er å utføre en risiko- og sårbarhetsanalyse (ROS-analyse) for så å gjennomføre tiltak som fører risikoene til et akseptabelt nivå. En annen hyppig brukt tilnærming er hendelseshåndtering der en planlegger hvordan det skal responderes på hendelser etter de er inntruffet. Rotårsaksanalyse skiller seg fra disse ved å gå i dybden på problemet, kartlegge hva slags rotårsaker som står bak, og innføre tiltak for å fjerne disse helt.
%%%%%%%%%%%%%%%%%%%%%%%%%%%%%%%%%%%%%%%%%%%%%%%%%%%%%%%%%%%%%%%%%%%%%%%%%%%%%%%%%%%%%

\section{Oppgavebeskrivelse}
Denne rapporten er en delrapport i en større oppgave om rotårsaksanalyse. Dette caset går inn på rotårsaken til misbruk av NTNU sine ressurser og infrastruktur til å utvinne kryptovaluta. De to siste årene har både verdien og antallet kryptovaluta økt drastisk. Det finnes per dags dato over 1500 forskjellige kryptovalutaer. Kryptovaluta blir "minet", eller utvinnet, ved bruk av regnekraft. Dette betyr at enhver datamaskin kan delta i utvinningen. Siden november 2017 har NTNU sett en økning i mining med 8000\% og får i dag flere varsler angående mining om dagen.  Etterhvert vil vanskelighetsgraden for å utvinne nye mynter øke. Når vanskelighetsgraden øker trenger en mer datakraft og større maskinrigger til å utvinne valutaene. 

NTNU forvalter stor regnekraft spredt på flere lokasjoner. NTNU har også hatt supermaskiner før, de har en nå og de får nå en ny supermaskin. Supermaskiner er store datamaskiner med enorm datakraft. Disse er spesielt attraktive for aktører å misbruke til å utvinne kryptovaluta. Siden trenden har økt de siste årene, og NTNU er i besittelse av mye regnekraft, må NTNU aktivt jobbe for å beskytte infrastrukturen. 

Siden dette er av økende trend, og Seksjon for Digital Sikkerhet har oppdaget at noe av universitetet sine ressurser har blitt brukt til utvinning av kryptovaluta, vil de undersøke måter å eliminere dette misbruket. 

Denne analysen går ut på å identifisere rotårsaken til misbruk av NTNU sine ressurser til utvinning av kryptovaluta, og foreslå tiltak for å eliminere den. I løpet av rapporten ønsker vi å svare på følgende forskningsspørsmål:

\begin{itemize}
    \item Hva er rotårsaken til at NTNU sin infrastruktur blir misbrukt til utvinning av kryptovaluta?
    \item Hvor godt fungerer rotårsaksanalyse i et case som omhandler misbruk av IT-infrastruktur?
\end{itemize}

I denne analysen definerer vi misbruk som all bruk av NTNU sin infrastruktur til kommersiell vinning. Personlig vinning er noe NTNU som en offentlig institusjon ikke har lov til å finansiere [KILDE!]. Når vi ser på misbruket så skiller vi mellom frivillig og ufrivillig misbruk. Med frivillig misbruk mener vi når noen med vilje misbruker universitet sine ressurser til personlig vinning. Med ufrivillig misbruk mener vi noen som utnytter interne brukere for å få tilgang til NTNU sin infrastruktur, for så å misbruke ressursene. Herunder regner vi alt fra mining der bruker besøker en nettside til personer som hacker seg inn i infrastrukturen.



I løpet av rapporten kommer vi også til å referere til ressursene og infrastrukturen som aktiva. 
\section{Metode}
Metodebruken i denne analysen er delt inn i syv steg som vist i \hyperref[fig:prosess]{Figur 1} under. I hvert steg av denne prosessen brukes det ulike verktøy for å hjelpe til med å forstå problemet, finne rotårsak, og tilslutt implementere tiltak for å eliminere årsakene. 
\begin{figure}[H]
    \centering
    \includegraphics[scale=0.6]{case_1/bilder/prosess.pdf}
    \label{fig:prosess}
    \caption[Rotårsaksanalyseprosessen]{Rotårsaksanalyseprosessen definert av Andersen og Fagerli}
\end{figure}
\chapter{Problemforståelse}
Denne fasen eksisterer for å passe på at en har forstått problemet i dypere detalj. Verktøyene som er relevante å bruke skal gi en bedre forståelse av blant annet omfanget og de ulike aspektene ved et problem. Jo bedre tilgang en har på informasjon, logger og dokumentasjon, jo bedre vil denne fasen kunne utføres. 


\section{Ytelsesmatrise}
Ytelsesmatrise er et diagram som tar i betraktning den nåværende ytelsen til en variabel. Dette kan bety flere ting og vurderes i ulike former, men i dette caset definerer vi ytelse som hvor godt de ulike variablene fungerer i dag. Det som gjør ytelsesmatrise så nyttig er at den også vurderer viktigheten, slik at en kan vurdere hvilken prioritering variablene som blir analysert har. I dette caset blir viktighet definert som hvor funksjonskritisk ressursene er for arbeidet som gjøres ved NTNU.

\subsection{Ønsket utbytte}
Ved bruk av dette verktøyet var det ønskelig å undersøke hvordan eksisterende kontroller fungerer i forhold til deres viktighet for NTNU. 

\subsection{Gjennomføring}
Prosessen startet ved å finne ut hvilke aspekter av problemet som skulle vurderes. Gruppen kom fram til å vurdere formene for kontroller som stopper eller reduserer sjansen for at trusselaktørene misbruker NTNU sin infrastruktur. 

Disse skulle vurderes basert på viktighet og ytelse. 

Matrisen ble tegnet opp i Excel der hver akse ble konstruert fra en til ni, og matrisen ble delt inn i fire områder:
\begin{description}
    \item[Uviktig:] Når både viktigheten og ytelsen er fra en til fem.
    \item[Overdrevent:] Når viktigheten er fra en til fem og ytelsen er fra fem til ni.
    \item[Ok:] Når både viktigheten og ytelsen er fra fem til ni.
    \item[Må forbedres:] Når viktigheten er fra fem til ni, mens ytelsen bare er fra en til fem.
\end{description}

\subsection{Resultater}
Variablene som ble vurdert til å kunne hjelpe til å redusere utvinning av kryptovaluta hos NTNU er som følger:
\begin{description}
    \item[Adgangskontroll på HPC klynger:] Klynger av tilkoblet maskinvare som sammen utgir svært høy ytelse. Er også kjent som superdatamaskiner. Disse er godt beskyttet med streng adgangskontroll og logging av alt som blir gjort.
    \item[Adgangskontroll på kritiske servere:] Adgangskontroll til servere som har en funksjonskritisk og/eller virksomhetskritisk rolle i driften av NTNU, som for eksempel DNS og DHCP servere. 
    \item[Adgangskontroll på andre servere:] Adgangskontroll til alle servere som ikke har en kritisk rolle i NTNU, men som fortsatt kan bli misbrukt. Inkluderer servere som står åpent ut mot nettet. 
    \item[Beskyttelse mot ufrivillig utvinning på datamaskiner:] Personer får tilgang til din datamaskin gjennom nettleseren og bruker den til å utvinne kryptovaluta. 
    \item[Policy på hva som er akseptabelt som BYOD:] Definerer hva som er lov å ta med av BYOD.
    \item[IT-reglement på krypto utvinning:] IT-reglementet spesifiserer per idag bare at det å bruke universitetets ressurser til kommersiell virksomhet ikke er greit. Det kan være vanskelig for folk å ta koblingen til at strøm er en slik ressurs og at utvinning av kryptovaluta kan regnes som kommersiell virksomhet. Sånn som IT-reglementet er idag er det heller ikke noen gode sanksjonsmuligheter mot folk som utvinner kryptovaluta.

\end{description}

Under ser vi hvor de ulike ressursene, eller aktiva, er plassert i henhold til de tidligere nevnte områdene.
\begin{figure}[H]
    \centering
    \includegraphics[scale=0.5]{case_3/bilder/ytelsesmatrise.pdf}
    \caption[Ytelsesmatrise]{Resultater fra ytelsesmatrisen}
    \label{fig:ytelsesmatrise}
\end{figure}

Det er mest kritisk å vurdere de variablene som havner under ``må forbedres''. Selv om noen havner under ``ok'', bør de fortsatt vurderes, men de vil ha lavere prioritet enn de nevnt over. Variabler som er uviktig eller overdrevent trenger man ikke vurdere nøye. Matrisen viser følgende prioriteringsgrunnlag til utbedring:

\begin{enumerate}
    \item IT-reglement på krypto utvinning
    \item Policy på hva som er greit som BYOD (IT-regement på BYOD som driver med utvinning)
    \item Beskyttelse mot ufrivillig utvinning på datamaskiner
    \item Adgangskontroll på andre servere
    \item Adgangskontroll på kritiske servere
    \item Adgangskontroll på HPC klynger
\end{enumerate}

\subsection{Konklusjon av verktøyet}
Verktøyet var nyttig for å finne frem til et prioritetsgrunnlag for de ulike enhetene som kan misbrukes eller misbruke NTNU sine ressurser. Vi ser at det er IT-reglementets mangel på spesifisering av kryptoutvinning som er det største problemet. 
\chapter{Idémyldring}
I dette steget av prosessen er målet å generere en liste over det vi tror kan være mulige årsaker til problemet. Det er en del forskjellige verktøy en kan bruke for å oppnå dette, men vi har valgt å benytte Idémyldring på basis av RCA boken \cite{RCA} sin fremgangsmåte for valg av verktøy.

\section{Idémyldring}
Med rotårsaksanalyse finnes det to ulike måter å gjennomføre idémyldring på: strukturert- og ustrukturert idémyldring. I den strukturerte versjonen får hver deltaker sin tur til å komme med en idé, og dette sikrer at alle får delta like mye. På den ustrukturerte måten kan alle komme med idéer etterhvert som de kommer på dem, og fungerer mye mer spontant enn den strukturerte. Det er spesielt viktig å ikke omformulere eller diskutere forslagene etterhvert som de kommer, dette skal gjøres etter idémyldringsøkten er over.

\subsection{Ønsket utbytte}
Ønsket utbytte ved å bruke idémyldring er å få en forståelse av hva som kan være rotårsaken til at NTNU sine ressurser blir misbrukt til utvinning av kryptovaluta. Vi ønsker også å skape et godt grunnlag av idéer som kan brukes i datainnsamling i neste del av oppgaven.

\subsection{Gjennomføring}
Vi startet idémyldringen med å finne ut hva som burde fokuseres på av hvordan og hvorfor skolen sine ressurser blir missbrukt til utvinning av kryptovaluta. Vi kom frem til at begge deler var like viktige, og at vi burde ha to idémyldringer for å komme med flest mulige idéer som ville kunne hjelpe oss i hva som burde spørres om i datainnsamlingen. De to ``basene'' til idémyldring vi brukte i idémyldringsprossesen var: hvordan eller hvorfor ``blir NTNU sine ressurser missbrukt i utvinning av kryptovaluta''. Idémyldring prosessen var litt annerledes enn de vi hadde i de andre oppgavene, der vi denne gangen hadde et mer idéskrivingsoppsett, der vi alle skrev inn de idéene vi hadde inn i et dokument.

\subsection{Resultater}
Etter at økten var ferdig ble det gjort en vurdering av hvilke momenter som hørte sammen eller hadde likhetstrekk og disse ble gruppert, se figur \ref{fig:idemyldring-hvordan} og \ref{fig:idemyldring-hvorfor} under.


%Dette kan være ved hjelp av script som miner mens en intern er på nettsiden eller annen skadevare.


\begin{figure}[H]
    \centering
    \includegraphics[scale=0.5]{case_3/bilder/idemyldring-hvordan.pdf}
    \caption[Idémyldring]{Resultater og gruppering av Hvordan}
     \label{fig:idemyldring-hvordan}
\end{figure}

\begin{figure}[H]
    \centering
    \includegraphics[scale=0.5]{case_3/bilder/idemyldring-hvorfor.pdf}
    \caption[Idémyldring]{Resultater og gruppering av Hvorfor}
    \label{fig:idemyldring-hvorfor}
\end{figure}

RESULTAT

\subsection{Konklusjon av verktøyet}
Verktøyet hjelper til med å få økt forståelse på hva som er årsaker til problemstilingen eller i denne samenheng to problemstilinger. Finnes det en klar problemstiling er det lett å komme med ideer på hva som er årsakene. Det gir et god overordnet bilde av situasjonen, men siden vi har lite tid på dette caset kommer det veldig mange idéer, der noen trolig blir unødvendige å følge.

\section{Nominell gruppeteknikk}
Nominell gruppeteknikk anbefales dersom det er idéer som må prioriteres. 

\subsection{Ønsket utbytte}
Ettersom vi har dårligere tid på dette caset enn på de tidligere, har vi valgt å benytte NGT for å prioritere idéer. NGT er også et verktøy vi er interessert i å prøve ut for hovedrapporten.

\subsection{Gjennomføring}
NGT ble utført med at vi alle satt oss ned sammen og hadde 15 poeng hver å gi til forskjellige idéer. Idéene ble laget utifra forarbeidet som var blitt gjort i idémyldringsprosessen, de idéene som lignet på hverandre ble slått sammen og noen ble litt omformulert. Idéene skulle få poengene 1, 2, 3, 4 eller 5 og alle disse 15 poengene skulle gis ut. Hver person ga ut sine 15 poeng for hva de trodde ville være det viktigste å fokusere på videre i analysen, og disse dataene kan sees i tabell \ref{tab:NGT}.

\subsection{Resultater}
Etter at hvert gruppemedlem hadde gitt ut sine 15 poeng satt vi igjen med 4 idéer som stakk seg ut med hvor mange poeng de fikk, disse er i fet skrift i tabell \ref{tab:NGT}. Disse fire årsakene er de vi kommer til å fokusere på i dette caset. De fire fokuspunktene kan deles etter de to problemstillingene ``hvordan og hvorfor''.

De to årsakene knyttet til ``hvorfor problemstillingen'' går ut på at det ikke er ulovlig med kryptoutvinning i henhold til gjeldende regelverk og faller derfor i en gråsone. En får ikke noen represalier for å holde på med kryptoutvinning, annet enn å bli kastet av nettet.

De neste årsakene går ut på hvilke aktører som bruker universitetet sine ressurser til utvinning av kryptovaluta, og hvordan de bruker ressursene. Internt misbruk av labutstyr går ut på at studenter eller ansatte har tilgang til diverse labber og PCer som kan brukes til utvinning av kryptovaluta. Både ondsinnet programvare som utvinner kryptovaluta og webintegrerte utvinnere gjøres av eksterne aktører. Pengene som tjenes her går som regel tilbake til diverse kriminelle nettverk. Det med utvinning av kryptovaluta har blitt mye mer utbredt som et alternativ til ransomware. Ransomware har blitt mindre lønnsomt i den siste tiden \cite{RW}.

\begin{table} [H]
    \begin{tabular}{ | m{2em} | m{30em} | m{3em} | }
        \hline
            \cellcolor{yellow}  & \cellcolor{yellow} \textbf{Årsak} & \cellcolor{yellow} Poeng \\
        \hline
           A& Bra utstyr de ikke betaler for & 5 \\
        \hline
          B & Bruteforce av servere & 0 \\
        \hline
          C & Enkel tilgang til datamaskiner på campus & 0 \\
        \hline
         \textbf{D} & \textbf{Ingen regler om utvinning av kryptovaluta i IT reglementet} & 11 \\
        \hline
          \textbf{E} & \textbf{Internt misbruk av labutstyr} & 8  \\
        \hline
          F & Kryptovaluta har hatt en stor økning i verdi det siste året & 0 \\
        \hline
         G & Liten sannsynlighet for å bli tatt & 2 \\
        \hline
         \textbf{H} & \textbf{Ondsinnet programvare som miner} &  16 \\
        \hline
         I & Slipper å betale for energi & 0 \\
        \hline
         J & Store svingninger, mulighet for “get rich quick" & 0 \\
        \hline
         K & Utilstrekkelig tilgangskontroll på supermaskiner & 0 \\
        \hline
         L & Utvinningen av kryptovaluta krever mer og mer datakraft & 4 \\
        \hline
         \textbf{M} & \textbf{Webintegret miner} & 14 \\
        \hline
    \end{tabular}
    \caption{Oversikt over prioritering av idéer ved hjelp av NGT}
    \label{tab:NGT}
\end{table}

\subsection{Konklusjon av verktøyet}
Verktøyet hjelper til med å bestemme hvilke idéer fra tidligere idémyldring det er viktig å følge videre i prosessen. Det negativt med denne metoden, er at alle har like mye de skulle ha sagt om en sak, selvom de kanskje ikke har like mye kunnskap om de forskjellige momentene. Det positive er at vi får prioritert årsakene i henhold til viktighet. 
\chapter{Datainnsamling}
Målet med denne fasen er å samle inn informasjon om problemet, og prøve å få informasjon som kan være med på å identifisere rotårsaken i de senere stegene i prosessen. Siden vi hadde ganske lite tid på dette siste caset, består informasjonsinnsamlingen vår av bare et kvalitativt intervju med Christoffer Vargtass som jobber på SOCen til NTNU. Vi håper at siden problemstillingen er av en så stor teknisk art holder det å snakke med Christoffer som har god teknisk erfaring og kunnskap fra NTNU sin SOC.

Målet i denne fasen er å samle inn et så bredt aspekt av informasjon som mulig gjennom et par mulige verktøy og teknikker beskrevet i boka om rotårsaksanalyse\cite{RCA}. Dette kapittelet går inn på hvordan informasjonen til caset ble samlet inn.

\section{Kvalitativt intervjue}
Før intervjuet med Christoffer fikk vi et utdrag av alarmer som SOCen får av de kjente signaturene som har med cryptomining å gjøre. Disse alarmene stammer fra minere som folk har fått lagt inn på pcene sine. Minere i browser som er lagt inn i diverse nettsider og trojanere. 

\section{Ønsker utbytte}
Ønsket utbytte med dette intervjuet var å få et overblikk over hvordan minerene blir brukt, hvordan de eksterne aktører får minerne inn på pcene til de som oppholder seg på NTNU nettet. 

\section{Gjennomføring}
Vi utformet et intervju som skulle brukes med Christoffer, sprøsmålene bestod av diverse temaer vi ønsket å få lys på for å finne rotårsaken. En del av spørsmålene gikk på å finne ut hvordan utvinningen blir oppdaget, hva slags handlingsrom de har for å håndtere hendelser, og hvordan utvinningen som regel foregår.
\section{Resultater}


\section{Konklusjon av verktøy}
Intervjuet fungerte bra, vi fikk mye informasjon om de tekniske aspektene ved problemet, og litt informasjon om mulige tiltak, og hvorfor noen av disse tiltakene ikke allerede er blitt implementert. Vi skulle gjerne ha hatt flere intervjuer å base dataanalysen vår på i neste steg, men på grunn av tidsbegrensninger ble dette ikke gjort.
\chapter{Dataanalyse}
I denne fasen analyseres dataene som er samlet inn, vi har ganske lite data i dette caset, vi har derfor et ganske lite sett med verktøy å bruke fra RCA boka \cite{RCA}

\section{Affinitetsdiagram}
Affinitetsdiagram brukes til å analysere data som det ikke er mulig å nummerere, eksempelvis meninger eller ideer. Affinitetsdiagram grupperer data og finner de underliggende korrelasjoner og likhetstrekk i gruppen.


\section{Ønsket utbytte}
Ønsket utbytte av å bruke affinitetsdiagram er å finne bindinger\/fellesnevnere som kan være til hjelp for å fjerne rotårsaken. 

\section{Gjennomføring }
Analysen ble gjennomført med å ta transkripsjon av intervjuet og stykke den opp i fem hovedgrupper.   

\begin{figure}[H]
    \centering
    \includegraphics[scale=0.6]{case_3/bilder/AD.pdf}
    \label{fig:AD_miner}
    \caption{Hvordan fungerer utvinning av kryptovaluta ved NTNU?}
\end{figure}

Vi finner mulige årsaker og tiltak som er satt på plass, og forhåpentligvis er rotårsaken blant dem. 

\section{Konklusjon av verktøy}
Verktøyet fungerte godt, selv med en liten datamengde. Vektøyet er effektivt til å strukturere transkripsjonen fra intervjuet til mer brukbare og oversiktlige nøkkelpunkter.

\chapter{Rotårsaksidentifisering}
Arbeidet i denne fasen går ut på å identifisere rotårsaken. I foregående fasen ble en rekke mulige årsaker identifisert og analysert, men nå er det tid for å finne den faktiske rotårsaken. Det er mange forskjellige verktøy som kan brukes i denne fasen, men vi har brukt et 5 whys og feiltreanalyse for vårt utgangspunkt. 

\section{5 whys}
5 Whys er et verktøy som prøver å gjøre et dypdykk i årsakene for å finne rotårsaken. Måten dette gjøres på er å hele tiden spørre ``Why?'', altså hvorfor på norsk, hver gang en ny årsak dukker opp. Det brukes ofte for å sjekke om de identifiserte årsakene er symptomer, lav-nivå årsaker eller rotårsaker. 

\subsection{Ønsket utbytte}
Ved å bruke 5 Whys prøver vi å finne ut hva som er rotårsaken til problemet.

\subsection{Gjennomføring}
Med dette verktøyet tar vi utgangspunkt i casebeskrivelsen; nemlig rotårsaken til kryptoutvinning på NTNU. Ut fra dette brukte vi funnene fra analysen for å komme på årsaker, samt prøvde å idémyldre et par nye. For hver av disse årsakene ble det spurt: ``Hvorfor er dette en årsak av det originale problemet?''. For hvert svar spør vi hvorfor igjen og igjen helt til vi finner rotårsaken. Det ble tatt utgangspunkt i fem iterasjoner, men det er mulighet for flere eller færre avhengig av om spørsmålet kan besvares på en fornuftig måte. 

\subsection{Resultater}
Det ble fremhevet fem årsaker som skulle analyseres. Fire av disse kom fra fiskebeindiagrammet over, og en fra idémyldring. Tabellene under viser resultatene fra gjennomføringen. 

\begin{table} [H]
    \centering
    \begin{tabular}{ | m{5em} | m{30em} | }
        \hline
            \cellcolor{yellow} Årsak: & \cellcolor{yellow} Ansatte og studenter utvinner krypto med universitetet sine ressurser                \\
        \hline
            Why? & Lønnsomhet                                    \\
        \hline
            Why? & Har ingen utgifter                                            \\
        \hline
            Why? & Bruker strøm og infrastrukturen til skolen                \\
        \hline
            Why? & Det er en gråsone i regelverket           \\
        \hline
            Why? & Ikke spesifisert godt nok i IT-reglementet   \\
        \hline
    \end{tabular}
    \caption[5 Whys: Ansatte og studenter utvinner krypto med skolens]{5 Whys på ansatte og studenter utvinner krypto med skolens}
    \label{5Whys-interne}
\end{table}
Det å utvinne krypto på universitetet sine ressurser er alt fra å kjøre en kryptominer på en PC til å sette opp en mining rig. I 5 Whys over kom vi frem til at lønnsomhet er primærgrunnen til at de driver med kryptoutvinning, men årsaken til at ansatte og studenter utvinner på universitet er at det ikke er spesifisert godt nok i IT-reglementet.   


\begin{table} [H]
    \centering
    \begin{tabular}{ | m{5em} | m{30em} | }
        \hline
            \cellcolor{yellow} Årsak: & \cellcolor{yellow} Eksterne trusselaktører utvinner krypto med skolens ressurser              \\
        \hline
            Why? & Lønnsomhet                                   \\
        \hline
            Why? & Enkelt å spre minere                                           \\
        \hline
            Why? & Folk går inn på waterholes og trykker på phishingmail               \\
        \hline
            Why? & Brukeren var ikke oppmerksom nok på e-mailen eller siden de gikk på           \\
        \hline
            Why? & Brukere har ikke fått nok opplæring i hvordan dette unngås    \\
        \hline
    \end{tabular}
    \caption[5 Whys: Eksterne trusselaktører utvinner krypto med skolens ressurser]{5 Whys Eksterne trusselaktører utvinner krypto med skolens ressurser}
    \label{5Whys-eksterne}
\end{table}

Årsaken til at eksterne trusselaktører utvinner krypto med skolen sine ressurser er fordi det er en lønnsom affære som er koster lite å distribuere og som det er liten sannsynlighet for å bli tatt for. Med eksterne trusselaktører mener vi folk som utvinner på andre sine datamaskiner gjennom å få skadevare installert på disse.

\begin{table} [H]
    \centering
    \begin{tabular}{ | m{5em} | m{30em} | }
        \hline
            \cellcolor{yellow} Årsak: & \cellcolor{yellow} Utvinnere som implementert inn i nettsider              \\
        \hline
            Why? & God fortjeneste                                   \\
        \hline
            Why? & Fordi de når en stor menge folk som utvinner krypto for dem                                           \\
        \hline
            Why? & Mange har ikke en annonseblokkering som også stopper utvinnere på nett               \\
        \hline
            Why? & På grunn av lite eller ingen opplæring til denne typen programvare           \\
        \hline
            Why? & Ikke prioritert    \\
        \hline
            Why? & Fordi det ikke er nok folk/ressurser    \\
        \hline
    \end{tabular}
    \caption[5 Whys: Minere som er implementert inn i nettsider]{5 Whys på ansatte og studenter utvinner krypto med skolens}
    \label{5Whys-minere}
\end{table}
Årsaken til at nettsider som har muligheten til å utvinne kryptovaluta er et problem er fordi, de starter å bli utbredt og de spør ofte ikke om godkjenning for utvinning på PCene til folk.

\subsection{Konklusjon av verktøy}
Verktøyet var svært nyttig for å gå dypt inn i årsakene. Et problem består ofte av flere nivåer av årsaker, og det tar 5 Whys hensyn til. I forhold til informasjonssikkerhet burde man passe på å ikke alltid ende opp med en årsak som er relatert til IT-reglement, da problemet ofte også kan være av teknisk art. Dette var en mulig fallgruve for oss, men det er usikkert om det bare gjelder dette caset. Et annet problem med verktøyet er at det kan bli for ensporet på en spesifikk tankegang. Det kan derfor anbefales i noen tilfeller å undersøke samme årsaken flere ganger dersom det er grunn til å tro at det finnes flere relevante 
årsaker ved ett av spørsmålene, som fører til en annen rotårsak.

\section{Feiltreanalyse}
Feiltreanalyse tar alle mulige årsaker i et diagram og identifiserer mulige linker. Analysen bygger på hva som ble gjort i 5 Whys. 

\subsection{Ønsket utbytte}
Ved bruk av dette verktøyet ønsker vi å få en oversikt over koblinger mellom de forskjellige årsakene. Vi ønsker også å få sortert ut de årsakene som NTNU ikke har mulighet til å gjøre.

\subsection{Gjennomføring}
Med dette verktøy tar vi utgangspunktet i resultatet fra 5 Whys til å finne rotårsakene. Her går vi steg for steg nedover og ser på hva som er årsaken til at enhver uønsket hendelse inntreffer.

\subsection{Resultater}
Vi har kommet fram til fire hovedgrunner til at kryptoutvinning på NTNU forekommer. Rotårsaken er sammensatt av disse årsakene definert i figur \ref{fig:feil_tre_analyse}. I dette caset er problemet delt inn i to deler; de interne og de eksterne. Det er to forskjellige typer årsaker, der interne går mer på regelverk og eksterne er mer teknisk.          

I figur \ref{fig:feil_tre_analyse} representerer trekanter ``eller'', halvsirkelen representer ``og''. De røde boksene er de årsakene vi ikke kan gjøre noe med, de grønne kan det gjøres noe med.
 \begin{figure}[H]
    \centering
    \includegraphics[scale=0.45]{case_3/bilder/feil_tre_analyse.pdf}
    \caption[Feiltreanalyse]{Feiltreanalyse}
    \label{fig:feil_tre_analyse}
\end{figure}

\subsection{Konklusjon av verktøy}
Verktøyet funger godt til å se hvordan de forskjellige årsakene er koblet sammen og hva som er hovedårsakene. Der man ser hvordan interne og eksterne er koblet sammen og hvordan de er forskjellige. 
\chapter{Rotårsakseliminering}
Arbeidet i denne fasen går ut på å finne tiltak som vil fjerne rotårsaken.

\section{Systematisk Innovativ Tenkning (SIT)}
Systematic Inventive Thinking inneholder fem hovedprinsipper:

\begin{enumerate}
    \item \textbf{Attributtavhengighet} Endre på en essensiell variabel.
    \item \textbf{Komponentkontroll} Ser på hvordan et produkt er tilknyttet omgivelsene sine.
    \item \textbf{Erstatning} Bytte ut en del av produktet med noe i omgivelsene til produktet.
    \item \textbf{Forkastning} Fjerne en del av produktet for å bedre det.
    \item \textbf{Oppdeling} Prøver å splitte et produkts attributter i to.
\end{enumerate}

\subsection{Ønsket utbytte}
Ved å bruke SIT-metoden ønsker vi å få kreative idéer på hvordan vi kan finne en løsning til utvinning av kryptovaluta ved NTNU. 

\subsection{Gjennomføring}
SIT burde helst gjennomføres av 10-12 personer, fra en rekke forskjellige fagområder, men siden vi ikke hadde så mange tok vi bare utgangspunkt i prosjektgruppen. 
\subsubsection{Komponenter} 
Her blir alle komponenter som omhandler problemet listet.
\begin{itemize}
    \item IT-reglement
    \item Annonseblokker
    \item Internett
    \item SOCen
    \item Brannmur
    \item Servere og datamaskiner
    \item Datalabber
    \item HPC-cluster
    \item Bring your own device (BYOD)
    \item Strøm
\end{itemize}

Når komponentene er gjort rede for, vil de fem SIT-prinsippene brukes sekvensielt på komponentene for å utvikle løsninger på problemene. Resultatene fremheves i neste seksjon. 


\subsection{Resultater}
Ikke alle SIT-prinsipper finner løsninger som er gjennomførbare for alle komponenter. I disse tilfellene vil det stå: ``Ikke gjennomførbart'

\paragraph{IT-reglement}
\begin{itemize}
    \item \textbf{Attributtavhengighet} Legge til et større fokus på utvinning av krypto.
    \item \textbf{Komponentkontroll} Gjennomføre en informasjonskampanje for å sette fokus på hva som er misbruk.
    \item \textbf{Erstatning} Ikke gjennomførbart.
    \item \textbf{Forkastning} Ikke gjennomførbart.
    \item \textbf{Oppdeling} Ikke gjennomførbart. 
\end{itemize}

\paragraph{Annonseblokker}
\begin{itemize}
    \item \textbf{Attributtavhengighet} Legge til blokkering av mining 
    \item \textbf{Komponentkontroll} Passe på at alle ansatte har annonseblokker installert som også stopper utvinning av krypto.
    \item \textbf{Erstatning} Ikke gjennomførtbart
    \item \textbf{Forkastning} Ikke gjennomførtbart
    \item \textbf{Oppdeling} Ikke gjennomførtbart.
\end{itemize}

\paragraph{Internett}
\begin{itemize}
    \item \textbf{Attributtavhengighet} Blokkere kryptoutvinning
    \item \textbf{Komponentkontroll} Automatisere at alle datamaskiner som utvinner krypto blir kastet av nettet.
    \item \textbf{Erstatning} Ikke gjennomførtbart.
    \item \textbf{Forkastning} Ikke gjennomførtbart.
    \item \textbf{Oppdeling} Ikke gjennomførtbart.
\end{itemize}


\paragraph{SOC}
\begin{itemize}
    \item \textbf{Attributtavhengighet} Øke antall ansatte.
    \item \textbf{Komponentkontroll} Økt prioritet til krypto.
    \item \textbf{Erstatning} Ikke gjennomførtbart.
    \item \textbf{Forkastning} Ikke gjennomførtbart.
    \item \textbf{Oppdeling} Gi forslag til implementering av tiltak til bachelorgrupper.
\end{itemize}

\paragraph{Servere og datamaskin}
\begin{itemize}
    \item \textbf{Attributtavhengighet} Strengere adgangskontroll.
    \item \textbf{Komponentkontroll} Ikke gjennomførtbart.
    \item \textbf{Erstatning} Ikke gjennomførtbart.
    \item \textbf{Forkastning} Ikke gjennomførtbart.
    \item \textbf{Oppdeling} Ikke gjennomførtbart.
\end{itemize}

\paragraph{Datalabber}
\begin{itemize}
    \item \textbf{Attributtavhengighet} Strengere adgangskontroll.
    \item \textbf{Komponentkontroll} Logging.
    \item \textbf{Erstatning} Svakere maskinvare på labbene.
    \item \textbf{Forkastning} Slutte å tilby labber, som kan brukes i sammenheng med krypto utvinning.
    \item \textbf{Oppdeling} Ikke gjennomførbart.
\end{itemize}

\paragraph{HPC-clustere}
\begin{itemize}
    \item \textbf{Attributtavhengighet} Øke tilgangskontrollen ytterligere.
    \item \textbf{Komponentkontroll} Ikke gjennomførbart.
    \item \textbf{Erstatning} Ikke gjennomførbart.
    \item \textbf{Forkastning} Ikke gjennomførbart.
    \item \textbf{Oppdeling} Ikke gjennomførbart.
\end{itemize}


\paragraph{Bring your own device (BYOD)}
\begin{itemize}
    \item \textbf{Attributtavhengighet} Ikke gjennomførbart.
    \item \textbf{Komponentkontroll} Kaste datamaskiner som ikke tilhører NTNU-personell av nettet slik at de manuelt må koble seg på igjen.
    \item \textbf{Erstatning} Ikke gjennomførbart.
    \item \textbf{Forkastning} Ikke gjennomførbart.
    \item \textbf{Oppdeling} Ikke gjennomførbart.
\end{itemize}

\paragraph{Strøm}
\begin{itemize}
    \item \textbf{Attributtavhengighet} Ikke gjennomførtbart
    \item \textbf{Komponentkontroll} Strømkvotering, overstiges kvoten må vedkommende betale for strømmen
    \item \textbf{Erstatning} Ikke gjennomførtbart
    \item \textbf{Forkastning} Ikke gjennomførtbart
    \item \textbf{Oppdeling} Ikke gjennomførtbart
\end{itemize}

Vi sorterer og beskriver de mest relevante idéer til videre utdyping:

\begin{description}
\item[Gjennomføre en informasjonskampanje om kommersielt misbruk av NTNU sin infrastruktur] Kampanjen skal få frem at det å bruke NTNU sine ressurser til kommersiell virksomhet bryter IT-reglementet, og at strøm inngår i NTNU sine ressurser.
\item[Legge til et større fokus på utvinning av krypto i IT-reglementet] Gi IT-reglement et større fokus på kryptoutvinning og klarere retningslinjer på hva som ikke er greit å gjøre.
\item[Legge til annonseblokker som stopper utvinning] Aktivere blokkering av utvinning-protokoll på annonseblokkere og passe på at alle har en annonseblokkeringstjeneste installert.
\item[Blokkere kryptoutvinning] Med dette mener vi å gjøre et eller flere tiltak som å blokkere DNS-forespørsel som omhandler kryptoutvinning.  
\item[Øke antall personell i SOC] SOC har mange oppgaver som er mer kritiske enn kryptoutvinning. Derfor foreslår vi å ansette flere, kanskje i kombinasjon med bacheloroppgaver.
\item[Strengere adgangskontroll] Begrenser tilgang til datalabber. 
\item[Logging] Øke bruk av logging  i datalabbene. 
\item[Kaste datamaskiner som ikke er kritisk infrastruktur av nettet] Ved midnatt blir alle datamaskiner eller servere som ikke er kritisk infrastruktur koblet av nettet og må manuelt koble seg på nettet igjen.
\end{description}

\subsection{Tiltaksplan}
Etter å ha brukt de fem SIT-prinsippene på hver komponent, og filtrert de, sitter vi igjen med et par idéer. I denne delen fremhever vi idéer i en tiltaksplan som vi anbefaler å implementere. 
Under beskrives de ulike tiltakene:

\begin{description}
    \item[Gjennomføre en informasjonskampanje om kommersielt misbruk av NTNU sin infrastruktur] Utvinning av kryptovaluta er en ny ting, hvor mange ikke er klar over hvordan universitetet sitt regelverk håndterer temaet. Vår anbefaling er å ha en kampanje der universitet informerer om hva som regnes som NTNU sine ressuser og hvordan disse ikke skal brukes til kommersiell virksomhet.
    \item[Legge til et større fokus på utvinning av krypto i IT-reglementet] Endre IT-reglementet slik at det blir tydelig at kryptoutvinning ikke er lovlig bruk av NTNU sine ressurser. 
    \item[Blokkere kryptoutvinning] Dette tiltaket går ut på å blokkere DNS-forespørsler tilknyttet kryptoutvinning. Slik at PCer som blir brukt i utvinning ikke kan hente nye oppgaver å løse. De velkjente DNSene blokkeres. Videre kan loggen bli benyttet for å legge nye domener inn i en svarteliste, eller justere de gamle DNSene.
    \item[Øke antall personell i SOC] SOCen kan ikke prioritere å stoppe utvinning av kryptovaluta. Derfor anbefaler vi å enten øke mengden personell i SOCen, eller gi utvikling av implementasjonsstrategi som en bachelor oppgave.
\end{description}


\subsection{Konklusjon av verktøyet}
 Verktøyet fungerte greit, der sorteringen og tiltaksplan fungerte godt. Det å finne komponentene og gjøre de fem SIT-prinsippene er der SIT hvertfall i informasjonsikkerhetssammenheng  ser ut til å fungere dårlig. Vi prøvde SIT i alle casene for å se hvor bra det gikk, og har kommet fram til at selv om sorteringen og tiltaksplanen fungerer bra vil vi heller anbefale seks tenkehatter for rotårsakselimineringen. Da det å bruke SIT-prinsippene i informasjonsikkerhet er tungvint og ser ut til å passe mer for et fysisk produkt. 
\chapter{Løsningsimplementering}
Arbeidet i denne fasen går ut på å utrede en tiltaksplan og lage et forslag til hvordan dette skal implementeres. I den foregående fasen ble løsningene til rotårsakene identifisert. Vi kom fram til fire tiltak som vil fjerne rotårsaken. Tiltakene kan deles inn i to deler, en for eksterne og en for interne. Den eksterne løsningen går på det tekniske og den interne går på IT-reglementet. 

Den eksterne løsningen er å øke ressursene til seksjonen for digital sikkerhet gjennom å ansatte flere eller å bruke bachelorstudenter til å utføre blokkering av DNS-forespørsler tilknyttet kryptoutvinning.

Den interne løsningen er å tydeliggjøre at det å drive kryptoutvinning på NTNU ikke er lovlig og gjennomføre enn informasjonskampanje rundt dette.

\section{Kraftfeltsanalyse}
Kraftfeltsanalyse er et verktøy som analyserer hva som hjelper og hva som hindrer implementering av tiltaket.  

\subsection{Ønsket utbytte}
Ønsket utbytte fra kraftfeltsanalyse er å få vite hva som er for og hva som er imot implementering av tiltakene. Dette verktøyet gir en plan over hvilke tiltak som er lettest å gjennomføre. 

\subsection{Gjennomføring}
Kraftfeltsanalysen ble gjort ved at vi tok tiltakene fra problemelimineringen og hadde en idémyldring for å se hva som talte for tiltakene og hva som var imot. 

\subsection{Resultat}
 Under har vi de fire kraftfeltsanalysene
 
 Informasjonskampanjen og endringen i IT-reglementet bør gjøre i kombinasjon med hverandre. Der IT-reglementet får klartgjort at selv om kryptoutvinning ikke ulovlig i henhold til norsk lov, er det imot NTNU sitt IT-reglement så langt det ikke er søkt om. Når endringen er gjort, gjennomføres informasjonskampanjen.   
 Under, i tabell \ref{fig:kampanje} og \ref{fig:IT-reglement}, viser resultatene fra kraftfeltsanalyse på informasjonskampanje og endring i IT-reglementet.
 \begin{figure}[H]
    \hspace{2.2cm}
    \includegraphics[scale=0.6]{case_3/bilder/Force-field1.pdf}
    \caption[Informasjonskampanje]{Oversikt over informasjonskampanjen }
    \label{fig:kampanje}
\end{figure}

 
 \begin{figure}[H]
    \hspace{2.6cm}
    \includegraphics[scale=0.6]{case_3/bilder/Force-field2.pdf}
    \caption[Endre IT-reglementet]{Endring i IT-reglementet}
    \label{fig:IT-reglement}
\end{figure}

Fra dataanalysen kom fram til at selv om det finnes tekniske løsninger, har ikke SOCen hatt mulighet til å implementere DNS blokkering på bakgrunn av mangel på ressurser. Figur \ref{fig:Blokkering} og \ref{fig:Oke-antall} viser hva som skal til for å blokkere DNS og hva som må til for å øke ressursene til SOCen.    
 \begin{figure}[H]
    \centering
    \includegraphics[scale=0.6]{case_3/bilder/Force-Field3.pdf}
    \caption[Blokkering]{Blokkering av DNS forespørsel}
    \label{fig:Blokkering}
\end{figure}

 \begin{figure}[H]
    \hspace{3.6cm}
    \includegraphics[scale=0.6]{case_3/bilder/Force-field4.pdf}
    \caption[Øke antall ansette i SOC]{Øke andel ansatte i SOC}
    \label{fig:Oke-antall}
\end{figure}

Figurene viser våres antakelser på hva som jobber for implementeringen og hva som jobber imot samt  estimater på styrken til antakelsene. 
\subsection{Konklusjon av verktøyet}
Verktøyet har potensiale til å fungere bra, der oversikten man får er bra så lenge datagrunnlaget er godt. Problemmet vårt var at vi jobbet med antakelser og ikke et datasett. Vi hadde ikke tid til å undersøke estimatene, så de har en høy usikkerhet.     
\chapter{Diskusjon og konklusjon}
\subsection{Diskusjon}


\subsubsection{Hva er rotårsaken til at NTNU sin infrastruktur blir misbrukt til utvinning av kryptovaluta?}
Caset virket i første omgang som temmelig rett frem, der man så at ønske om å bli fort rik var rotårsaken. Da oppgaven ble studert nøyere oppdagdes det at den er langt mer komplisert. Det er ikke en klar rotårsak, men en kombinasjon av flere årsaker, deriblant lønnsomhet.

\subsection{Rotårsak: Uklarhet i IT-reglementet angående kryptoutvinning}
Vi har vurdert uklarhet i IT-reglementet som hovedårsaken bak utvinning hos de ansatte og studenter, der de utnytter universitetets ressurser. Dette gjør vi fordi IT-reglementet ikke nevner utvinning av kryptovaluta eksplisitt og fordi kryptovaluta har vært en trend i media den siste tiden. Siden kryptoutvinning ikke er ulovlig og har blitt betegnet som den nye måten å bli rik på, tenker nok flere ikke over at personlig vinning ikke er lov i henhold til IT-reglement. Her er det informasjonskampanjen kommer inn. Den vil gjøre at folk blir oppmerksom på hva de gjør og hvilke represalier som kan forekomme. Det er knyttet to svakhet til denne løsningen for hvorfor NTNUs ressurser blir misbrukt. Disse er: informasjonskampanje vil nødvendigvis ikke endre oppførselen til de som er klar over regelbruddet, og endringen i IT-reglementet og informasjonskampanjen vil kunne hjelpe til å stoppe den frivillige utvinningen av kryptovaluta.

%\subsection{Rotårsak: Eksterne aktører utvinner krypto med universitetets ressurser}
%En vanlig måte for eksterne aktører å utvinne på er å bruke PCene til intetanende som et botnet \cite{Botnet}. Her blir datamaskinene infisert av skadevare som får dem til å jobbe for den eksterne aktøren. Siden dette er en utbredt måte å angripe på, anser vi det som et godt tiltak å blokkere DNS-adressene til som blir mest brukt. En annen måte som ser ut til å bli mer vanlig er såkalt ``cryptojacking'', der kryptoutvinning blir gjort av javascriptkode på nettsider brukeren besøker \cite{12577042320171101}. 

\subsection{Rotårsak: Universitetet har ikke ressurser til å prioritere håndtering av kryptoutvinning}
Universitetet har 

\subsection{Tidsbruk}
\label{tidsbruk_case_3}


\subsection{Videre arbeid}




%\subsubsection{Hvor godt fungerer rotårsaksanalyse i et case som %omhandler misbruk av IT-infrastruktur med begrenset tid?}
%Rotårsaksanalyse fungerer ikke spesielt godt i et case der det er %sterke begrensninger på tid. Det vil si vi følte ikke at analysen ga %noe mer enn de løsningene som allerede var kjent. Siden %datainnsamling og dataanalyse er de mest tidkrevende elementete og %noen av de viktigste elementene i en rotårsaksanalyse.
\chapter{Vedlegg: Transcript av intervju}
\includepdf[pages=1-3]{bilder/transcript.pdf}
\label{Transcript_intervju}

\bibliographystyle{ntnuthesis/ntnubachelorthesis}
\bibliography{case_3/bibliografi}

\appendix %after this line all chapters will have leters instead of numbers
%\input{projectplan}
%\input{gantt}
%\input{main/meetinglog}
%\input{progressreviews}
%\input{worklog}

\end{document}