\chapter{Datainnsamling}
Målet med denne fasen er å samle inn informasjon om problemet, og prøve å få informasjon som kan være med på å identifisere rotårsaken i de senere stegene i prosessen. Siden vi hadde ganske lite tid på dette siste caset, består informasjonsinnsamlingen vår av bare et kvalitativt intervju med Christoffer Vargtass som jobber på SOCen til NTNU. Vi håper at siden problemstillingen er av en så stor teknisk art holder det å snakke med Christoffer som har god teknisk erfaring og kunnskap fra NTNU sin SOC.

Målet i denne fasen er å samle inn et så bredt aspekt av informasjon som mulig gjennom et par mulige verktøy og teknikker beskrevet i boka om rotårsaksanalyse\cite{RCA}. Dette kapittelet går inn på hvordan informasjonen til caset ble samlet inn.

\section{Kvalitativt intervjue}
Før intervjuet med Christoffer fikk vi et utdrag av alarmer som SOCen får av de kjente signaturene som har med cryptomining å gjøre. Disse alarmene stammer fra minere som folk har fått lagt inn på pcene sine. Minere i browser som er lagt inn i diverse nettsider og trojanere. 

\section{Ønsker utbytte}
Ønsket utbytte med dette intervjuet var å få et overblikk over hvordan minerene blir brukt, hvordan de eksterne aktører får minerne inn på pcene til de som oppholder seg på NTNU nettet. 

\section{Gjennomføring}
Vi utformet et intervju som skulle brukes med Christoffer, sprøsmålene bestod av diverse temaer vi ønsket å få lys på for å finne rotårsaken. En del av spørsmålene gikk på å finne ut hvordan utvinningen blir oppdaget, hva slags handlingsrom de har for å håndtere hendelser, og hvordan utvinningen som regel foregår.
\section{Resultater}


\section{Konklusjon av verktøy}
Intervjuet fungerte bra, vi fikk mye informasjon om de tekniske aspektene ved problemet, og litt informasjon om mulige tiltak, og hvorfor noen av disse tiltakene ikke allerede er blitt implementert. Vi skulle gjerne ha hatt flere intervjuer å base dataanalysen vår på i neste steg, men på grunn av tidsbegrensninger ble dette ikke gjort.