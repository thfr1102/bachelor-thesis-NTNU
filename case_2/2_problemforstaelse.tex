\chapter{Problemforståelse}
Denne fasen eksisterer for å passe på at en har forstått problemet i dypere detalj. Verktøyene som er relevante å bruke skal gi en bedre forståelse av blant annet omfanget og de ulike aspektene ved et problem. Jo bedre tilgang en har på logger og dokumentasjon, jo bedre vil denne fasen kunne utføres. Grunnet god erfaring ved bruk av verktøyet kritiske hendelser i det foregående caset, og på grunn av tilgang på relevant informasjon til bruk i verktøyet, har vi igjen valgt å bruke dette. 


\section{Kritiske hendelser}
For å gå lære mer om bakgrunnen til problemet bruker vi verktøyet kritiske hendelser for å se på frekvensen av misbruk som er registrert fra de kompromitterte kontoene. Slik kan vi kartlegge og forstå hva de kompromitterte kontoene brukes til. Denne informasjonen ble gitt fra oppdragsgiver.

\subsection{Ønsket utbytte}
Ved bruk av dette verktøyet ønsker vi å få en oversikt over hvilke handlinger de kompromitterte kontoene utfører. Dette går i stor grad ut på hva som er motivasjonen til trusselaktørene. Vi ønsker å danne et bilde av hva de ønsker å oppnå ved å kompromittere kontoene, slik at vi kan bruke den informasjonen senere til å finne rotårsaken til at de blir kompromittert. 

\subsection{Gjennomførelse}
Sammen med oppgavebeskrivelsen fikk vi en liste over loggførte sikkerhetshendelser som hadde foregått det siste året hvor kompromitterte kontoer var involvert. Vi sorterte disse dataene i synkende rekkefølge og la det inn i en tabell for å se kritikaliteten til de enkelte sikkerhetshendelsene, og dermed fokusområdene til trusselaktørene. 

\subsection{Resultater}
\begin{table} [H]
    \begin{tabular}{ | m{18em} | m{18em} | }
        \hline
            \cellcolor{yellow} Sikkerhetshendelser & \cellcolor{yellow} Frekvens \\
        \hline
            Spam & 46  \\
        \hline
            Misuse (uthenting av forskningsartikler) & 26 \\
        \hline
            Negligible/Fixed/Failed Attack  & 8 \\
        \hline
            Phishing & 7 \\
        \hline
            Whaling & 2 \\
        \hline
            Brute force & 2 \\
        \hline
            DDOS out & 1 \\
        \hline
            Traded credentials & 1 \\
        \hline
            Hackingtools exploits and kits & 1 \\
        \hline
            Copyright/Piracy & 1 \\
        \hline
    \end{tabular}
    \caption{Oversikt over hva kompromitterte ansattkontoer blir brukt til}
    \label{kritisk_tabell_2}
\end{table}

Fra tabellen ser vi at spam er den hendelsen med høyest frekvens, men etter samtaler med oppdragsgiver var ikke dette problemet av størst viktighet, da dette er noe som enkelt blir lagt merke til, og er trolig ikke hovedgrunnen til at aktørene aktivt går inn for å kompromittere NTNU sine brukerkontoer. Når det kommer til misuse i tabellen ovenfor referer det til hendelser der uvedkommende misbruker NTNU sine ressurser, spesielt i form av å stjele forskningsartikler på NTNU sin regning. Dette var en av de største problemene med kompromittering av kontoene, siden det førte til økonomisk tap for NTNU. 


\subsection{Konklusjon av verktøyet}
I dette caset fungerte metoden ekstremt godt siden vi hadde all data på forhånd, og trengte bare få satt det i system. Verktøyet hjalp veldig til å se hvilke symptomer som er av høyest frekvens. Det negative med verktøyet i dette caset var at det ikke beskriver hvor kritiske enkelthendelsene er, men bare hvor mange det er av de. Dette måtte vi forhøre oss med oppdragsgiver for å komme fram til. I tabellen vår ser vi at spam ligger på topp, men det viser seg at misuse egentlig er mer kritisk, selv om det ligger under. 




