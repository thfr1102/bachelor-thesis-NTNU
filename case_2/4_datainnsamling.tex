\chapter{Datainnsamling}
Målet i denne fasen er å samle inn et så bredt aspekt av informasjon som mulig gjennom et par mulige verktøy og teknikker beskrevet i boka om rotårsaksanalyse\cite{RCA}. Dette kapittelet går inn på hvordan informasjonen til caset ble samlet inn.



%----------------------KVANTITATIVE SPØRREUNDERSØKELSE--------------------------
\section{Elektronisk spørreundersøkelse}
Det finnes i hovedsak to forskjellige undersøkelsestyper, kvantitative og kvalitative spørreundersøkelser. Kvalitative undersøkelser går ut på å spørre få personer, og samle mer detaljerte svar av høyere kvalitet, gjerne langsvar. Kvantitative undersøkelser fungerer motsatt i at det er fokus på mange tilbakemeldinger slik at en kan senke usikkerhet knyttet til svarkvalitet og gjøre statistisk analyse på resultatene. I vår undersøkelse har vi i hovedsak valgt å fokusere på en kvantitativ undersøkelse, samt gjøre kvalitative telefonintervjuer for å fylle hull der det trengs.

Grunnen til at vi valgte kvantitativ spørreundersøkelse er at vi ønsker at den skal være så anonym som mulig, siden spørreundersøkelsen omhandler temaer som ansatte kanskje vil finne sensitive. 

\subsection{Ønsket utbytte}
Med den elektroniske spørreundersøkelsen ønsker vi å få informasjon fra personer som allerede har fått brukeren sin kompromittert. Informasjonen vil bestå av blant annet personens passordvaner, kjennskap til retningslinjer om passordbruk og epost-aktivitet. 

\subsection{Gjennomførelse}


\subsection{Resultater}


\subsection{Konklusjon av verktøy}




%--------------------------KVALITATIVE INTERVJUER--------------------------------
\section{Kvalitative intervjuer}
Grunnen til at vi ønsker å ha kvalitative intervjuer i tilegg til de kvantitative, er å få en mer dybdeforståelse på hvorfor kontoer blir kompromittert. 

\subsection{Ønsket utbytte}
Ønsket utbytte av kvalitative intervjuer er å få mer utfyllende informasjon etter endt elektronisk spørreundersøkelse. Kvalitative intervjuer vil kunne gi mer utfyllende og nyansert informasjon enn en kvantitativ undersøkelse. Mer utfyllende og nyansert i form av at intervjuobjektene sine svar forhåpentligvis gir svar som er mer utfyllende, og at gjerne vil dele informasjon med noen de snakker med.

\subsection{Gjennomførelse}
Kvalitative intervjuer blir gjennomført ved å bruke telefonintervju som verktøy. 

Prosessen startet ved å utforme spørsmål som skal stilles til intervjukandidatene. Vi vil ringe de aktuelle kandidatene, de som har fått kontoen kompromittert. 

Et godt intervju vil være avhengig av gode spørsmål. Det er da vår oppgave å finne de beste spørsmålene, som vil gi oss nok informasjon om hvorfor konten ble kompromittert.  

\subsection{Resultater}


\subsection{Konklusjon av verktøy}
