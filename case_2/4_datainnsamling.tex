\chapter{Datainnsamling}
Målet i denne fasen er å samle inn et så bredt aspekt av informasjon som mulig gjennom et par mulige verktøy og teknikker beskrevet i boka om rotårsaksanalyse\cite{RCA}.



%----------------------KVANTITATIVE SPØRREUNDERSØKELSE--------------------------
\section{Elektronisk spørreundersøkelse}
Det finnes i hovedsak to forskjellige undersøkelsestyper, kvantitative og kvalitative spørreundersøkelser. Kvalitative undersøkelser går ut på å spørre få personer, og samle mer detaljerte svar av høyere kvalitet, gjerne langsvar. Kvantitative undersøkelser fungerer motsatt i at det er fokus på mange tilbakemeldinger slik at en kan senke usikkerhet knyttet til svarkvalitet og gjøre statistisk analyse på resultatene. I vår undersøkelse har vi i hovedsak valgt å fokusere på en kvantitativ spørreundersøkelse, med noen få kvalitative elementer. 

Grunnen til at vi valgte i hovedsak kvantitativ spørreundersøkelse er at vi ønsker å finne relasjoner mellom dataene vi samler inn. Det er også mulighet for å gjøre statistiske beregninger på disse, noe vi anser som relevant for datainnsamlingen i dette caset. Fremstilling av data i grafer og tabeller er også et moment som gjorde at valget ble kvantitativ metode. 

\subsection{Ønsket utbytte}
Med den elektroniske spørreundersøkelsen ønsker vi å få informasjon fra personer som allerede har fått brukeren sin kompromittert. Informasjonen vil bestå av blant annet personens passordvaner, kjennskap til retningslinjer om passordbruk og epost-aktivitet. Vi håper å få minst 30 respondenter, som vil være akkurat nok for en kvalitativ analyse. 

\subsection{Gjennomføring}
Under idémyldringen ble det avdekket en rekke faktorer som kunne være medvirkende i at ansatte og studenter ved NTNU fikk sin konto på avveie. Vi brukte denne informasjonen aktivt da spørreundersøkelsen ble konstruert. Spørreundersøkelsen slik den fremstår for respondenten finnes i vedlegg \ref{undersokelse_norsk}. Med spørreundersøkelsen ble det sendt med en liten tekst for å oppmuntre de til å ta den. Her ble det brukt mye patos ettesom dette er et ømfintlig tema. For nesten hvert spørsmål ble en tilhørende hypotese bestemt. I tabell \ref{tab:hypoteser} under har vi listet opp hypotesene til de tilhørende spørsmålene.


% Table generated by Excel2LaTeX from sheet 'Ark2'
\begin{table}[H]
  \centering
  \caption{Hypoteser til spørsmålene}
    \begin{tabular}{|p{20.215em}|p{20.57em}|}
    \hline
    \rowcolor{yellow} Spørsmål & Hypoteser \\
    \hline
    Din alder? & Eldre er overrepresentert i statistikken over tapte kontoer \\
    \hline
    Ditt kjønn? & Flere kvinner enn menn har blitt kompromittert \\
    \hline
    Hva er din primærrolle ved NTNU? & Det er ansatte som er målgruppen til trusselaktørene \\
    \hline
    I hvilken by jobber/studerer du primært? & Gjøvik har høyere sikkerhetskompetanse \\
    \hline
    I hvor mange år har du jobbet/studert ved NTNU eller de tidligere høgskolene? & Folk med lavere ansiennitet kjenner universitetets retningslinjer bedre \\
    \hline
    Når fant du ut at NTNU kontoen din var blitt kompromittert? & Folk vet ikke at de har blitt kompromittert før de blir kontaktet \\
    \hline
    Har du noen formening om hvor lang tid kontoen var kompromittert før Seksjon for Digital Sikkerhet kontaktet deg? & Ingen hypotese grunnet vagt spørsmål \\
    \hline
    Har du noen formening om hvordan kontoen din ble kompromittert? & Ingen hypotese grunnet åpent svar \\
    \hline
    Bruker du din NTNU e-post til å registrere deg på ulike tjenester på nett i forbindelse med jobben/studiet? & Over halvparten bruker NTNU e-post til tjenester i forbindelse med jobb \\
    \hline
    Bruker du din NTNU e-post til å registrere deg på tjenester på nett til privat bruk? & Under halvparten bruker NTNU e-post til privat bruk \\
    \hline
    På en skala fra 1-6, der 1 er lite bevisst og 6 er svært bevisst, hvor bevisst er du på sikkerhet når du... & \multirow{4}[2]{*}{Folk er generelt sett lite bevisste} \\
    besøker nettsider? & \multicolumn{1}{r|}{} \\
    lager passord? & \multicolumn{1}{r|}{} \\
    sjekker e-post? & \multicolumn{1}{r|}{} \\
    \hline
    Har du i din tid hos NTNU lagt merke til phishing-forsøk mot deg på din NTNU e-post? & Phishing er en svært utbredt angrepsvektor \\
    \hline
    Har du blitt lurt av phishing på din NTNU e-post? & Av de som har blitt kompromittert har under en tredjedel blitt lurt av phishing \\
    \hline
    Har du i løpet av din tid ved NTNU eller de andre høgskolene, oppdaget virus eller annen skadevare på maskinen din? & Virus og annen skadevare er utbredt \\
    \hline
    Bruker du ditt NTNU passord på flere tjenester? & Passordgjenbruk er utbredt \\
    \hline
    Brukte du regler til å generere ditt NTNU passord? & De fleste bruker ikke passordregler til å generere passord \\
    \hline
    Er ditt NTNU passord tilfeldig sammensatt av bokstaver, tall og/eller tegn? & De fleste bruker ikke tilfeldig sammensatte passord \\
    \hline
    Hvor mange tegn består ditt NTNU passord av? & Over halvparten har passord på under 12 tegn \\
    \hline
    Har du i løpet av din tid ved NTNU delt NTNU passordet ditt med andre? & Passorddeling er ikke særlig utbredt \\
    \hline
    Omtrent hvor ofte bytter du ditt NTNU passord? & Passord byttes for det meste sjeldnere enn hvert andre år \\
    \hline
    Bruker du en passordmanager? & De fleste bruker ikke passordmanager, og mange vet ikke engang hva det er \\
    \hline
    På en skala fra 1 til 6, hvor godt kjent er du med... & \multirow{4}[2]{*}{Det er svært lite kjennskap til retningslinjer} \\
    NTNU sine retningslinjer for behandling av brukernavn, passord og andre autentiseringsdata? & \multicolumn{1}{r|}{} \\
    IT reglementet til NTNU? & \multicolumn{1}{r|}{} \\
    NTNU sine prinsipper for informasjonssikkerhet? & \multicolumn{1}{r|}{} \\
    \hline
    Har du fått opplæring i passordsikkerhet fra NTNU? & Under en fjerdedel har fått opplæring i passordskikk fra NTNU \\
    \hline
    \end{tabular}%
  \label{tab:hypoteser}%
\end{table}%

Undersøkelsen var kvalitetskontrollert flere ganger av forskjellige personer, inkludert medstudenter, veileder og ikke minst oppdragsgiver. Undersøkelsen ble laget i SelectSurvey, med tilhørende NTNU tema for utformingen. Dette ble gjort for å få spørreundersøkelsen til å virke legitim, siden den har NTNU sin logo i hjørnet og nettadressen tilhører NTNU sitt domene. 

Spørreundersøkelsen ble sendt ut til totalt 167 personer, men den nådde bare 157 av e-post addressene. Alle disse hadde fått sin NTNU konto kompromittert i tidsperioden 1. November 2016 til 1. April 2018. E-post listen ble opprettet av oppdragsgiver basert på intern data og sent ut på vegne av Seksjon for Digital Sikkerhet. 

Det ble oppdaget et par småfeil i spørreundersøkelsen etter den var utsendt. Blant annet var det glemt et ``vet ikke'' alternativ på spørsmålet om de hadde en formening om hvor lang tid det hadde gått fra kompromittering til de fikk beskjed. Dette fikk vi fikset ved å legge til tekst om at du kunne la det være blankt om du ikke visste, slik at det ikke gikk altfor mye ut over svarene, og spørsmålet ble endret til å ikke være obligatorisk. Det var også glemt en kommentarboks helt i slutten av spørreundersøkelsen, som vi bestemte at vi ikke kunne plassere inn etter den var utsendt. Det var også et par småfeil i formulering, men dette fikk vi endret underveis. Endringene ble gjort på natten da vi antok ingen svarte. 

\subsection{Resultater}
Av de som e-posten ble sendt ut til fikk vi totalt 72 gyldige respondenter, mens 26 stoppet rett etter åpning av undersøkelsen eller underveis. Undersøkelsen var aktiv i perioden fra 20. April til og med 27. April. I løpet av denne tiden ble det sendt ut én purring den 25. April. 

\subsection{Konklusjon av verktøy}
Dersom en feil er stor i en spørreundersøkelse kan ikke denne rettes opp i underveis. Dette er en av svakhetene med denne typen datainnsamling. Det kan i værste fall få konsekvenser for datasettet. Heldigvis var det ikke kritisk for oss, og vi fikk rettet opp det meste. Det positive var at vi hadde håpet på litt over 30 svar, men endte opp med 72, som var omtrent 45\% av de som mottok e-posten. Kanskje det var bruken av patos i e-post og innledning, i tillegg til at dette var et relevant tema for dem, som førte til høy svarprosent. 
