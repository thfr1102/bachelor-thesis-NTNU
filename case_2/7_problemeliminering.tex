\chapter{Rotårsakseliminering}
Denne fasen går ut på å finne tiltak som kan eliminere rotårsakene til kompromitterte kontoer ved NTNU. 

\section{Systematisk Innovativ Tenkning (SIT)}
Systematic Inventive Thinking inneholder fem hovedprinsipper:

\begin{enumerate}
    \item \textbf{Attributtavhengighet} Endre på en essensiell variabel.
    \item \textbf{Komponentkontroll} Ser på hvordan et produkt er tilknyttet omgivelsene sine.
    \item \textbf{Erstatning} Bytte ut en del av produktet med noe i omgivelsene til produktet.
    \item \textbf{Forkastning} Fjerne en del av produktet for å bedre det.
    \item \textbf{Oppdeling} Prøver å splitte et produkts attributter i to.
\end{enumerate}

\subsection{Ønsket utbytte}
Ved å bruke SIT metoden ønsker vi å få kreative idéer på hvordan vi kan finne en løsning til kompromitterte kontoer ved NTNU. 

\subsection{Gjennomføring}
SIT burde helst gjennomføres av 10-12 personer, fra en rekke forskjellige fagområder, men siden vi ikke hadde så mange tok vi bare utgangspunkt i prosjektgruppen. 
\subsubsection{Komponenter} 
Her blir alle komponenter som omhandler problemet listet.

\begin{itemize}
    \item E-postadresse
    \item Brukernavn
    \item Autentisering
    \item IT-reglement
    \item Retningslinjer for behandling av brukernavn, passord og andre autentiseringsdata
    \item Prinsipper for informasjonssikkerhet (inkludert Policy)
    \item Påloggingssystem
    \item E-post filter
\end{itemize}

Når komponentene er gjort rede for, vil de fem SIT prinsippene brukes sekvensielt på komponentene for å utvikle løsninger på problemene. Resultatene fremheves i neste seksjon. 


\subsection{Resultater}
Ikke alle SIT-prinsipper finner løsninger som er gjennomførbare for alle komponenter. I disse tilfellene vil det stå: ``Ikke gjennomførbart''. 

\paragraph{E-post}
\begin{itemize}
    \item \textbf{Attributtavhengighet} Ikke gjennomførbart.
    \item \textbf{Komponentkontroll} Bevisstgjøringskampanje for god e-postskikk.
    \item \textbf{Erstatning} Gi folk kurs og hjelp til å ordne private e-postadresser.
    \item \textbf{Forkastning} Ikke gjennomførbart.
    \item \textbf{Oppdeling} Gi folk en egen epost adresse som bare skal brukes til privat bruk.
\end{itemize}

\paragraph{Brukernavn}
\begin{itemize}
    \item \textbf{Attributtavhengighet} Ikke gjennomførbart.
    \item \textbf{Komponentkontroll} Ikke gjennomførbart.
    \item \textbf{Erstatning} Ha mer tilfeldig brukernavn som er vanskeligere å gjette.
    \item \textbf{Forkastning} Ikke la e-postadressen kunne brukes som brukernavn.
    \item \textbf{Oppdeling} Ikke gjennomførbart.
\end{itemize}

\paragraph{Autentisering}
\begin{itemize}
    \item \textbf{Attributtavhengighet} Krav om sterkere passord.
    \item \textbf{Komponentkontroll} Bruke passordmanager for å behandle passord. 
    \item \textbf{Erstatning} Ikke gjennomførbart.
    \item \textbf{Forkastning} Ikke gjennomførbart.
    \item \textbf{Oppdeling} Gå over til 2-faktor autentisering.
\end{itemize}

\paragraph{IT-reglement}
\begin{itemize}
    \item \textbf{Attributtavhengighet} Utbedre IT-reglementet med tydeligere krav rundt passord. 
    \item \textbf{Komponentkontroll} Henvise til de andre styringsdokumentene.
    \item \textbf{Erstatning} Ikke gjennomførbart.
    \item \textbf{Forkastning} Ikke gjennomførbart.
    \item \textbf{Oppdeling} Ikke gjennomførbart.
\end{itemize}

\paragraph{Retningslinjer for behandling av brukernavn, passord og andre autentiseringsdata}
\begin{itemize}
    \item \textbf{Attributtavhengighet} Ikke gjennomførbart.
    \item \textbf{Komponentkontroll} Bevisstgjøringskampanje.
    \item \textbf{Erstatning} Legge retningslinjene inn i IT-reglementet.
    \item \textbf{Forkastning} Ikke gjennomførbart.
    \item \textbf{Oppdeling} Ikke gjennomførbart.
\end{itemize}

\paragraph{Prinsipper for informasjonssikkerhet}
\begin{itemize}
    \item \textbf{Attributtavhengighet} Ikke gjennomførbart.
    \item \textbf{Komponentkontroll} Integrere det i et ISMS
    \item \textbf{Erstatning} Ikke gjennomførbart.
    \item \textbf{Forkastning} Ikke gjennomførbart.
    \item \textbf{Oppdeling} Ikke gjennomførbart.
\end{itemize}

\paragraph{Påloggingssystem}
\begin{itemize}
    \item \textbf{Attributtavhengighet} Øk minimum antall tegn på passord fra 8 til 10. 
    \item \textbf{Komponentkontroll} Overholde kravet om å bytte passord hver 12. måned.
    \item \textbf{Erstatning} Ikke gjennomførbart. 
    \item \textbf{Forkastning} Ikke gjennomførbart. 
    \item \textbf{Oppdeling} Enhetskontroll på nye innlogginger.
\end{itemize}

\paragraph{E-post filter}
\begin{itemize}
    \item \textbf{Attributtavhengighet} Forbedre e-post filter.
    \item \textbf{Komponentkontroll} Ikke gjennomførbart. 
    \item \textbf{Erstatning} Ikke gjennomførbart. 
    \item \textbf{Forkastning} Ikke gjennomførbart. 
    \item \textbf{Oppdeling} Ikke gjennomførbart. 
\end{itemize}

Vi sorterer og beskriver de mest relevante idéer til videre utdyping:

\begin{description}
\item[Bevisstgjøringskampanje for god e-postskikk]
Med god e-postskikk så mener vi at brukerene er bevisste på om e-post er legitim eller ikke, og at NTNU e-post ikke skal bli benyttet til andre tjenester.

\item[Gi folk kurs og hjelp til å ordne private e-postadresser]
Det er et problem at folk bruker sin NTNU e-post til andre tjenester. Dette kan være fordi de ikke har en egen privat e-post de kan bruke til dette. Dette tiltaket vil hjelpe de med å anskaffe en privat e-post, så de ikke bruker NTNU e-posten på andre ting en det den er ment for.

\item[Ikke la e-postadressen kunne brukes som brukernavn]
Dersom e-postadressen er brukt på andre tjenester med samme passord som NTNU, kan trusselaktørene også kompromittere NTNU kontoen. Hvis ikke e-postadressen lar seg bruke som brukernavn, vil det senke risikoen for at de får logget seg på. 

\item[Krav om sterkere passord]
Krav om sterkere passord i form av økt minimumslengde fra 8 til 10 tegn. 

\item[Overholde kravet om å bytte passord hver 12. måned]
I retningslinjene for behandling av autentiseringsdata er det krav om å bytte passord hver 12. måned. Dette blir ikke håndhevet. Et mulig tiltak er derfor å automatisk kreve endring av passord hver 12. måned. 

\item[Gå over til 2-faktor autentisering]
For å hindre at kontoen blir kompromittert dersom passordet ble det kan man benytte seg av 2-faktor autentisering. Dette skaper ekstra redundans. 

\item[Bruke passordmanager for å behandle passord]
Det blir lettere å behandle lange, kompliserte og unike passord med en passordmanager. 

\item[Utbedre IT-reglementet med tydeligere krav rundt passord]
Per nå er det eneste som står i IT-reglementet rundt passord at man skal bytte passord dersom man har mistanke om at noen vet det. Dette mener vi ikke er nok, og burde utbedres, for eksempel ved å referere til retningslinjer for behandling av autentiseringsdata. 

\item[Henvise til de andre styringsdokumentene]
Per nå er det lite henvisning til andre styringsdokumenter som gjør det vanskelig og tungvint for brukerene å lete igjennom dokumentene. Dette burde samles på ett sted og henvise til hverandre.

\item[Bevisstgjøringskampanje rundt autentiseringsdata]
Bevisstgjøringskampanjen skal få frem at passordet til NTNU kontoen skal ikke bli brukt til andre tjenester for å sikre at uvedkommende ikke får tilgang til kontoen.

\item[Integrere Prinsipper for informasjonssikkerhet i et ISMS]
Etter hva vi har fått av informasjon fra oppdragsgiver har ikke NTNU et skikkelig ISMS. Dette er noe som før eller siden bør være på plass. 

\item[Enhetskontroll på nye innlogginger]
En mulig måte å gjennomføre dette på er å sende en e-post eller sms om å autorisere enheten når det er første gang du logger på, på den enheten. Eventuelt validere den for 30 dager av gangen før dette må gjøres på nytt. 

\end{description}

\subsection{Tiltaksplan}
Etter å ha brukt de fem SIT-prinsippene på hver komponent, og filtrert de, sitter vi igjen med et par idéer. I denne delen fremhever vi idéer i en tiltaksplan som vi anbefaler å implementere. 
Under beskrives de ulike tiltakene:

\begin{description}
    \item[Bevisstgjøringskampanje for god e-postskikk og behandling av autentiseringsdata] Denne bevisstgjøringskampanjen vil inkludere opplæring i deteksjon av phishing e-post, beste praksis innen behandling av brukernavn, passord og andre autentiseringsdata, og innsikt i eksisterende dokumentasjon som NTNU har på informasjonssikkerhet. 
    \item[Krav om strengere passordkontroll] Dette tiltaket vil inkludere en økning av minimum passordlengde fra 8 til 10 tegn, og innføre en automatisk funksjon som pålegger deg å bytte passord hver 12. måned, slik det er krav om i retningslinjene. 
    \item[Implementer 2-faktor autentisering] Tiltaket går ut på å implementere 2-faktor autentisering for hver innlogging. Vi anbefaler å bruke sms, som gir deg en kode du kan logge inn med. 
    \item[Enhetskontroll og informering på nye innlogginger] Dette tiltaket går på å ha en enhetskontroll der ansvarlig bruker får sms når noen logger inn fra en ny enhet. Dersom dette var et legitimt påloggingsforsøk fra brukeren kan han/hun validere enheten for en gitt periode. Vi anbefaler 30 dager av gangen, men dette kan muligens også spesifiseres av brukeren. 
    \item[Utbedre IT-reglement til å inkludere og samle retningslinjer og krav] Dette tiltaket går på å utbedre passordkrav i tillegg til å henvise retningslinjer og andre styrende dokumenter. Dette burde samles på èn Innsida-side slik at det er lett å finne frem de nødvendige dokumentene til informasjonssikkerhet. 
    \item[Anbefale brukere å benytte passordmanager] Dette tiltaket går på å inkludere passordmanager som et anbefalt verktøy på samlesiden til IT-reglementet, retningslinjer og andre styrende dokumenter
\end{description}

\subsection{Konklusjon av verktøy}
På mange av komponentene var det vanskelig å finne tiltak og løsninger til hver av de fem hovedprinsippene. Dette førte til at det ble ganske mange steder vi måtte skrive ``Ikke gjennomførbart''. Enkelte av prinsippene kunne også være mer relevant for en spesifikk komponent som gjorde at det kunne være flere tiltak på ett prinsipp. Ellers fant vi mange gode tiltak som vil enten fjerne rotårsakene eller senke risikoen til at dette skjer. 