\chapter*{Kortfattet sammendrag}
I dette caset undersøkte vi rotårsaken til kompromitterte konotoer ved NTNU. I desember 2017 kom det også en stor datadump som innholdt over 5000 kontoer affiliert med NTNU, som alle inneholdt både brukernavn og passord. 101 av disse var fortsatt aktive, det vil si at brukernavn og passord var fortsatt gyldige. Universitetet betaler for tilgang til databaser som inneholder tusenvis av forskningsartikler og andre artikler. 26 av de kompromitterte brukerene ble brukt av aktører for å skaffe seg tilgang til forskningartikler. Konsekvensen ved å ha kompromitterte kontoer som laster ned forskningsartikler er ikke bare at NTNU taper penger, men at NTNU risikerer å bli blokkert fra databasene. 

I dette prosjektet har vi benyttes oss av metodene og verktøyene som boken om rotårsaksanalyse \cite{RCA} tar for seg.

Vi hadde en hypotese om at phishing og gjenbruk av passord er de største årsakene til kompromittert konto. Dette viste seg å være korrekt, i tillegg til dette viste det seg også at dårlig kjennskap til styrende dokumenter og at tilgangskontroll på kontoene var dårlig. 