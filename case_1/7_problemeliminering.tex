\chapter{Problemeliminering}
Arbeidet i denne fasen går ut på å finne løsninger for å eliminere rotårsaken. I denne fasen har vi valgt å bruke vektøyet seks tenke hatter for å finne de beste måtene å få eliminert rotårsaken på.

\section{De seks tenkehattene}
De seks tenkehattene er et diskusjons verktøy, som går ut på at man gir personer forskjellige hatter/roller til prossessen. De forskjellige rollene er som følger:
\begin{description}
    \item [Hvit hatt] kynisk, faktabasert og systematisk.
    \item [Rød hatt] er den emosjonelle personen, han som følger magefølelsen og egen intuisjon
    \item[Svart hatt] pessimistisk og negativ, fokuserer på hvordan idéen ikke vil
    fungere
    \item [Gul hatt] optimistisk og positiv, fokuserer på hva som skal til for at løsningen skal kunne fungere
    \item[Grønn hatt] fokuserer på kreativitet og skal prøve å bygge på idéer
    \item[Blå hatt] assosieres med himmelen, og skal se problemet fra et større perspektiv.
\end{description}

\subsection{Ønsket utbytte}
Ønsket utbytte med bruk av seks tenkehatter, er å skape en forståelse rundt rotårsaken og komme opp meg mulige tiltak for å eliminere problemet. Siden problemet virker vanskelig å fikse fra skolen sin side måtte vi komme med noen kreative løsninger, og de seks tenkehattene fungerer godt for dette.

\subsection{Gjennomførelse}
Siden vi bare var fire tok to av oss på seg to hatter og resten en. Så startet vi å diskutere problemstillingen og hvordan vi burde gå inn for å eleminere rotårsaken, vi kom frem til en rekke mulige løsninger. Etter at vi var ferdig med de seks tenkehattene gikk vi fort igjennom de forskjellige forslagene, for å se på hva som var praktisk gjennomførbare og ikke, på slutten av prosessen tok vi altså å brukte de to siste punktene fra SIT, for å luke ut de beste forslagene, og for å bestemme oss for hva som var gjennomførbart

\subsection{Resultater}
Ut i fra prosessen med de seks tenkehattene kom vi fram til både gjennomførbare og ikke gjennomførbare tiltak. 
\subsubsection*{Gjennomførbare tiltak}
\begin{itemize}
    \item Se på de forskjellige brevene fra rettighetshaverene og se om det er noen av dem som er mer representert enn andre, og se om det er noen mulighet for å tilby filmer og serier fra disse selskapene.
    \item Stenge torrentprotokoellen for alle på nettverket.
    \item Grense på nedlastning og opplastning av data.
    \item Være strengere når det gjelder oppfølging av IT-reglementet. 
    \item Bytte ISP til studentboligene.
    \item Oppmerksomhetskampanje om konsekvenser.
    \item Avtale med kino for billige/gratis nye filmer
%%%%%%%%%%%%%%%%%%%%%%%%%%%%%%%%%%%%%%%%%%%%%%%%%%%%%%%%%%%%%%%%%%%%%%%%%%%%%%%%%%%%%%%%%%%%
    \item Stenge nettet i 5 min for alle om noen starter å laste ned, kollektiv straff. %OBS OBS BURDE FJERNES!!!!!!!!!!!!!!!! FRA HOVEDRAPPORTEN
\end{itemize}

\subsubsection*{Ikke gjennomførbare tiltak}
\begin{itemize}
    \item Alt av materiale blir gratis og tilgjengelig på ett samlet sted
    \item Fjerne geografiske blokkeringer
\end{itemize}

Noen av disse vil ikke være gjennomførbare for NTNU så vi har valgt å ikke ta de med videre til implementering, men vi følte de kunne være interessante å nevne. Av de 11 forslagene er det kun 4-5 vi mener har høy sannsynlighet for å bli kvitt rotårsaken, helt eller delvis, eller flytter rotårsaken vekk fra NTNU ansvarsområde. Etter videre vurdering har vi kommet fram til de mest lovende løsningene for å fjerne rotårsaken: 
\subsubsection*{Disse fjerner rotårsaken til at folk laster ned}
\begin{enumerate}
    \item Alt av materiale blir gratis og tilgjengelig på ett samlet sted
    \item Fjerne geografiske blokkeringer
\end{enumerate}
\subsubsection*{Disse fjerner rotårsaken til at skolen får notifikasjon fra opphavsrettshaverne}
\begin{enumerate}
    \item Bytte ISP til studentboligene
    \item Stenge torrentprotokollen for alle på nettverket
\end{enumerate}
Rotårsaken til at folk bedriver ulovlig fildeling er ikke et problem som lett kan løses av skolen. Vi har gitt et par forslag til hva som kan fjerne rotårsaken helt, men dette er ikke gjennomførbart. Vi har også foreslått et par tiltak som ikke nødvendigvis fjerner rotårsaken for hvorfor folk laster ned, men flytter rotårsaken bort fra NTNU's ansvarsområde.
%%%%%%%%%%%%%%%%% FJERNE FFFFFFFFFFFFFFFFFFFFFFFFFFFFFF
En annen løsning som kunne vært et stilig tankeprosjekt, er å stenge nettet i 5 min for hver gang NTNU får en notifikasjon på brudd på opphavsretten. 
%%%%%%%%%%%%%%%% FJERNE FFFFFFFFFFFFFFFFFFFFFFFFFFFFFF

\subsection{Konklusjon av verktøy}
Verktøyet fungerte godt til å starte en diskusjon rundt elimineringsalternativer, men det utviklet seg fort til en form for idémyldring, siden det var vanskelig å forholde oss til de forskjellige sinnstilstandene/hattene. Det var også et par komplikasjoner siden to av oss måtte ha to hatter samtidig. Ellers var det en kreativ og morsom prosess der mange gode alternativer kom fram. 
