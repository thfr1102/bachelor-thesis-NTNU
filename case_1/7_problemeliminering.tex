\chapter{Rotårsakseliminering}
Arbeidet i denne fasen går ut på å finne løsninger for å eliminere rotårsaken. I denne fasen har vi valgt å bruke vektøyet seks tenke hatter og SIT for å finne de beste måtene å få eliminert rotårsaken på. 

\section{De seks tenkehattene}
De seks tenkehattene er et diskusjons verktøy, som går ut på at man gir personer forskjellige hatter/roller til prossessen. De forskjellige rollene er som følger:
\begin{description}
    \item [Hvit hatt] kynisk, faktabasert og systematisk.
    \item [Rød hatt] er den emosjonelle personen, han som følger magefølelsen og egen intuisjon
    \item[Svart hatt] pessimistisk og negativ, fokuserer på hvordan idéen ikke vil
    fungere
    \item [Gul hatt] optimistisk og positiv, fokuserer på hva som skal til for at løsningen skal kunne fungere
    \item[Grønn hatt] fokuserer på kreativitet og skal prøve å bygge på idéer
    \item[Blå hatt] assosieres med himmelen, og skal se problemet fra et større perspektiv.
\end{description}

\subsection{Ønsket utbytte}
Ønsket utbytte med bruk av seks tenkehatter, er å skape en forståelse rundt rotårsaken og komme opp meg mulige tiltak for å eliminere problemet. Siden problemet virker vanskelig å fikse fra skolen sin side måtte vi komme med noen kreative løsninger, og de seks tenkehattene fungerer godt for dette.

\subsection{Gjennomføring}
Siden vi bare var fire tok to av oss på seg to hatter og resten en. Så startet vi å diskutere problemstillingen og hvordan vi burde gå inn for å eleminere rotårsaken, vi kom frem til en rekke mulige løsninger. Etter at vi var ferdig med de seks tenkehattene gikk vi fort igjennom de forskjellige forslagene, for å se på hva som var praktisk gjennomførbare og ikke, på slutten av prosessen tok vi altså å brukte de to siste punktene fra SIT, for å luke ut de beste forslagene, og for å bestemme oss for hva som var gjennomførbart

\subsection{Resultater}
Ut i fra prosessen med de seks tenkehattene kom vi fram til både gjennomførbare og ikke gjennomførbare tiltak. 
\subsubsection{Gjennomførbare tiltak}
\begin{itemize}
    \item Tilby produkter fra selskap som skolen får flest notifikasjoner fra
    \item Stenge torrentprotokollen for alle på nettverket.
    \item Grense på nedlastning og opplastning av data.
    \item Være strengere når det gjelder oppfølging av IT-reglementet. 
    \item Bytte ISP til studentboligene.
    \item Oppmerksomhetskampanje om konsekvenser.
    \item Avtale med kino for billige/gratis nye filmer
\end{itemize}

\subsubsection{Ikke gjennomførbare tiltak}
\begin{itemize}
    \item Alt av materiale blir gratis og tilgjengelig på ett samlet sted
    \item Fjerne geografiske blokkeringer
\end{itemize}

Noen av disse vil ikke være gjennomførbare for NTNU så vi har valgt å ikke ta de med videre til implementering, men vi følte de kunne være interessante å nevne. Av de 11 forslagene er det kun 4-5 vi mener har høy sannsynlighet for å bli kvitt rotårsaken, helt eller delvis, eller flytter rotårsaken vekk fra NTNU ansvarsområde. Etter videre vurdering har vi kommet fram til de mest lovende løsningene for å fjerne rotårsaken: 

\subsubsection*{Tiltak som fjerner rotårsaken til at folk laster ned}

\begin{enumerate}
    \item Alt av materiale blir gratis og tilgjengelig på ett samlet sted
    \item Fjerne geografiske blokkeringer
    \item Tilby produkter fra selskap som skolen får mest notis fra
    \item Prøve for studenter for å få full hastighet på nettverk
\end{enumerate}

\subsubsection*{Tiltak som fjerner rotårsaken til at skolen får notifikasjon fra opphavsrettshaverne}

\begin{enumerate}
    \item Bytte ISP til studentboligene
    \item Stenge torrentprotokollen for alle på nettverket
\end{enumerate}

Rotårsaken til at folk bedriver ulovlig fildeling er ikke et problem som lett kan løses av skolen. Vi har gitt et par forslag til hva som kan fjerne rotårsaken helt, men av disse er ikke alle gjennomførbare for skolen. Vi har også foreslått et par tiltak som ikke fjerner rotårsaken og noen som fjerner rotårsaken til hvorfor folk laster ned. De tiltaken som ikke fjerner rotårsaken vil flytte rotårsaken bort fra NTNU's ansvarsområde.

\subsection{Konklusjon av verktøy}
Verktøyet fungerte godt til å starte en diskusjon rundt elimineringsalternativer, men det utviklet seg fort til en form for idémyldring, siden det var vanskelig å forholde oss til de forskjellige sinnstilstandene/hattene. Det var også et par komplikasjoner siden to av oss måtte ha to hatter samtidig. Ellers var det en kreativ og morsom prosess der mange gode alternativer kom fram. 

\section{Systematisk Innovativ Tenkning (SIT)}
Systematic Inventive Thinking inneholder fem hovedprinsipper:

\begin{enumerate}
    \item \textbf{Attributtavhengighet} Endre på en essensiell variabel.
    \item \textbf{Komponentkontroll} Ser på hvordan et produkt er tilknyttet omgivelsene sine.
    \item \textbf{Erstatning} Bytte ut en del av produktet med noe i omgivelsene til produktet.
    \item \textbf{Forkastning} Fjerne en del av produktet for å bedre det.
    \item \textbf{Oppdeling} Prøver å splitte et produkts attributter i to.
\end{enumerate}

\subsection{Ønsket utbytte}
Ved å bruke SIT metoden ønsker vi å få kreative løsninger på hvordan vi kan stoppe ulovlig nedlasting.

\subsection{Gjennomføring}

\subsubsection{Komponenter} Alle komponenter blir listet.

\begin{itemize}
    \item Lite kunnskap om konsekvenser
    \item Raskt internett
    \item Alt er tilgjengelig samtidig overalt
    \item Økonomi
  %  \item Kjedsomhet - laste ned serier/spill for å ha noe å gjøre
\end{itemize}

\subsubsection{Idémyldring med SIT hovedprinsipper} I dette steget ble SIT prinsippene brukt til å finne løsninger til komponentene i det foregående steget.

\paragraph{Lite kunnskap om konsekvenser}
\begin{itemize}
    \item \textbf{Attributtavhengighet} Løfte opp kunnskapen om konsekvenser og lover.
    \item \textbf{Komponentkontroll} Ha en test for å se at folk forstår IT-reglementet og mulig kursing.
    \item \textbf{Erstatning} Ikke gjennomførbart
    \item \textbf{Forkastning} Ikke gjennomførbart
    \item \textbf{Oppdeling} Ikke gjennomførbart
\end{itemize}

\paragraph{Raskt internett som skolen eier}
\begin{itemize}
    \item \textbf{Attributtavhengighet} Sperre torrenting protokollen.
    \item \textbf{Komponentkontroll} Ikke gjennomførbart.
    \item \textbf{Erstatning} Ha tregere internett, slik at folk ikke laster ned hjemme.
    \item \textbf{Forkastning} Fjerne internett fra Sit boligene/bytte ISP.
    \item \textbf{Oppdeling} Ikke gjennomførbart
\end{itemize}

\paragraph{Tilgjengelighet}
\begin{itemize}
    \item \textbf{Attributtavhengighet} Få rettighethaverne til å samle alt innhold til ett sted 
    \item \textbf{Komponentkontroll} Ikke gjennomførbart.
    \item \textbf{Erstatning} Tilby gratis filmer/serier og spill.
    \item \textbf{Forkastning} Ikke gjennomførbart
    \item \textbf{Oppdeling} Ikke gjennomførbart
\end{itemize}

\paragraph{Økonomi}
\begin{itemize}
    \item \textbf{Attributtavhengighet} Mer penger fra lånekassen. 
    \item \textbf{Komponentkontroll} Skolen tilbyr det som blir lastet mest ned
    \item \textbf{Erstatning} Skolen gjør innhold gratis
    \item \textbf{Forkastning} Ikke gjennomførbart
    \item \textbf{Oppdeling} Ikke gjennomførbart
\end{itemize}

%\paragraph{Kjedsomhet}
%\begin{itemize}
    %\item \textbf{Attributtavhengighet} Skolen har flere fritidsgrupper (kanskje skole finansiert bokklubb)
    %\item \textbf{Komponentkontroll} Skolen har mer fokus på å vise frem fritidsgrupper.
   % \item \textbf{Erstatning} Nye aktiviteter
  %  \item \textbf{Forkastning} Ikke gjennomførbart
 %   \item \textbf{Oppdeling} Ikke gjennomførbart
%\end{itemize}

%\paragraph{}
%\begin{itemize}
%    \item \textbf{Attributtavhengighet}
%    \item \textbf{Komponentkontroll}
%    \item \textbf{Erstatning}
%    \item \textbf{Forkastning}
%    \item \textbf{Oppdeling}
%\end{itemize}



\subsection{Resultater}
Løsningene sorteres og diskuteres ut fra hvilke som best lar seg gjennomføre. 
\begin{description}
    \item[Lite kunnskap om konsekvenser] Løfte opp kunnskapen om konsekvenser og lover.  Måten vi kan gjøre dette på er å ha et kurs og dette kurset kan ha en påfølgende quiz.
    \item[Raskt internett som skolen eier] Det å sperre torrenting protokollen er nok ikke gjennomførbart fordi skolen er eneste internettleverandør. Det å bytte ISP ved studentboligene er mulig, men vil kreve videre utredning.
    \item[Tilgjengelighet] Det at alt materialet blir samlet på ett sted vil være en krevende prosess.
    \item[Økonomi] Skolen sjekker hva som blir lastet ned mest og prøver å få til en avtale med rettighetshaverne om en billig god måte å få tilbudt dette til studentene.
    %\item[Kjedsomhet] Ha en bedre informasjonskanal enn facebook for å få delt ut informasjon om de ulike aktivitetene på skolen.
\end{description}




\subsubsection{Løsningsgenerering} Man diskutere videre de mest lovende ideene samtidig som man lager planer for hvordan man gjennomfører ideene.

Vi tror de mest lovende idéene er:
\begin{description}
    \item [Kurs] Skolen tilbyr et kurs, gjerne ett nettkurs, med en tilhørende prøve for å se hvor mye folk får med seg.
    \item [Bytte ISP] Bytter Sit ISP vil ansvaret for notisene flyttes fra universitetet til den nye ISP. 
    \item [Tilby produkter] Her kan skolen se gjennom de ulike notisene de har fått, og finne ut hvilket selskap som blir lastet ned mest, for så å prøve å tilby deres filmer/serier.
    \item [Alt materialet tilgjengelig] Om alt materialet er tilgjengelig på ett sted vil dette fjerne problemet med at eneste/enkleste måten å få tak i produkter er ved å laste dem ned.
\end{description}


\subsection{Konklusjon av verktøy}
Som et verktøy for å finne løsninger følte vi at den ikke ga noe mer enn det vi alt hadde fått fra de seks tenkehattene. I den forstand at vi kom på løsningene alt på tenkehattene og diskuterte dem godt i den prosessen. Vi følte noen av prinsippene som skulle brukes i SIT prosessen ikke ga skikkelig mening i forhold til oppgaven, der det var lite som kunne forkastes eller splittes.