\chapter{Løsningsimplementering}
Arbeidet i denne fasen går ut på å implementere løsningen til rotårsaken. I den foregående fasen ble løsningene til rotårsaken identifisert til å være bytte av ISP på SIT-boligene eller å slå av torrenting protokollen. Begge disse løsningen vil stoppe NTNU fra å motta notifikasjoner for brudd på opphavsrett, men vil ikke stoppe beboere fra å laste ned. Implementering av disse løsningene krever ikke en implementeringssplan fra vår side, så derfor har vi ikke et behov for å bruke noen verktøy her.

\section{Gjennomførelse}
Gjennomførelse av disse forslagene vil i hovedsak administreres av skolen og SIT.

\subsection{SIT bytter ISP}
Det å flytte ISPen over til SIT, eller splitte studenthjemmene fra skolenettverket, vil være en dyr og tidkrevende prosess, men dette kan være verdt å gjennomføre. 

\subsection{NTNU stenger torrent-protokollen for hele skolenettet}
Blokkering av torrent-protokollen vil kunne skade legitim bruk av torrents. Det må på forhånd gjøres en utredning for å undersøke konsekvensene av dette, siden det må implementeres for hele nettverket, inkludert på campus. 