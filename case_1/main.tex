%% This document gives an example on how to use the ntnubachelorthesis
%% LaTeX document class.
%% Use oneside for PDF delivery and twoside for printing in a book style
%% use language english, norsk, nynorsk and one of the following shortenings
%%  ``BSP'' Bachelor i Spillprogrammering,\\
%%  ``BRD'' Bachelor i drift av nettverk og datasystemer,\\
%%  ``BIS'' Bachelor i Informasjonssikkerhet,\\
%%  ``BPU'' Bachelor i Programvareutvikling, \\
%%  ``BIND'' Bachelor i Ingeniorfad - data, \\
%%  ``BADR'' Bachelor i drift av datasystemer, \\
%%  ``BIT'' Bachelor i informatikk, \\
%%  ``BABED'' Bachelor i IT-støttet bedriftsutvikling.
%%   for example \documentclass[BIS,norsk,twoside]{ntnuthesis/ntnubachelorthesis}

\documentclass[BIS,norsk,oneside]{ntnuthesis/ntnubachelorthesis}
\thesistitle{Case 1: Ulovlig fildeling på universitetsnettet til NTNU}
\thesisshorttitle{Case 1: Ulovlig fildeling på universitetsnettet til NTNU}
\thesisauthor{Thomas Huse, Philip Nyblom, Ole Martin Søgnen, Fredrik Løvaas Theien}
\thesisdate{16.05.2018}

\usepackage{csvsimple}
\usepackage{booktabs}
\usepackage{gnuplottex}
\usepackage[T1]{fontenc}
\usepackage[utf8]{inputenc}     % For utf8 encoded .tex files because...
\usepackage[norsk]{babel}       % For Norwegian labeling
\usepackage{graphicx}           % For inclusion of graphics
\PassOptionsToPackage{hyphens}{url}
\usepackage{url}
\usepackage{hyperref}    % For cross references in pdf
\usepackage{tabularx}
\usepackage[table]{xcolor}
\usepackage{colortbl}
\usepackage{placeins}
\usepackage{pdfpages}
\usepackage{enumitem}
\usepackage{multirow}
\usepackage{lscape}
\usepackage{bigstrut}
\usepackage{verbatim}
\usepackage{float}


\definecolor{darkgreen}{rgb}{0,0.5,0}
\definecolor{apricot}{RGB}{255, 217, 178}

\lstset{        basicstyle=\ttfamily,
                keywordstyle=\color{blue}\ttfamily,
                stringstyle=\color{darkred}\ttfamily,
                commentstyle=\color{darkgreen}\ttfamily,
}


%Typesetting of C++
\newcommand{\CPP}[0]{{C\nolinebreak[4]\hspace{-.1em}\raisebox{.1ex}{\small\bf +\hspace{-.1em}+\ }}}



%\newcommand{\comment}[1]{\textcolor{blue}{\emph{#1}}}  %% use of the colour and you can see how to use commands with parts \comment{so what}

%% The class files defines these two
%% \newcommand{\NTNU}{Norwegian University for Science and Technology} %

% you can create you one #define like structures using the \newcommand feature
% you can change behaviour using \renewcommand

\newcommand{\com}[1]{{\color{red}#1}} % supervisor comment
%\renewcommand{\com}[1]{} %remove starting % to remove supervisor comments
% This will appear in text \com{Lecuters comment} and be visible unless you uncomment
% the renewcommand line.

\newcommand{\todo}[1]{{\color{green}#1}} % items to do
%\renewcommand{\todo}[1]{} %remove starting % to remove items to do

\newcommand{\n}[1]{{\color{blue}#1}} % other comment
%\renewcommand{\n}[1]{} %remove starting % to remove notes

\newcommand{\dn}[1]{} % add the d to a note to say that you have finished with it.

\newcommand{\gj}{NTNU i Gj\o{}vik}


% Norwegian Characters,  needs the {} or to be separate from the next letters
% \o{}   \aa{}   \ae{}   so at the end of a word you can use \o  \aa   \ae
% \O{}   \AA{}   \AE{}   you can also just leave a space and latex will remove it
%    eg, NTNU i Gj\o vik  or NTNU i Gj\o{}vik

\graphicspath{ {bilder/} }

\begin{document}

\thesistitlepage

\tableofcontents
\addtocontents{toc}{\protect\setcounter{tocdepth}{1}}
\listoffigures
\listoftables
%\lstlistoflistings

\chapter*{Kortfattet sammendrag}
I dette caset undersøkte vi rotårsaken til at NTNU sin infrastruktur blir misbrukt til utvinning av kryptovaluta. Utvinning av kryptovaluta har blitt en økende trend det siste året, der det har kommet frem flere historier i media. Dette gjør at flere vil prøve seg og de bruker da universitetet sin infrastruktur som ikke skal bli brukt til kommersiell virksomhet. Kriminelle vil heller ikke gå glipp av denne gyldne muligheten og har sine script de bruker. 

I dette prosjektet har vi benyttes oss av metodene og verktøyene som boken om rotårsaksanalyse \cite{RCA} tar for seg.

Vi hadde en hypotese om... Dette viste seg å være... Ut ifra intervjuet vi gjennomførte og analysen av dette, kom vi frem til at... ... var den største faktoren som gjorde at NTNU sin infrastruktur ble misbrukt til utvinning av kryptovaluta.
\chapter{Introduksjon}
%--------------------MÅ INN I HOVEDRAPPORT------------------------------------------
Rotårsaksanalyser er et lite brukt verktøy innen informasjonssikkerhet, men er av økende betydning. Vanlig tilnærming til informasjonssikkerhetsstyring er å utføre en risiko- og sårbarhetsanalyse (ROS-analyse) for så å gjennomføre tiltak som fører risikoene til et akseptabelt nivå. En annen hyppig brukt tilnærming er hendelseshåndtering der en planlegger hvordan det skal responderes på hendelser etter de er inntruffet. Rotårsaksanalyse skiller seg fra disse ved å gå i dybden på problemet, kartlegge hva slags rotårsaker som står bak, og innføre tiltak for å fjerne disse helt.
%%%%%%%%%%%%%%%%%%%%%%%%%%%%%%%%%%%%%%%%%%%%%%%%%%%%%%%%%%%%%%%%%%%%%%%%%%%%%%%%%%%%%

\section{Oppgavebeskrivelse}
Denne rapporten er en delrapport i en større oppgave om rotårsaksanalyse. Dette caset går inn på rotårsaken til misbruk av NTNU sine ressurser og infrastruktur til å utvinne kryptovaluta. De to siste årene har både verdien og antallet kryptovaluta økt drastisk. Det finnes per dags dato over 1500 forskjellige kryptovalutaer. Kryptovaluta blir "minet", eller utvinnet, ved bruk av regnekraft. Dette betyr at enhver datamaskin kan delta i utvinningen. Siden november 2017 har NTNU sett en økning i mining med 8000\% og får i dag flere varsler angående mining om dagen.  Etterhvert vil vanskelighetsgraden for å utvinne nye mynter øke. Når vanskelighetsgraden øker trenger en mer datakraft og større maskinrigger til å utvinne valutaene. 

NTNU forvalter stor regnekraft spredt på flere lokasjoner. NTNU har også hatt supermaskiner før, de har en nå og de får nå en ny supermaskin. Supermaskiner er store datamaskiner med enorm datakraft. Disse er spesielt attraktive for aktører å misbruke til å utvinne kryptovaluta. Siden trenden har økt de siste årene, og NTNU er i besittelse av mye regnekraft, må NTNU aktivt jobbe for å beskytte infrastrukturen. 

Siden dette er av økende trend, og Seksjon for Digital Sikkerhet har oppdaget at noe av universitetet sine ressurser har blitt brukt til utvinning av kryptovaluta, vil de undersøke måter å eliminere dette misbruket. 

Denne analysen går ut på å identifisere rotårsaken til misbruk av NTNU sine ressurser til utvinning av kryptovaluta, og foreslå tiltak for å eliminere den. I løpet av rapporten ønsker vi å svare på følgende forskningsspørsmål:

\begin{itemize}
    \item Hva er rotårsaken til at NTNU sin infrastruktur blir misbrukt til utvinning av kryptovaluta?
    \item Hvor godt fungerer rotårsaksanalyse i et case som omhandler misbruk av IT-infrastruktur?
\end{itemize}

I denne analysen definerer vi misbruk som all bruk av NTNU sin infrastruktur til kommersiell vinning. Personlig vinning er noe NTNU som en offentlig institusjon ikke har lov til å finansiere [KILDE!]. Når vi ser på misbruket så skiller vi mellom frivillig og ufrivillig misbruk. Med frivillig misbruk mener vi når noen med vilje misbruker universitet sine ressurser til personlig vinning. Med ufrivillig misbruk mener vi noen som utnytter interne brukere for å få tilgang til NTNU sin infrastruktur, for så å misbruke ressursene. Herunder regner vi alt fra mining der bruker besøker en nettside til personer som hacker seg inn i infrastrukturen.



I løpet av rapporten kommer vi også til å referere til ressursene og infrastrukturen som aktiva. 
\section{Metode}
Metodebruken i denne analysen er delt inn i syv steg som vist i \hyperref[fig:prosess]{Figur 1} under. I hvert steg av denne prosessen brukes det ulike verktøy for å hjelpe til med å forstå problemet, finne rotårsak, og tilslutt implementere tiltak for å eliminere årsakene. 
\begin{figure}[H]
    \centering
    \includegraphics[scale=0.6]{case_1/bilder/prosess.pdf}
    \label{fig:prosess}
    \caption[Rotårsaksanalyseprosessen]{Rotårsaksanalyseprosessen definert av Andersen og Fagerli}
\end{figure}
\chapter{Problemforståelse}
Denne fasen eksisterer for å passe på at en har forstått problemet i dypere detalj. Verktøyene som er relevante å bruke skal gi en bedre forståelse av blant annet omfanget og de ulike aspektene ved et problem. Jo bedre tilgang en har på informasjon, logger og dokumentasjon, jo bedre vil denne fasen kunne utføres. 


\section{Ytelsesmatrise}
Ytelsesmatrise er et diagram som tar i betraktning den nåværende ytelsen til en variabel. Dette kan bety flere ting og vurderes i ulike former, men i dette caset definerer vi ytelse som hvor godt de ulike variablene fungerer i dag. Det som gjør ytelsesmatrise så nyttig er at den også vurderer viktigheten, slik at en kan vurdere hvilken prioritering variablene som blir analysert har. I dette caset blir viktighet definert som hvor funksjonskritisk ressursene er for arbeidet som gjøres ved NTNU.

\subsection{Ønsket utbytte}
Ved bruk av dette verktøyet var det ønskelig å undersøke hvordan eksisterende kontroller fungerer i forhold til deres viktighet for NTNU. 

\subsection{Gjennomføring}
Prosessen startet ved å finne ut hvilke aspekter av problemet som skulle vurderes. Gruppen kom fram til å vurdere formene for kontroller som stopper eller reduserer sjansen for at trusselaktørene misbruker NTNU sin infrastruktur. 

Disse skulle vurderes basert på viktighet og ytelse. 

Matrisen ble tegnet opp i Excel der hver akse ble konstruert fra en til ni, og matrisen ble delt inn i fire områder:
\begin{description}
    \item[Uviktig:] Når både viktigheten og ytelsen er fra en til fem.
    \item[Overdrevent:] Når viktigheten er fra en til fem og ytelsen er fra fem til ni.
    \item[Ok:] Når både viktigheten og ytelsen er fra fem til ni.
    \item[Må forbedres:] Når viktigheten er fra fem til ni, mens ytelsen bare er fra en til fem.
\end{description}

\subsection{Resultater}
Variablene som ble vurdert til å kunne hjelpe til å redusere utvinning av kryptovaluta hos NTNU er som følger:
\begin{description}
    \item[Adgangskontroll på HPC klynger:] Klynger av tilkoblet maskinvare som sammen utgir svært høy ytelse. Er også kjent som superdatamaskiner. Disse er godt beskyttet med streng adgangskontroll og logging av alt som blir gjort.
    \item[Adgangskontroll på kritiske servere:] Adgangskontroll til servere som har en funksjonskritisk og/eller virksomhetskritisk rolle i driften av NTNU, som for eksempel DNS og DHCP servere. 
    \item[Adgangskontroll på andre servere:] Adgangskontroll til alle servere som ikke har en kritisk rolle i NTNU, men som fortsatt kan bli misbrukt. Inkluderer servere som står åpent ut mot nettet. 
    \item[Beskyttelse mot ufrivillig utvinning på datamaskiner:] Personer får tilgang til din datamaskin gjennom nettleseren og bruker den til å utvinne kryptovaluta. 
    \item[Policy på hva som er akseptabelt som BYOD:] Definerer hva som er lov å ta med av BYOD.
    \item[IT-reglement på krypto utvinning:] IT-reglementet spesifiserer per idag bare at det å bruke universitetets ressurser til kommersiell virksomhet ikke er greit. Det kan være vanskelig for folk å ta koblingen til at strøm er en slik ressurs og at utvinning av kryptovaluta kan regnes som kommersiell virksomhet. Sånn som IT-reglementet er idag er det heller ikke noen gode sanksjonsmuligheter mot folk som utvinner kryptovaluta.

\end{description}

Under ser vi hvor de ulike ressursene, eller aktiva, er plassert i henhold til de tidligere nevnte områdene.
\begin{figure}[H]
    \centering
    \includegraphics[scale=0.5]{case_3/bilder/ytelsesmatrise.pdf}
    \caption[Ytelsesmatrise]{Resultater fra ytelsesmatrisen}
    \label{fig:ytelsesmatrise}
\end{figure}

Det er mest kritisk å vurdere de variablene som havner under ``må forbedres''. Selv om noen havner under ``ok'', bør de fortsatt vurderes, men de vil ha lavere prioritet enn de nevnt over. Variabler som er uviktig eller overdrevent trenger man ikke vurdere nøye. Matrisen viser følgende prioriteringsgrunnlag til utbedring:

\begin{enumerate}
    \item IT-reglement på krypto utvinning
    \item Policy på hva som er greit som BYOD (IT-regement på BYOD som driver med utvinning)
    \item Beskyttelse mot ufrivillig utvinning på datamaskiner
    \item Adgangskontroll på andre servere
    \item Adgangskontroll på kritiske servere
    \item Adgangskontroll på HPC klynger
\end{enumerate}

\subsection{Konklusjon av verktøyet}
Verktøyet var nyttig for å finne frem til et prioritetsgrunnlag for de ulike enhetene som kan misbrukes eller misbruke NTNU sine ressurser. Vi ser at det er IT-reglementets mangel på spesifisering av kryptoutvinning som er det største problemet. 
\chapter{Idémyldring}
I dette steget i prosessen er målet å generere en liste over det vi tror kan være mulige årsaker til problemet. Det er en del forskjellige verktøy en kan bruke for å oppnå dette, men vi har valgt å benytte Idémyldring på basis av RCA boken \cite{RCA} sin fremgangsmåte for valg av verktøy, og på bakgrunn av vår tidligere kunnskap om hvordan brukere vanligvis kompromitteres. 

\section{Idémyldring}
I rotårsaksanalyse finnes det to ulike måter å gjennomføre idémyldring på: strukturert- og ustrukturert idémyldring. I den strukturerte versjonen får hver deltaker sin tur til å komme med en idé, og dette sikrer at alle får delta like mye. På den ustrukturerte måten kan alle komme med idéer etterhvert som de kommer på dem, og fungerer mye mer spontant enn den strukturelle. Det er spesielt viktig å ikke omformulere eller diskutere forslagene etterhvert som de kommer, dette skal gjøres etter idémyldringsøkten er over.

\subsection{Ønsket utbytte}
Ønsket utbytte ved å bruke idémyldring var for å få en forståelse av hva som kan være rotårsaken til at ansatte sine kontoer blir kompromittert, og hvordan passord og brukernavn kan komme på avveie.

\subsection{Gjennomførelse}
Det første som ble gjort når økten startet var å kommunisere og skrive opp problemstillingen på en tavle. Vi valgte å organisere idémyldringen som et tankekart ettersom dette var en kjent løsning for gruppen. Den ustrukturerte tilnærmingen til idémyldring ble brukt på grunn av dens uformelle struktur. Ingen er heller dominerende i gruppen, som gjør det mulig for alle å komme med innspill. Hvis noen i gruppen hadde vært dominerende hadde vi heller gått over til å bruke skriftlig idémyldring, også kjent som idéskriving. 

Vi diskuterte og prøvde å komme på mulige måter trusselaktører kan kompromittere brukerkontoer til de ansatte ved universitetet, og hvordan passord og brukernavn kan komme på avveie.

\subsection{Resultater}
Etter øktene var ferdig ble det gjort en vurdering av resultatene og de ble kategorisert i henhold til likhetstrekk, under en fellesnevner som for eksempel sosial manipulering. Resultater og grupperinger er som vist i figur \ref{fig:idemyldring} under.

\begin{figure}[H]
    \centering
    \includegraphics[scale=0.5]{case_2/bilder/idemyldring.pdf}
    \caption[Idémyldring]{Resultater og gruppering av idémyldringen}
    \label{fig:idemyldring}
\end{figure}

Resultatene er gruppert inn i 4 hovedkategorier:
\begin{description}
    \item [Dårlig sikkerhet] er alt fra enkle passord til passordgjenbruk.
    \item [Tap av data] inkluderer at for eksempel sider som dropbox har en lekkasje av brukere.
    \item [Sosial manipulering] vil si å få tak i informasjon ved å lure noen.
    \item [Ondsinnet programvare] er programvare brukt som hjelpemiddel for å få tak i brukerinformasjon.
\end{description}

\subsection{Konklusjon av verktøyet}
Dette var et effektivt verktøy for å få en overordnet oversikt over hva årsakene til at brukerkontoer blir kompromittert og grunner til at passord og brukernavn kan være på avveie. Verktøyet fungerte godt til å komme med mulige måter for aktører å få tilgang til brukerkontoer, eneste som trekker tilbake er det at vi ikke har fått et godt bilde av hvorfor angripere velger å gå etter ansattkontoer, selv om vi fikk et litt simpelt bilde på dette med problemforståelsen.

\chapter{Datainnsamling}
Datainnsamlingen er kritisk til resultatet av oppgaven. Målet i denne fasen er å samle inn et så bredt aspekt av informasjon som mulig gjennom et par mulige verktøy og teknikker beskrevet i boka om rotårsaksanalyse\cite{RCA}. Vi har valgt å bruke spørreundersøkelser som et verktøy for å hente inn informasjon om dette caset fordi vi skal hente inn informasjon fra et stort antall studenter som bor i studentbyene SiT administrerer.

\section{Elektronisk spørreundersøkelse}
Det finnes i hovedsak to forskjellige undersøkelsestyper, kvantitative og kvalitative spørreundersøkelser. Kvalitative undersøkelser går ut på å spørre utvalgte personer, og er ofte mye mer detaljerte og samler svar av høyere kvalitet. Kvantitative undersøkelser fungerer motsatt i at det er fokus på mange tilbakemeldinger slik at en kan senke usikkerhet knyttet til svarkvalitet. \cite{} I vår situasjon har vi valgt kvantitativ undersøkelse på bakgrunn av et par faktorer. For det første ønsker vi at undersøkelsen skal være helt anonym, siden spørsmålene omhandler potensielle lovbrudd. For det andre er målgruppen et stort antall personer, så det kan være nyttig å samle inn data fra så mange av de som mulig.

\subsection{Ønsket utbytte}
Det vi ønsker å få ut av spørreundersøkelsen er data på utvalgte spørsmål vi mener er relevante for oppgaven. Denne fasen er utelukkende for å innhente informasjon, bearbeiding av informasjonen skjer i neste fase. Spørsmålene er utarbeidet for å utforske hvorfor studenter som bor i studentbyer laster ned opphavsrettsbeskyttet materiale, som blant annet inkluderer undersøkelse av økonomiske perspektiver og tilgjengelighet på tjenester. I tillegg ønsker vi også innsikt i hvordan dette kan stoppes. 

Hypotesen vi går inn i undersøkelsen med er at folk laster ned opphavsrettsbeskyttet materiale fordi det er lett tilgjengelig, tilknyttet liten til ingen kostnad og ikke minst fordi det er svært lav risiko for represalier.

\subsection{Gjennomføring}
Prosessen startet ved å utforme spørreundersøkelsen. En god undersøkelse vil alltid kreve kartlegging av demografi, og i vår undersøkelse valgte vi å kartlegge studentby, kjønn, alder og fakultet. Kjønn og alder er ganske selvforklarende, mens studentby ble valgt på bakgrunn av at Kallerud har mye raskere nedlasting- og opplastingshastighet enn de andre stedene. Vi anså også at det ville være forskjell på hvor mange som laster ned mellom for eksempel informatikkstudiene og helsestudiene. Videre ble resultatene i de foregående fasene brukt til å utforme spørsmålene. Spørreundersøkelsen inkluderer spørsmål om hvor godt en rekke påstander stemmer for den enkelte der respondentene svarer på en likert-skala fra 1-5, der 1 er i liten grad og 5 er i stor grad. Likert-skala ble valgt fordi det er en anerkjent måte å få inn kvantitative svar på hvor enige personer er med en påstand. Samtidig kan man enkelt sammeligne forholdene mellom svarene på de forskjellige påstandene ved hjelp av statistisk analyse. Til slutt inkluderes spørreundersøkelsen et spesielt viktig spørsmål om hva som skal til for at personen stopper med ulovlig nedlasting. Dette er et frisvar der vi kommer til å analysere individuelle svar hver for seg. Spørreundersøkelsen kan leses i sin helhet i \hyperref[undersokelse]{vedlegg A}.
\newline
Det er alltid en stor utfordring å finne nok respondenter til spørreundersøkelser. Vi setter en del krav til antall respondenter og utfører en rekke tiltak for å oppnå nok besvarelser, slik at undersøkelsen kan si noe om rotårsaken med relativt høy sikkerhet. Det er et krav å få minst 30 besvarelser som hadde lastet ned opphavsrettsbeskyttet materiale mens de har bodd i hybelen. Videre er det også ønskelig med relativ likhet i antall respondenter mellom de ulike fakultetene og studentbyene. Det hadde vært ideelt med minst 30 respondenter i hver kategori her også, men det er ønsketenkning i denne sammenhengen. Under prosessen ble også totalt antall beboere fra alle studentbyene i SiT bolig kartlagt. Boligtorget ga oss innbyggertallene fra hver studentby, og vi regnet oss frem til totalt 522 beboere. Det er viktig å presisere at det kan være usikkerheter knyttet til disse tallene, siden det kan hende ikke alle boligene har en beboer.

Siden spørreundersøkelsen er elektronisk var et av de første tiltakene som ble gjennomført å spre den på relevante facebook-sider. Et av prosjektgruppens medlemmer jobber på Studenthuset her på Gjøvik, og fikk spørreundersøkelsen delt på deres facebook-side. Senere i prosessen ble også undersøkelsen delt på facebook-siden til linjeforeningen INGa og klassesidene til sykepleierne og webutvikling. Undersøkelsen ble også delt gjennom venner og bekjente; disse var for det meste informatikkstudenter. I tillegg ble det laget en plakat som ble hengt opp på oppslagstavler på skolen og i vaskeriene i de ulike studentbyene, og mange ble også plassert i postkassene til SiT boliger. Plakaten finnes i \hyperref[plakat]{Vedlegg B}. 

\subsection{Resultater}
Undersøkelsen endte med 97 svar totalt, dette er 18.6\% av de 522 beboerne i SiT bolig. Av disse var det 34 som svarte at de hadde lastet ned opphavsrettsbeskyttet materiale i hybelen, det er 35\% av de spurte. Videre resultater og analyse av disse blir diskutert i neste fase. 

\subsection{Konklusjon av verktøy}
Konklusjonen er at spørreundersøkelser er en effektiv metode for å samle inn store mengder data. En må likevel være konsekvent på tiltakene som iverksettes for å få inn respondenter. Det er også viktig i kvantitative spørreundersøkelser å få relativ likhet i demografien for å kunne si noe om de ulike kategoriene. Dette kan i noen tilfeller være krevende. En annen ting som er svært sentralt i bruk av verktøyet er utformingen av spørsmålene. Det er vanskelig å legge til spørsmål etter spørreundersøkelsen allerede er utsendt, og hva en da trenger ekstra informasjon må dette supplementeres i en mulig kvalitativ undersøkelse eller liknende. 
\chapter{Dataanalyse}
I denne fasen analyseres dataene som er samlet inn, vi har ganske lite data i dette caset, vi har derfor et ganske lite sett med verktøy å bruke fra RCA boka \cite{RCA}

\section{Affinitetsdiagram}
Affinitetsdiagram brukes til å analysere data som det ikke er mulig å nummerere, eksempelvis meninger eller ideer. Affinitetsdiagram grupperer data og finner de underliggende korrelasjoner og likhetstrekk i gruppen.


\section{Ønsket utbytte}
Ønsket utbytte av å bruke affinitetsdiagram er å finne bindinger\/fellesnevnere som kan være til hjelp for å fjerne rotårsaken. 

\section{Gjennomføring }
Analysen ble gjennomført med å ta transkripsjon av intervjuet og stykke den opp i fem hovedgrupper.   

\begin{figure}[H]
    \centering
    \includegraphics[scale=0.6]{case_3/bilder/AD.pdf}
    \label{fig:AD_miner}
    \caption{Hvordan fungerer utvinning av kryptovaluta ved NTNU?}
\end{figure}

Vi finner mulige årsaker og tiltak som er satt på plass, og forhåpentligvis er rotårsaken blant dem. 

\section{Konklusjon av verktøy}
Verktøyet fungerte godt, selv med en liten datamengde. Vektøyet er effektivt til å strukturere transkripsjonen fra intervjuet til mer brukbare og oversiktlige nøkkelpunkter.

\chapter{Rotårsaksidentifisering}
Arbeidet i denne fasen går ut på å identifisere rotårsaken. I foregående fasen ble en rekke mulige årsaker identifisert og analysert, men nå er det tid for å finne den faktiske rotårsaken. Det er mange forskjellige verktøy som kan brukes i denne fasen, men vi har valgt oss en type årsak-virkning diagram for vårt utgangspunkt.

\section{Årsak-virkningsdiagram}
Et årsak-virkning diagram er et diagram som analyserer forholdene mellom et problem og dets årsaker. Det kombinerer aspekter ved idémyldring med systematisk analyse. Det finnes to typer årsak-virkning diagrammer, fiskebeindiagram og prosessdiagram. Mens et prosessdiagram er mer direkte fokusert på problemet på innsiden av forretningsprosessene, er et fiskebeindiagram en mer generell tilnærming for å adressere alle potensielle årsaker\cite{RCA}. Fiskebeindiagram passer best i dette caset.

\subsection{Ønsket utbytte}
Ved bruk av dette verktøyet ønsker vi å sitte igjen med en visuell fremstilling av rotårsaken til problemet. Dette vil gjøres ved å identifisere hva som skaper årsakene vi har funnet fram til i foregående fase.

\subsection{Gjennomføring}
Det er anbefalt å bruke en tusjtavle for å tegne opp fiskebeindiagrammet, men vi valgte å bruke et nettbasert program som er laget for å skape diagrammer med flere brukere involvert i sanntid. De hadde en egen mal for fiskebeindiagram som vi valgte å gå ut fra. Stegene vi fulgte i prosessen er hentet fra boka om rotårsaksanalyse \cite{RCA} og ble som følger:
\begin{enumerate}
    \item Vi beskrev problemet klart og tydelig
    \item Vi tegnet opp problemet på slutten av fiskebeindiagrammet
    \item Vi identifiserte hovedkategoriene av årsakene til problemet og tegnet det opp på fiskebeinene i diagrammet
    \item Vi idémyldret alle mulige årsaker i hver kategori, en kategori om gangen, og skrev det inn i diagrammet fortløpende
    \item Til slutt analyserte vi de identifiserte årsakene og bestemte de mest sannsynlige rotårsakene
\end{enumerate}

\subsection{Resultater}
Vi beskrev problemet som ulovlig fildeling på skolenettet etter tittelen på caset. Hovedkategoriene vi ønsket å utforske har vi basert på dataanalysen i forrige fase for å finne de mest relevante. Disse var etter vår mening Økonomi, Risiko og Tilgjengelighet. Idémyldringen var i stor grad basert på data og funn fra analysen, med innslag fra den første idémyldringen og andre nye idéer som dukket opp.

\begin{figure}[H]
    \centering
    \includegraphics[scale=0.5]{case_1/bilder/fiskebein.pdf}
    \label{fig:fiskebein}
    \caption[Fiskebein]{Fiskebeindiagram over hovedkategorier og årsaker}
\end{figure}

Etter videre analyse av figuren har vi kommet fram til at rotårsaken til fildeling er en kombinasjon flere faktorer, men én skiller seg ut, nemlig tilgjengelighet. Med tilgjengelighet mener vi spesifikt at folk bedriver ulovlig fildeling fordi det er dårligere utvalg på alternative tjenester i Norge. Det finnes også noen mindre årsaker som påvirker folk til å laste ned. Blant dem er at mange føler tjenestene ikke er verdt prisen de må betale når de bare får tilgang på en begrenset mengde materiale. Den siste årsaken går på at håndheving av lovene knyttet til ulovlig fildeling ikke blir prioritert, og derfor har skapt en kultur der det er sosialt akseptabelt å laste ned. Rotårsakene er listet etter viktighet der den første er hovedårsaken:

\begin{enumerate}
    \item Dårligere utvalg på alternative tjenester i Norge
    \item Tjenestene er ikke verdt prisen
    \item Håndheving og kommunisering av lovene knyttet til ulovlig fildeling blir ikke prioritert
\end{enumerate}

\subsection{Konklusjon av verktøy}
Verktøyet fungerte som en strukturert måte å kategorisere alle mulige rotårsaker vi hadde kommet fram til i tidligere faser. Det var også mulighet for å inkludere andre mulige årsaker som vi hadde i tankene gjennom prosessen. Det kan kanskje i noen sammenhenger være vanskelig å kategorisere de ulike årsakene dersom disse gruppene ikke er klart definert på forhånd, men i denne sammenhengen var det veldig intuitivt. Det kan også hende idémyldringen blir litt for fokusert på de ulike kategoriene siden man definerer disse på forhånd. Ellers et veldig godt verktøy som ga oss relevante resultater. 
\chapter{Problemeliminering}
Arbeidet i denne fasen går ut på å finne løsninger for å eliminere rotårsaken. I denne fasen har vi valgt å bruke vektøyet seks tenke hatter for å finne de beste måtene å få eliminert rotårsaken på.

\section{De seks tenkehattene}
De seks tenkehattene er et diskusjons verktøy, som går ut på at man gir personer forskjellige hatter/roller til prossessen. De forskjellige rollene er som følger:
\begin{description}
    \item [Hvit hatt] kynisk, faktabasert og systematisk.
    \item [Rød hatt] er den emosjonelle personen, han som følger magefølelsen og egen intuisjon
    \item[Svart hatt] pessimistisk og negativ, fokuserer på hvordan idéen ikke vil
    fungere
    \item [Gul hatt] optimistisk og positiv, fokuserer på hva som skal til for at løsningen skal kunne fungere
    \item[Grønn hatt] fokuserer på kreativitet og skal prøve å bygge på idéer
    \item[Blå hatt] assosieres med himmelen, og skal se problemet fra et større perspektiv.
\end{description}

\subsection{Ønsket utbytte}
Ønsket utbytte med bruk av seks tenkehatter, er å skape en forståelse rundt rotårsaken og komme opp meg mulige tiltak for å eliminere problemet. Siden problemet virker vanskelig å fikse fra skolen sin side måtte vi komme med noen kreative løsninger, og de seks tenkehattene fungerer godt for dette.

\subsection{Gjennomførelse}
Siden vi bare var fire tok to av oss på seg to hatter og resten en. Så startet vi å diskutere problemstillingen og hvordan vi burde gå inn for å eleminere rotårsaken, vi kom frem til en rekke mulige løsninger. Etter at vi var ferdig med de seks tenkehattene gikk vi fort igjennom de forskjellige forslagene, for å se på hva som var praktisk gjennomførbare og ikke, på slutten av prosessen tok vi altså å brukte de to siste punktene fra SIT, for å luke ut de beste forslagene, og for å bestemme oss for hva som var gjennomførbart

\subsection{Resultater}
Ut i fra prosessen med de seks tenkehattene kom vi fram til både gjennomførbare og ikke gjennomførbare tiltak. 
\subsubsection*{Gjennomførbare tiltak}
\begin{itemize}
    \item Se på de forskjellige brevene fra rettighetshaverene og se om det er noen av dem som er mer representert enn andre, og se om det er noen mulighet for å tilby filmer og serier fra disse selskapene.
    \item Stenge torrentprotokoellen for alle på nettverket.
    \item Grense på nedlastning og opplastning av data.
    \item Være strengere når det gjelder oppfølging av IT-reglementet. 
    \item Bytte ISP til studentboligene.
    \item Oppmerksomhetskampanje om konsekvenser.
    \item Avtale med kino for billige/gratis nye filmer
%%%%%%%%%%%%%%%%%%%%%%%%%%%%%%%%%%%%%%%%%%%%%%%%%%%%%%%%%%%%%%%%%%%%%%%%%%%%%%%%%%%%%%%%%%%%
    \item Stenge nettet i 5 min for alle om noen starter å laste ned, kollektiv straff. %OBS OBS BURDE FJERNES!!!!!!!!!!!!!!!! FRA HOVEDRAPPORTEN
\end{itemize}

\subsubsection*{Ikke gjennomførbare tiltak}
\begin{itemize}
    \item Alt av materiale blir gratis og tilgjengelig på ett samlet sted
    \item Fjerne geografiske blokkeringer
\end{itemize}

Noen av disse vil ikke være gjennomførbare for NTNU så vi har valgt å ikke ta de med videre til implementering, men vi følte de kunne være interessante å nevne. Av de 11 forslagene er det kun 4-5 vi mener har høy sannsynlighet for å bli kvitt rotårsaken, helt eller delvis, eller flytter rotårsaken vekk fra NTNU ansvarsområde. Etter videre vurdering har vi kommet fram til de mest lovende løsningene for å fjerne rotårsaken: 
\subsubsection*{Disse fjerner rotårsaken til at folk laster ned}
\begin{enumerate}
    \item Alt av materiale blir gratis og tilgjengelig på ett samlet sted
    \item Fjerne geografiske blokkeringer
\end{enumerate}
\subsubsection*{Disse fjerner rotårsaken til at skolen får notifikasjon fra opphavsrettshaverne}
\begin{enumerate}
    \item Bytte ISP til studentboligene
    \item Stenge torrentprotokollen for alle på nettverket
\end{enumerate}
Rotårsaken til at folk bedriver ulovlig fildeling er ikke et problem som lett kan løses av skolen. Vi har gitt et par forslag til hva som kan fjerne rotårsaken helt, men dette er ikke gjennomførbart. Vi har også foreslått et par tiltak som ikke nødvendigvis fjerner rotårsaken for hvorfor folk laster ned, men flytter rotårsaken bort fra NTNU's ansvarsområde.
%%%%%%%%%%%%%%%%% FJERNE FFFFFFFFFFFFFFFFFFFFFFFFFFFFFF
En annen løsning som kunne vært et stilig tankeprosjekt, er å stenge nettet i 5 min for hver gang NTNU får en notifikasjon på brudd på opphavsretten. 
%%%%%%%%%%%%%%%% FJERNE FFFFFFFFFFFFFFFFFFFFFFFFFFFFFF

\subsection{Konklusjon av verktøy}
Verktøyet fungerte godt til å starte en diskusjon rundt elimineringsalternativer, men det utviklet seg fort til en form for idémyldring, siden det var vanskelig å forholde oss til de forskjellige sinnstilstandene/hattene. Det var også et par komplikasjoner siden to av oss måtte ha to hatter samtidig. Ellers var det en kreativ og morsom prosess der mange gode alternativer kom fram. 

\chapter{Løsningsimplementering}
Arbeidet i denne fasen går ut på å utrede en tiltaksplan og lage et forslag til hvordan dette skal implementeres. I den foregående fasen ble løsningene til rotårsakene identifisert. Vi kom fram til fire tiltak som vil fjerne rotårsaken. Tiltakene kan deles inn i to deler, en for eksterne og en for interne. Den eksterne løsningen går på det tekniske og den interne går på IT-reglementet. 

Den eksterne løsningen er å øke ressursene til seksjonen for digital sikkerhet gjennom å ansatte flere eller å bruke bachelorstudenter til å utføre blokkering av DNS-forespørsler tilknyttet kryptoutvinning.

Den interne løsningen er å tydeliggjøre at det å drive kryptoutvinning på NTNU ikke er lovlig og gjennomføre enn informasjonskampanje rundt dette.

\section{Kraftfeltsanalyse}
Kraftfeltsanalyse er et verktøy som analyserer hva som hjelper og hva som hindrer implementering av tiltaket.  

\subsection{Ønsket utbytte}
Ønsket utbytte fra kraftfeltsanalyse er å få vite hva som er for og hva som er imot implementering av tiltakene. Dette verktøyet gir en plan over hvilke tiltak som er lettest å gjennomføre. 

\subsection{Gjennomføring}
Kraftfeltsanalysen ble gjort ved at vi tok tiltakene fra problemelimineringen og hadde en idémyldring for å se hva som talte for tiltakene og hva som var imot. 

\subsection{Resultat}
 Under har vi de fire kraftfeltsanalysene
 
 Informasjonskampanjen og endringen i IT-reglementet bør gjøre i kombinasjon med hverandre. Der IT-reglementet får klartgjort at selv om kryptoutvinning ikke ulovlig i henhold til norsk lov, er det imot NTNU sitt IT-reglement så langt det ikke er søkt om. Når endringen er gjort, gjennomføres informasjonskampanjen.   
 Under, i tabell \ref{fig:kampanje} og \ref{fig:IT-reglement}, viser resultatene fra kraftfeltsanalyse på informasjonskampanje og endring i IT-reglementet.
 \begin{figure}[H]
    \hspace{2.2cm}
    \includegraphics[scale=0.6]{case_3/bilder/Force-field1.pdf}
    \caption[Informasjonskampanje]{Oversikt over informasjonskampanjen }
    \label{fig:kampanje}
\end{figure}

 
 \begin{figure}[H]
    \hspace{2.6cm}
    \includegraphics[scale=0.6]{case_3/bilder/Force-field2.pdf}
    \caption[Endre IT-reglementet]{Endring i IT-reglementet}
    \label{fig:IT-reglement}
\end{figure}

Fra dataanalysen kom fram til at selv om det finnes tekniske løsninger, har ikke SOCen hatt mulighet til å implementere DNS blokkering på bakgrunn av mangel på ressurser. Figur \ref{fig:Blokkering} og \ref{fig:Oke-antall} viser hva som skal til for å blokkere DNS og hva som må til for å øke ressursene til SOCen.    
 \begin{figure}[H]
    \centering
    \includegraphics[scale=0.6]{case_3/bilder/Force-Field3.pdf}
    \caption[Blokkering]{Blokkering av DNS forespørsel}
    \label{fig:Blokkering}
\end{figure}

 \begin{figure}[H]
    \hspace{3.6cm}
    \includegraphics[scale=0.6]{case_3/bilder/Force-field4.pdf}
    \caption[Øke antall ansette i SOC]{Øke andel ansatte i SOC}
    \label{fig:Oke-antall}
\end{figure}

Figurene viser våres antakelser på hva som jobber for implementeringen og hva som jobber imot samt  estimater på styrken til antakelsene. 
\subsection{Konklusjon av verktøyet}
Verktøyet har potensiale til å fungere bra, der oversikten man får er bra så lenge datagrunnlaget er godt. Problemmet vårt var at vi jobbet med antakelser og ikke et datasett. Vi hadde ikke tid til å undersøke estimatene, så de har en høy usikkerhet.     
\chapter{Diskusjon}
Dette kapittelet eksisterer for å reflektere litt over prosessen og resultatene vi kom fram til. Vi vil også diskutere effekten ved bruk av rotårsaksanalyse til å løse informasjonssikkerhetsrelaterte hendelser knyttet til ulovlig fildeling. 

\section{Resultater}
Her nevner vi kort våre resultater i de viktigste fasene som hadde direkte innvirkning på identifisering og eliminering av rotårsaken. 

\subsection{Datainnsamlingen}
Det var forskjellige metoder å drive med datainnsamling, vi valgte kvantitativ spørreundersøkelse for å spørre mest mulig beboere fra Sit. Vi fikk svarprosent på ca 18\%, og vi forventet svarprosent på 15-20\% av alle beboere i Sit boligene i Gjøvik. Vi valgte å forholde oss til studentbyene i Gjøvik, ikke i Trondheim eller Ålesund. Av studentbyene vi spurte, fikk vi desidert mest svar fra Kallerud. Av alle som svarte var det 50\% som bodde på Kallerud

Vi postet et innlegg på Huset ansatte, og tok kontakt med linjeforeningene der vi spurte om de kunne legge ut en link til undersøkelsen. Her fikk vi best respons fra Huset ansatte og INGa sine facebooksider. Vi la også en plakat i litt under halvparten av postkassene på Kallerud og på Sørbyen, og vi fikk grei respons fra dette.

Etter at det hadde gått en uke oversatte vi spørreundersøkelsen til engelsk, der vi hadde en kommunikasjonskanal som kunne sende denne til alle de internasjonale studentene, der mange av disse bor på Sit hybler. Vi fikk rundt 20 respondenter fra de internasjonale, og alle disse resultatene ble oversatt til norsk og lagt inn i et samlet spørreskjema.


\subsection{Dataanalysen}
Kartlegging av omfanget viste at 35\% av de spurte drev med ulovlig fildeling. Dette var lavere enn vi trodde, men fortsatt mange. De fleste var småforbrukere, men det var også en del storforbrukere som laster ned over ti torrents i måneden. Fra dataanalysen kunne vi også konkludere med at tilgjengelighet var en svært viktig grunn til at folk lastet ned. Selv de som hadde tilgang på mange strømmetjenester svarte at tilgjengeligheten var en viktig grunn til at de lastet ned. Økonomi var viktig for noen, men også uviktig for en god del. Det viste seg også at det var dårlig håndhevelse og kommunikasjon av lover og regler. 

Det var også forskjeller blant demografiene. Menn var en stor andel av de som lastet ned, mens kvinner nesten ikke lastet ned noe. Når det kommer til fakultet var IT fakultetet overrepresentert i nedlastingsstatistikken. De var også de som kjente til IT reglement og konsekvenser best. Generelt sett var det lite kunnskap om IT reglement, og varierende kjennskap til konsekvenser. 

\subsection{Rotårsaksidentifiseringen}
Vi valgte å bruke fiskebein for å strukturerte vår rotårsaksidentifisering. Som et verktøy fungerte det meget godt, der vi klarte å få organisert årsakene inn i tre hovedgrupper: Økonomi, Tilgjengelighet og Risiko. Vi analyserte hver hovedgruppe nøyere og kom fram til en årsak for hver gruppe: Dårligere utvalg på alternative tjenester i Norge, Tjenestene er ikke verdt prisen og Håndheving og kommunisering av lovene knyttet til ulovlig fildeling blir ikke prioritert. 

\subsection{Rotårsakselimineringen}
Det vi kom frem til her var fire forskjellige løsninger for å fjerne rotårsaken, der to av dem var mer globale og ikke gjennomførbare for skolen, og to mer praktisk gjennomførbare som ikke fjerner det at folk laster ned, men skyver problemet over til andre. De ikke gjennomførbare var at alt av materiale blir gratis og tilgjengelig på ett samlet sted, og fjerne de geografiske blokkeringene. De gjennomførbare var at man bytter ISP til studentboligene for å forflytte problemet bort fra NTNU's ansvarsområde, og stenge torrentprotokollen for alle på nettverket. 

\subsection{Løsningsimplementeringen}
Vi benyttet trediagram (figur \ref{fig:Tre-diagram}) for å illustrere de ulike arbeidsoppgavene som kreves for å implementere tiltakene.

\section{Diskusjon}
%-- PRØV Å NEVNE BRUK AV ROTÅRSAKSANALYSE I INFOSEC --%

\subsection{Rotårsak: Dårligere utvalg på alternative tjenester i Norge}
Ut fra vår analyse viser det seg at dette er hovedårsaken til at studenter laster ned ulovlig. Siden Netflix og de andre strømmetjenestene har inntatt markedet har filmer og serier gruppert seg mellom de. Tjenestene ønsker også flest mulig orginale serier som bare er hos dem. Dette gjør at tilgjengeligheten på filmer og serier går ned, med mindre man abonnerer på alt. Men selv da får man ikke tilgang på alt. Mange filmer og serier er geografisk blokkert i Norge, som gjør tilgjengeligheten til et enda større problem. I musikkstrømmingsmiljøet er problemet noe mindre. Selv om ulovlig musikknedlasting ikke er borte, har det blitt redusert \cite{musikkstream}. Noe av grunnen til dette er at musikkbransjen er mer sentralisert i hvem som eier rettighetene, også kjent som et oligopol. Dette gjør det lettere for strømmetjenestene å skaffe lisenser for musikk, og kan tilby det folk trenger på ett sted. Det er også mye mindre orginalt innhold i disse tjenestene i forhold til strømmetjenester for filmer og serier. 

\subsection{Rotårsak: Tjenestene er ikke verdt prisen}
Jo flere tjenester det blir, jo mer må man betale for å få tilgang på mer materiale. Filmer og serier blir spredt utover markedet på flere tjenester som så og si koster det samme. Dette fører til at hver enkelt tjeneste blir mindre verdt pengene man må betale for å få tilgang. Det skal sies at strømming er en revolusjonerende løsning i forhold til å kjøpe hver enkelt film for seg selv, men hvis man må betale for fem forskjellige strømmetjenester for å få tilgang til det man har lyst på, hvorfor ikke bare laste ned gratis? Analysen vår viste at det å betale for tjenester ikke var noe problem for studentene; problemet var at de ikke føler de får det de betaler for. 

\subsection{Rotårsak: Håndheving og kommunisering av lovene knyttet til ulovlig fildeling blir ikke prioritert}
Det eksisterer allerede regler på ulovlig nedlasting på universitetsnettet. Problemet er derimot at det er vanskelig å håndheve de. Andre arbeidsoppgaver har heller blitt prioritert. Enkelte tiltak har heller ikke vært lovlige for NTNU å gjennomføre for å stoppe de som driver med ulovlig fildeling. For eksempel er det ikke lov å overvåke enkeltboliger hos Sit, og heller ikke straffe enkeltpersoner dersom de laster ned, siden det blir regnet som inngrep i den private sfæren. Dette har datatilsynet fortalt Seksjon for Digital Sikkerhet. 

\subsection{Nytteverdien ved bruk av rotårsaksanalyse innen informasjonssikkerhet}
Det er fortsatt få studier som prøver å sette lys på nytteverdien ved bruk av rotårsaksanalyse innen informasjonssikkerhet. I løpet av dette caset har vi gjort oss en erfaring basert på verktøybruken. Basert på resultatene for caset kan det sies å ha fungert bra, men på den ene siden vet vi ikke helt hvor bra det har fungert før tiltakene er implementert, og det er kontrollert at symptomene minker eller forsvinner helt. På den andre siden har et tidligere bachelorprosjekt allerede kommet frem til at nytteverdien er stor. De stilte blant annet spørsmål om hvor godt det fungerer på case med lite tid og ressurser, samt mye tid og ressurser \cite{RCARapport}. Det ble i begge sammenhenger konkludert med at det ga gode resultater. 

\subsection{Diskusjon rundt prosessen}
Problemforståelsen i informasjonssikkerhetssammenheng er ganske enkelt å utføre siden mange logger gir godt grunnlag for en kritisk hendelsestabell. I dette caset ble det brukt for å kartlegge hva som blir lastet ned, dette var informasjon som var viktig for å utforme spørreundersøkelsen. Det er et verktøy som er verdt å benytte ofte. Idémyldringen krever en del bakgrunnskunnskap om problemet og fagområdet, så hvis en ikke har gjort en god problemforståelse, kan det gå ut over idémyldringen. I dette caset fungerte det bra å gjøre den muntlig, men det vil også fungere bra å utføre en idéskriving dersom ikke alle kan være tilstede sammen. 

Under datainnsamlingen var det vanskelig å planlegge gode tiltak for å distribuere spørreundersøkelsen, og det førte til et større tidsbruk enn det vi hadde sett for oss. Grunnet til større tidbruk var blant annet at Sit ikke gir ut tilgang til mailingliste som gjorde at vi måtte være mer kreative med innsamlingsmetodene. En annen faktor er at vi hadde liten til ingen erfaring med spørreundersøkelser, og tok innsamlingsprosessen litt på sparket. 

Av tiltakene vi utførte for å få flere folk til å delta i spørreundersøkelsen, var det mest effektive tiltaket å poste undersøkelsen på facebooksider. Vi la også merke til at de kanalene vi delte spørreundersøkelsen på, så kunne vi forvente mest svar samme dag. Når det gikk lengre tid så falt svarresponsen og vi måtte purre etter en uke for å få opp antall svar. Å dele ut plakater fungerte også helt greit, men det kan virke litt nærgående for noen. 

Under datanalysen ble det ved hjelp av histogram og affinitetsdiagram funnet flere datasett av interesse. ANOVA-analyse fant noe, men på grunn av at manglende datagrunnlag kunne vi ikke trekke noen konklusjoner. Grunnen til at vi hadde et manglende datagrunnlag kom av at det var rundt 35 \% som laster ned, når vi da skulle gjøre analyser på folk som lastet ned og se på dette i forhold med demografi, ble altså svarmengden snever. Siden vi ikke hadde vært borti SPSS før tok det litt tid å gjøre seg kjent med programmet. Verktøy som histogrammer og affinitetsdiagrammer var nyttig for å trekke konklusjoner i dette caset.

Fiskebein fungerte utmerket som et verktøy og gjorde at vi fant ut at tilgjengelighet, økonomi og risiko er de tre hovedgrunnene vi har til at folk laster ned, dette stemmer relativt godt med antakelsene våre, fra før vi begynte med casen. Disse var at vi antok tilgjengelighet og økonomi ville være de største faktorene til at folk driver med nedlasting, resultatene vi fikk fra datanalysen pekte derimot på at det for det meste var tilgjengelighet som var en stor faktor. Økonomi var det færre folk som brydde seg om enn vi først antok. 

Elimineringen var den siste store etappen, her brukte vi de seks tenkehattene, denne metoden er nok en av de vanskeligere å benytte seg av. siden den ble brukt i en gruppe på 4 personer, der noen må ta på seg flere hatter. Resultatene var gode, men det var krevende å holde seg til rollene sine. 

Implementeringen kunne vi ikke gjøre så mye med selv, men tiltakene beskrevet er ikke veldig avanserte selv om det kan kreve mye tid og ressurser siden det er snakk om ganske drastiske endringer. 

\subsection{Videre arbeid}
Videre arbeid kan være å gå gjennom forslagene våre, og se om noen av disse er fornuftige å implementere, og vil være verdt kostnadene. Ulovlig fildeling på skolenettet er et veldig vanskelig problem å fjerne rotårsaken til siden tilgjengelighet er den store drivkraften bak nedlasting. Det kan også være nyttig å undersøke hvilke opphavsrettshavere som sender notifikasjoner, og se om skolen kan tilby tjenester hvis de fleste kommer fra ett sted. Videre arbeid kan også inkludere en risikoanalyse på risikoen dersom skolen ikke gjør noe med dette problemet.

\subsection{Tidsbruk}
Kommer...

%Røttene til dette problemet strekker seg dessverre dypere enn våre landegrenser og det er begrenset hva NTNU kan gjøre for å fjerne disse.

\bibliographystyle{ntnuthesis/ntnubachelorthesis}
\bibliography{case_1/bibliografi}

\appendix %after this line all chapters will have leters instead of numbers
\chapter*{Vedlegg A: Spørreundersøkelse}
\label{undersokelse}
\includepdf[pages={1-4}]{bilder/case1_sporreundersokelse}

\chapter*{Vedlegg B: Plakat}
\label{plakat}
\begin{figure}[H]
    \centering
    \includegraphics[scale=0.25]{case_1/bilder/plakat.pdf}
    \caption[Plakat]{Plakat som ble brukt i forbindelse med promotering av spørreundersøkelsen}
    \label{fig:plakat}
\end{figure}

\chapter*{Vedlegg C: Frekvenstabeller}
\label{frekvens}

% Table generated by Excel2LaTeX from sheet 'Ark1'
\begin{table}[htbp]
  \centering
    \begin{tabular}{|l|r|r|r|l|}
    \hline
    \multicolumn{5}{|p{30em}|}{\textcolor[rgb]{ .6,  .2,  0}{\textbf{\begin{center}Kjønn\end{center}}}} \\
    \hline
    \textcolor[rgb]{ .2,  .2,  .6}{} & \multicolumn{1}{p{5.355em}|}{\textcolor[rgb]{ .2,  .2,  .6}{Frequency}} & \multicolumn{1}{p{5.355em}|}{\textcolor[rgb]{ .2,  .2,  .6}{Percent}} & \multicolumn{1}{p{5.355em}|}{\textcolor[rgb]{ .2,  .2,  .6}{Valid Percent}} & \multicolumn{1}{p{5.355em}|}{\textcolor[rgb]{ .2,  .2,  .6}{Cumulative Percent}} \\
    \hline
    \textcolor[rgb]{ .2,  .2,  .6}{Kvinne} & \textcolor[rgb]{ .6,  .2,  0}{27} & \textcolor[rgb]{ .6,  .2,  0}{27.8} & \textcolor[rgb]{ .6,  .2,  0}{27.8} & \multicolumn{1}{r|}{\textcolor[rgb]{ .6,  .2,  0}{27.8}} \\
    \hline
    \textcolor[rgb]{ .2,  .2,  .6}{Mann}  & \textcolor[rgb]{ .6,  .2,  0}{70} & \textcolor[rgb]{ .6,  .2,  0}{72.2} & \textcolor[rgb]{ .6,  .2,  0}{72.2} & \multicolumn{1}{r|}{\textcolor[rgb]{ .6,  .2,  0}{100.0}} \\
    \hline
    \textcolor[rgb]{ .2,  .2,  .6}{Total} & \textcolor[rgb]{ .6,  .2,  0}{97} & \textcolor[rgb]{ .6,  .2,  0}{100.0} & \textcolor[rgb]{ .6,  .2,  0}{100.0} & \textcolor[rgb]{ .6,  .2,  0}{} \\
    \hline
    \end{tabular}%
  \caption{Frekvenstabell av kjønn}
  \label{tab:kjonn}%
\end{table}%

\begin{table}[htbp]
  \centering
    \begin{tabular}{|l|r|r|r|l|}
    \hline
    \multicolumn{5}{|p{30em}|}{\textcolor[rgb]{ .6,  .2,  0}{\textbf{\begin{center}Alder\end{center}}}} \\
    \hline
    \textcolor[rgb]{ .2,  .2,  .6}{} & \multicolumn{1}{p{5.355em}|}{\textcolor[rgb]{ .2,  .2,  .6}{Frequency}} & \multicolumn{1}{p{5.355em}|}{\textcolor[rgb]{ .2,  .2,  .6}{Percent}} & \multicolumn{1}{p{5.355em}|}{\textcolor[rgb]{ .2,  .2,  .6}{Valid Percent}} & \multicolumn{1}{p{5.355em}|}{\textcolor[rgb]{ .2,  .2,  .6}{Cumulative Percent}} \\
    \hline
    \textcolor[rgb]{ .2,  .2,  .6}{Under 20} & \textcolor[rgb]{ .6,  .2,  0}{9} & \textcolor[rgb]{ .6,  .2,  0}{9.3} & \textcolor[rgb]{ .6,  .2,  0}{9.3} & \multicolumn{1}{r|}{\textcolor[rgb]{ .6,  .2,  0}{9.3}} \\
    \hline
    \textcolor[rgb]{ .2,  .2,  .6}{20-25}  & \textcolor[rgb]{ .6,  .2,  0}{72} & \textcolor[rgb]{ .6,  .2,  0}{74.2} & \textcolor[rgb]{ .6,  .2,  0}{74.2} & \multicolumn{1}{r|}{\textcolor[rgb]{ .6,  .2,  0}{83.5}} \\
    \hline
    \textcolor[rgb]{ .2,  .2,  .6}{26-30} & \textcolor[rgb]{ .6,  .2,  0}{11} & \textcolor[rgb]{ .6,  .2,  0}{11.3} & \textcolor[rgb]{ .6,  .2,  0}{11.3} & \multicolumn{1}{r|}{\textcolor[rgb]{ .6,  .2,  0}{94.9}} \\
    \hline
    \textcolor[rgb]{ .2,  .2,  .6}{31-35} & \textcolor[rgb]{ .6,  .2,  0}{4} & \textcolor[rgb]{ .6,  .2,  0}{4.1} & \textcolor[rgb]{ .6,  .2,  0}{4.1} & \multicolumn{1}{r|}{\textcolor[rgb]{ .6,  .2,  0}{99.0}} \\
    \hline
    \textcolor[rgb]{ .2,  .2,  .6}{Over 35} & \textcolor[rgb]{ .6,  .2,  0}{1} & \textcolor[rgb]{ .6,  .2,  0}{1.0} & \textcolor[rgb]{ .6,  .2,  0}{1.0} & \multicolumn{1}{r|}{\textcolor[rgb]{ .6,  .2,  0}{100.0}} \\
    \hline
    \textcolor[rgb]{ .2,  .2,  .6}{Total} & \textcolor[rgb]{ .6,  .2,  0}{97} & \textcolor[rgb]{ .6,  .2,  0}{100.0} & \textcolor[rgb]{ .6,  .2,  0}{100.0} & \multicolumn{1}{r|}{\textcolor[rgb]{ .6,  .2,  0}{}} \\
    \hline
    \end{tabular}%
  \caption{Frekvenstabell av alder}
  \label{tab:alder}%
\end{table}%

\begin{table}[htbp]
  \centering
    \begin{tabular}{|l|r|r|r|l|}
    \hline
    \multicolumn{5}{|p{30em}|}{\textcolor[rgb]{ .6,  .2,  0}{\textbf{\begin{center}Studentby\end{center}}}} \\
    \hline
    \textcolor[rgb]{ .2,  .2,  .6}{} & \multicolumn{1}{p{5.355em}|}{\textcolor[rgb]{ .2,  .2,  .6}{Frequency}} & \multicolumn{1}{p{5.355em}|}{\textcolor[rgb]{ .2,  .2,  .6}{Percent}} & \multicolumn{1}{p{5.355em}|}{\textcolor[rgb]{ .2,  .2,  .6}{Valid Percent}} & \multicolumn{1}{p{5.355em}|}{\textcolor[rgb]{ .2,  .2,  .6}{Cumulative Percent}} \\
    \hline
    \textcolor[rgb]{ .2,  .2,  .6}{Kallerud} & \textcolor[rgb]{ .6,  .2,  0}{49} & \textcolor[rgb]{ .6,  .2,  0}{50.5} & \textcolor[rgb]{ .6,  .2,  0}{50.5} & \multicolumn{1}{r|}{\textcolor[rgb]{ .6,  .2,  0}{50.5}} \\
    \hline
    \textcolor[rgb]{ .2,  .2,  .6}{Nordbyen}  & \textcolor[rgb]{ .6,  .2,  0}{13} & \textcolor[rgb]{ .6,  .2,  0}{13.4} & \textcolor[rgb]{ .6,  .2,  0}{13.4} & \multicolumn{1}{r|}{\textcolor[rgb]{ .6,  .2,  0}{63.9}} \\
    \hline
    \textcolor[rgb]{ .2,  .2,  .6}{Sentrum} & \textcolor[rgb]{ .6,  .2,  0}{11} & \textcolor[rgb]{ .6,  .2,  0}{11.3} & \textcolor[rgb]{ .6,  .2,  0}{11.3} & \multicolumn{1}{r|}{\textcolor[rgb]{ .6,  .2,  0}{75.3}} \\
    \hline
    \textcolor[rgb]{ .2,  .2,  .6}{Sørbyen} & \textcolor[rgb]{ .6,  .2,  0}{24} & \textcolor[rgb]{ .6,  .2,  0}{24.7} & \textcolor[rgb]{ .6,  .2,  0}{24.7} & \multicolumn{1}{r|}{\textcolor[rgb]{ .6,  .2,  0}{100.0}} \\
    \hline
    \textcolor[rgb]{ .2,  .2,  .6}{Total} & \textcolor[rgb]{ .6,  .2,  0}{97} & \textcolor[rgb]{ .6,  .2,  0}{100.0} & \textcolor[rgb]{ .6,  .2,  0}{100.0} & \multicolumn{1}{r|}{\textcolor[rgb]{ .6,  .2,  0}{}} \\
    \hline
    \end{tabular}%
  \caption{Frekvenstabell av studentby}
  \label{tab:studentby}%
\end{table}%

\begin{table}[htbp]
  \centering
    \begin{tabular}{|p{5.355em}|r|r|r|l|}
    \hline
    \multicolumn{5}{|p{30em}|}{\textcolor[rgb]{ .6,  .2,  0}{\textbf{\begin{center}Fakultet\end{center}}}} \\
    \hline
    \multicolumn{1}{|r|}{\textcolor[rgb]{ .2,  .2,  .6}{}} & \multicolumn{1}{p{5.355em}|}{\textcolor[rgb]{ .2,  .2,  .6}{Frequency}} & \multicolumn{1}{p{5.355em}|}{\textcolor[rgb]{ .2,  .2,  .6}{Percent}} & \multicolumn{1}{p{5.355em}|}{\textcolor[rgb]{ .2,  .2,  .6}{Valid Percent}} & \multicolumn{1}{p{5.355em}|}{\textcolor[rgb]{ .2,  .2,  .6}{Cumulative Percent}} \\
    \hline
    \textcolor[rgb]{ .2,  .2,  .6}{Fakultet for arkitektur og design} & \textcolor[rgb]{ .6,  .2,  0}{15} & \textcolor[rgb]{ .6,  .2,  0}{15.5} & \textcolor[rgb]{ .6,  .2,  0}{15.5} & \multicolumn{1}{r|}{\textcolor[rgb]{ .6,  .2,  0}{15.5}} \\
    \hline
    \textcolor[rgb]{ .2,  .2,  .6}{Fakultet for informasjonsteknologi og elektroteknikk} & \textcolor[rgb]{ .6,  .2,  0}{52} & \textcolor[rgb]{ .6,  .2,  0}{53.6} & \textcolor[rgb]{ .6,  .2,  0}{53.6} & \multicolumn{1}{r|}{\textcolor[rgb]{ .6,  .2,  0}{69.1}} \\
    \hline
    \textcolor[rgb]{ .2,  .2,  .6}{Fakultet for ingeniørvitenskap} & \textcolor[rgb]{ .6,  .2,  0}{13} & \textcolor[rgb]{ .6,  .2,  0}{13.4} & \textcolor[rgb]{ .6,  .2,  0}{13.4} & \multicolumn{1}{r|}{\textcolor[rgb]{ .6,  .2,  0}{82.5}} \\
    \hline
    \textcolor[rgb]{ .2,  .2,  .6}{Fakultet for medisin og helsevitenskap} & \textcolor[rgb]{ .6,  .2,  0}{11} & \textcolor[rgb]{ .6,  .2,  0}{11.3} & \textcolor[rgb]{ .6,  .2,  0}{11.3} & \multicolumn{1}{r|}{\textcolor[rgb]{ .6,  .2,  0}{93.8}} \\
    \hline
    \textcolor[rgb]{ .2,  .2,  .6}{Fakultet for økonomi} & \textcolor[rgb]{ .6,  .2,  0}{6} & \textcolor[rgb]{ .6,  .2,  0}{6.2} & \textcolor[rgb]{ .6,  .2,  0}{6.2} & \multicolumn{1}{r|}{\textcolor[rgb]{ .6,  .2,  0}{100.0}} \\   
    \hline
    \textcolor[rgb]{ .2,  .2,  .6}{Total} & \textcolor[rgb]{ .6,  .2,  0}{97} & \textcolor[rgb]{ .6,  .2,  0}{100.0} & \textcolor[rgb]{ .6,  .2,  0}{100.0} & \textcolor[rgb]{ .6,  .2,  0}{} \\
    \hline
    \end{tabular}%
  \caption{Frekvenstabell av fakultet}
  \label{tab:fakultet}%
\end{table}%

\chapter*{Vedlegg D: Diverse Histogrammer}
\label{vedlegg:histogrammer}

\begin{figure}[H]
    \centering
    \includegraphics[scale=0.45]{case_1/bilder/IT_lasterned.pdf}
    \caption[IT-lasterned]{Forholdet mellom IT studier og andre når det kommer til nedlasting}
    \label{fig:IT-lasterned}
\end{figure}

\begin{figure}[H]
    \centering
    \includegraphics[scale=0.45]{case_1/bilder/reglement_lasterned.pdf}
    \caption[reglement-lasterned]{Forholdet mellom kjennskap til IT reglement og om en laster ned}
    \label{fig:reglement-lasterned}
\end{figure}

\begin{figure}[H]
    \centering
    \includegraphics[scale=0.45]{case_1/bilder/reglement_fakultet.pdf}
    \caption[reglement-fakultet]{Hvor godt kjennskap de ulike fakultetene har med IT reglementet}
    \label{fig:reglement-fakultet}
\end{figure}

\begin{figure}[H]
    \centering
    \includegraphics[scale=0.45]{case_1/bilder/antalltorrents.pdf}
    \caption[antalltorrents]{Hvor mange torrents folk laster ned i løpet av en måned}
    \label{fig:antalltorrents}
\end{figure}

\chapter*{Vedlegg E: SPSS analyser}
\label{vedlegg:ANOVA}

%-----------------------------------------------ONEWAY ANOVA - FAKULTET MOT PÅSTAND------------------------
%-----------------------------------------------DESCRIPTIVES-----------------------------------------------
\begin{figure}[H]
    \centering
    \includegraphics[scale=0.7]{case_1/bilder/DESCRIPTIVES_fakultet-pastand_LIGGENDE.pdf}
    \label{fig:DESCRIPTIVES_fakultet-påstand}
    \caption{Fakultet mot påstander}
\end{figure}


%-----------------------------------------------ANOVA-------------------------------------------------------
\begin{figure}[H]
    \centering
    \includegraphics[scale=0.7]{case_1/bilder/ANOVA_fakultet-pastand.pdf}
    \label{fig:ANOVA_fakultet-påstand}
    \caption{Fakultet mot påstander - ANOVA}
\end{figure}


%-----------------------------------------------POST HOC ANALYSE--------------------------------------------
\begin{figure}[H]
    \centering
    \includegraphics[scale=0.7]{case_1/bilder/Post_Hoc_test_fakultet-pastand.pdf}
    \label{fig:POST-HOC_fakultet-påstand}
    \caption{Fakultet mot påstander - POST HOC}
\end{figure}



%-----------------------------------------------ONEWAY ANOVA - IT/ANDRE MOT PÅSTAND------------------------
%-----------------------------------------------DESCRIPTIVES-----------------------------------------------
\begin{figure}[H]
    \centering
    \includegraphics[scale=0.7]{case_1/bilder/DESCRIPTIVES_IT-ANDRE-pastand.pdf}
    \label{fig:DESCRIPTIVES_IT/ANDRE-påstand}
    \caption{Fakultet delt inn i IT og andre mot påstander - DESCRIPTIVES}
    
\end{figure}

%-----------------------------------------------ANOVA-------------------------------------------------------
\begin{figure}[H]
    \centering
    \includegraphics[scale=0.7]{case_1/bilder/ANOVA_fakultet-pastand.pdf}
    \label{fig:ANOVA_IT/ANDRE-påstand}
    \caption{Fakultet delt inn i IT og andre mot påstander - ANOVA}
\end{figure}




\end{document}