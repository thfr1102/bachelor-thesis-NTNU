\chapter{Datainnsamling}
Datainnsamlingen er kritisk til resultatet av oppgaven. Målet i denne fasen er å samle inn et så bredt aspekt av informasjon som mulig gjennom et par mulige verktøy og teknikker beskrevet i boka om rotårsaksanalyse\cite{RCA}. Vi har valgt å bruke spørreundersøkelser som et verktøy for å hente inn informasjon om dette caset fordi vi skal hente inn informasjon fra et stort antall studenter som bor i studentbyene SiT administrerer.

\section{Elektronisk spørreundersøkelse}
Det finnes i hovedsak to forskjellige undersøkelsestyper, kvantitative og kvalitative spørreundersøkelser. Kvalitative undersøkelser går ut på å spørre utvalgte personer, og er ofte mye mer detaljerte og samler svar av høyere kvalitet. Kvantitative undersøkelser fungerer motsatt i at det er fokus på mange tilbakemeldinger slik at en kan senke usikkerhet knyttet til svarkvalitet. \cite{} I vår situasjon har vi valgt kvantitativ undersøkelse på bakgrunn av et par faktorer. For det første ønsker vi at undersøkelsen skal være helt anonym, siden spørsmålene omhandler potensielle lovbrudd. For det andre er målgruppen et stort antall personer, så det kan være nyttig å samle inn data fra så mange av de som mulig.

\subsection{Ønsket utbytte}
Det vi ønsker å få ut av spørreundersøkelsen er data på utvalgte spørsmål vi mener er relevante for oppgaven. Denne fasen er utelukkende for å innhente informasjon, bearbeiding av informasjonen skjer i neste fase. Spørsmålene er utarbeidet for å utforske hvorfor studenter som bor i studentbyer laster ned opphavsrettsbeskyttet materiale, som blant annet inkluderer undersøkelse av økonomiske perspektiver og tilgjengelighet på tjenester. I tillegg ønsker vi også innsikt i hvordan dette kan stoppes. 

Hypotesen vi går inn i undersøkelsen med er at folk laster ned opphavsrettsbeskyttet materiale fordi det er lett tilgjengelig, tilknyttet liten til ingen kostnad og ikke minst fordi det er svært lav risiko for represalier.

\subsection{Gjennomførelse}
Prosessen startet ved å utforme spørreundersøkelsen. En god undersøkelse vil alltid kreve kartlegging av demografi, og i vår undersøkelse valgte vi å kartlegge studentby, kjønn, alder og fakultet. Kjønn og alder er ganske selvforklarende, mens studentby ble valgt på bakgrunn av at Kallerud har mye raskere nedlasting- og opplastingshastighet enn de andre stedene. Vi anså også at det ville være forskjell på hvor mange som laster ned mellom for eksempel informatikkstudiene og helsestudiene. Videre ble resultatene i de foregående fasene brukt til å utforme spørsmålene. Spørreundersøkelsen inkluderer spørsmål om hvor godt en rekke påstander stemmer for den enkelte der respondentene svarer på en likert-skala fra 1-5, der 1 er i liten grad og 5 er i stor grad. Likert-skala ble valgt fordi det er en anerkjent måte å få inn kvantitative svar på hvor enige personer er med en påstand. Samtidig kan man enkelt sammeligne forholdene mellom svarene på de forskjellige påstandene ved hjelp av statistisk analyse. Til slutt inkluderes spørreundersøkelsen et spesielt viktig spørsmål om hva som skal til for at personen stopper med ulovlig nedlasting. Dette er et frisvar der vi kommer til å analysere individuelle svar hver for seg. Spørreundersøkelsen kan leses i sin helhet i \hyperref[undersokelse]{vedlegg A}.
\newline
Det er alltid en stor utfordring å finne nok respondenter til spørreundersøkelser. Vi setter en del krav til antall respondenter og utfører en rekke tiltak for å oppnå nok besvarelser, slik at undersøkelsen kan si noe om rotårsaken med relativt høy sikkerhet. Det er et krav å få minst 30 besvarelser som hadde lastet ned opphavsrettsbeskyttet materiale mens de har bodd i hybelen. Videre er det også ønskelig med relativ likhet i antall respondenter mellom de ulike fakultetene og studentbyene. Det hadde vært ideelt med minst 30 respondenter i hver kategori her også, men det er ønsketenkning i denne sammenhengen. Under prosessen ble også totalt antall beboere fra alle studentbyene i SiT bolig kartlagt. Boligtorget ga oss innbyggertallene fra hver studentby, og vi regnet oss frem til totalt 522 beboere. Det er viktig å presisere at det kan være usikkerheter knyttet til disse tallene, siden det kan hende ikke alle boligene har en beboer.

Siden spørreundersøkelsen er elektronisk var et av de første tiltakene som ble gjennomført å spre den på relevante facebook-sider. Et av prosjektgruppens medlemmer jobber på Studenthuset her på Gjøvik, og fikk spørreundersøkelsen delt på deres facebook-side. Senere i prosessen ble også undersøkelsen delt på facebook-siden til linjeforeningen INGa og klassesidene til sykepleierne og webutvikling. Undersøkelsen ble også delt gjennom venner og bekjente; disse var for det meste informatikkstudenter. I tillegg ble det laget en plakat som ble hengt opp på oppslagstavler på skolen og i vaskeriene i de ulike studentbyene, og mange ble også plassert i postkassene til SiT boliger. Plakaten finnes i \hyperref[plakat]{Vedlegg B}. 

\subsection{Resultater}
Undersøkelsen endte med 97 svar totalt, dette er 18.6\% av de 522 beboerne i SiT bolig. Av disse var det 34 som svarte at de hadde lastet ned opphavsrettsbeskyttet materiale i hybelen, det er 35\% av de spurte. Videre resultater og analyse av disse blir diskutert i neste fase. 

\subsection{Konklusjon av verktøy}
Konklusjonen er at spørreundersøkelser er en effektiv metode for å samle inn store mengder data. En må likevel være konsekvent på tiltakene som iverksettes for å få inn respondenter. Det er også viktig i kvantitative spørreundersøkelser å få relativ likhet i demografien for å kunne si noe om de ulike kategoriene. Dette kan i noen tilfeller være krevende. En annen ting som er svært sentralt i bruk av verktøyet er utformingen av spørsmålene. Det er vanskelig å legge til spørsmål etter spørreundersøkelsen allerede er utsendt, og hva en da trenger ekstra informasjon må dette supplementeres i en mulig kvalitativ undersøkelse eller liknende. 